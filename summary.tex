{\ss\bf Résumé}

Les préoccupations liées à l'étalement urbain ont été traduites depuis 2000
dans la loi en France.
La principale réponse des pouvoirs publics reste la planification,
à travers les SCoT et les PLU.

L'efficacité de cette réponse est limitée et des leviers complémentaires
doivent être recherchés.
Les observatoires territoriaux sont quant à eux des outils émergents, réclamés
par les collectivités territoriales, qui étayent la prise de décision publique,
dans un contexte de réforme territoriale et de redéfinition du rôle de l'État,
mais peinent encore à se structurer et à apporter les éléments nécessaires
pour orienter l'action locale.

Les indicateurs actuellement disponibles sont contestés et ne facilitent que
partiellement le débat public. L'un des freins qui persistent est la difficulté
de dépasser un débat seulement centré sur la mesure de l'étalement urbain
et de l’artificialisation pour mobiliser une description
des processus territoriaux sur lesquels il est possible d’agir.

La place et la fonction de l'agriculture apparaissent comme des problématiques
structurantes dans le développement durable des territoires,
complémentaires d'autres approches comme celle des trames vertes et bleues,
ce qui nous amène à nous intéresser aux interactions entre ville et agriculture.
Dans quelle mesure ces interactions déterminent-elles la dynamique de changement
d’utilisation du sol et les équilibres qui pourraient ou contenir
l’étalement urbain ou l’orienter vers un meilleur compromis ?

Dans ce projet de recherche, nous proposons de recourir à l'analyse spatiale
pour mobiliser les données disponibles (imagerie satellitaire,
données topographiques, données socio-économiques) et caractériser
qualitativement et quantitativement les interactions entre ville et agriculture,
et de là, modéliser les processus territoriaux conduisant à ou limitant
la consommation d'espaces agricoles et naturels dans les zones soumises à
des changements rapides.

Notre objectif sera d’élaborer une démarche qui aide les acteurs territoriaux
à construire une connaissance partagée des relations entre espace urbain et
agriculture qui éclaire les arbitrages liés à l'utilisation du sol.
La construction de modèles numériques, par exemple sous forme de
graphe d’interactions, doit permettre dans ce contexte de formaliser
la connaissance des processus, d’explorer différents scénarios par la simulation
et de sélectionner des systèmes d'indicateurs utiles et utilisables pour guider
les décisions collectives. La modélisation d’accompagnement sera un outil
privilégié pour valider et enrichir les modèles numériques proposés et d’étudier
comment cette démarche permet aux acteurs d'expliciter réflexivement
les compromis qui sont faits aux différentes échelles.

Notre projet met en avant les apports des outils numériques à la médiation
territoriale à travers la géoprospective. Il s'inscrit plus généralement
dans la perspective du développement des observatoires territoriaux
à l’échelle locale, régionale et nationale.

\blank[2*big]

{\ss\bf Mots-clés} :
géoprospective, dynamiques périurbaines, politiques publiques,
ingénierie territoriale, observatoires territoriaux,
cartographie de l'utilisation du sol, interactions nature-territoire-société,
modélisation des processus, méthodes multi-échelles