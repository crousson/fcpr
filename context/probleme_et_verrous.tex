% \page[yes]
\section{Problème}

La ville dense, économe en espace, s'oppose à la ville étalée,
résultat de la périurbanisation, de la recherche d'aménités environnementales
mais aussi d'évitement de déséconomies d'agglomération et d'externalités négatives
liées à la densité.

Si la forme urbaine dense n'offre aucune
garantie en elle-même de durabilité \cite[neuman_compact_2005],
la ville-campagne ou la cité-jardin de \citet{howard_tomorrow_2003}
apparaissent comme une utopie.
Cette utopie urbaine, qui peut être considérée comme le modèle limite
de l'organisation urbaine polycentrique et étalée,
correspond à une vision aspatiale (ou post-spatiale)
de la société, où les contraintes liées à l'espace sont abolies ;
son caractère utopique met en perspective
l'importance de l'espace dans l'organisation urbaine
et la structuration du territoire.

Il semble donc y avoir une contradiction inévitable entre l'objectif affiché de limiter
l'étalement urbain et la nouvelle demande de maintien d'espaces ouverts
et d'usages agricoles à proximité de la ville.

Les indicateurs existants ne suffisent pas à dépasser cette contradiction
et à soutenir le processus adaptatif qui permettrait aux acteurs territoriaux
d'adapter les règles d'urbanisme et les normes environnementales
pour (mieux) réaliser leurs objectifs de développement durable.


\section{Verrous scientifiques et méthodologiques}

En relation avec le problème tel que nous l'avons défini,
nous identifions plusieurs verrous à lever pour améliorer
les connaissances des processus territoriaux
et développer des approches de gestion collective mieux informées :

\startitemize[n]

\item il est encore difficile de passer de cas d'études descriptifs
	à des modèles explicatifs, en reliant les dynamiques d'occupation du sol
	aux processus de décision et
	au système de règles qui gouverne le système territorial ;

\item afin de mener des méta-analyses qui permettraient de mieux comprendre
	les dynamiques territoriales et comment les acteurs peuvent faire évoluer
	les règles de gouvernance,
	il est nécessaire d'accumuler davantage de résultats de cas d'études
	en les organisant sous forme de données plus facilement comparables ;

\item l'information utile et utilisable n'est pas toujours accessible,
	soit parce que certaines données importantes ne sont pas disponibles,
	soit parce que les données existantes ne sont pas utilisables
	en tant que telles et qu'il est nécessaire de concevoir
	des services informationnels adaptés aux besoins des acteurs ;

\item les approches quantitatives, basées sur des indicateurs, permettent de donner
	une vue générale des dynamiques territoriales mais restent,
	en l'absence de modélisation, assez pauvres en informations.
	La modélisation, en explicitant les relations sémantiques et causales
	entre les composantes du système territorial, permettrait de déduire
	des systèmes d'indicateur plus pertinents.

% \item sciences sociales vs. sciences écologiques
%  Scientific knowledge is needed to enhance ef-
% forts to sustain SESs, but the ecological and social
% sciences have developed independently and do not
% combine easily (Ostrom, 2009)

\stopitemize

Si les données sur l'occupation du sol sont de plus en plus abondantes,
les bases de données administratives ne sont pas conçues pour
suivre les changements et sont largement incomplètes en zone périurbaine à
l'échelle des \SCoT\ et des \PLUi\ :
d'une part, l'artificialisation diffuse et le développement pavillonnaire
avec des extensions inférieures à 25 ha en surface sont majoritairement
comptabilisées comme espaces agricoles par Corine Land Cover \cite[aubry_les_2015],
et d'autre part, les systèmes agricoles atypiques caractéristiques de ces zones
sortent souvent du cadre de la \PAC\ et sont donc mal
observées par le registre parcellaire graphique (\RPG).