% \page[yes]
\section[probleme]{Problème}

Il semble y avoir une contradiction inévitable entre l'objectif affiché de limiter
l'étalement urbain, érigé en nouvelles normes environnementales,
et les nouvelles demandes de maintien d'espaces ouverts
et d'usages agricoles à proximité de la ville.

Cette contradiction est multiple :

\startitemize[a]

\item la ville dense, économe en espace, s'oppose à la ville étalée,
	résultat de la périurbanisation, de la recherche d'aménités environnementales
	mais aussi d'évitement de déséconomies d'agglomération et d'externalités négatives
	liées à la densité ;

\item si la profession agricole appelle collectivement à la protection des sols agricoles
	et du foncier, les agriculteurs et les propriétaires sont individuellement
	incités à céder leurs terrains au profit de l'urbanisation et
	à participer à la spéculation foncière ;

\item bien que la crise économique de 2008 ait eu pour conséquence
	un ralentissement de l'étalement urbain,
	le renchérissement prévisible des coûts de l'énergie et de transport
	mais aussi l'évolution des modes de vie avec l'avènement de la société numérique
	pourrait limiter l'attractivité des grands centres urbains
	au profit des centres secondaires, et favoriser paradoxalement des formes urbaines
	plus étalées ;

\item les contraintes budgétaires de plus en plus fortes des pouvoirs publics
	et la nécessité de trouver un équilibre de court terme entre politiques économiques d'une part,
	et politiques sociales et environnementales d'autre part, limitent considérablement
	leurs possibilités d'intervention, tandis que, au delà des discours de principes,
	certains acteurs institutionnels, des communes par exemple,
	ne jouent tout simplement pas le jeu.

\stopitemize

Il ne faut pas négliger non plus que le sol n'est pas une ressource renouvelable :
sa transformation irréversible au profit de l'urbanisation peut apparaître comme une
fatalité qui réduit à peu de chose la motivation de certains acteurs à lutter contre.

Les indicateurs d’occupation du sol et de suivi de l’étalement urbain existants
ne suffisent pas à dépasser ces contradictions.
Ils sont utiles dans la mesure où ils nourrissent le « débat sur les enjeux du
territoire, sur les questions et les difficultés qu’ils soulèvent et la façon
d’y répondre » \cite[briquel_indicateurs_2013].
Ces indicateurs doivent avoir pour but de soutenir le processus adaptatif
qui permettrait aux acteurs territoriaux
de repérer les {\it passagers clandestins} (soit les acteurs qui ne jouent pas le jeu),
d'identifier les incohérences qui résultent de décisions insufisamment coordonnées,
et de se donner la capacité de faire évoluer les règles d'urbanisme et les normes environnementales
en fonction des résultats obtenus pour (mieux) réaliser leurs objectifs de développement durable.

Pour ce faire, il est nécessaire de disposer de davantage de moyens de suivi
des changements et d'évaluation des règles en vigueur,
mais aussi de mieux partager l'information,
car plus les arbitrages se font de manière discrétionnaire,
en évitant le débat public ou en se soustrayant au regard
de la société civile, plus les intérêts particuliers
peuvent avoir d'influence.


\section[verrous]{Verrous scientifiques et méthodologiques}

En relation avec le problème tel que nous l'avons défini,
nous identifions plusieurs verrous à lever pour améliorer
les connaissances des processus territoriaux
et développer des approches de gestion collective mieux informées :

\startitemize[n]

\item il est encore difficile de passer de cas d'études descriptifs
	à des modèles explicatifs, en reliant les dynamiques d'occupation du sol
	aux processus de décision et
	au système de règles qui gouverne le système territorial ;

\item afin de mener des méta-analyses qui permettraient de mieux comprendre
	les dynamiques territoriales et comment les acteurs peuvent faire évoluer
	les règles de gouvernance,
	il est nécessaire d'accumuler davantage de résultats de cas d'études
	en les organisant sous forme de données plus facilement comparables ;

\item l'information utile et utilisable n'est pas toujours accessible,
	soit parce que certaines données importantes ne sont pas disponibles
	ou faciles d'accès (données sur les transactions foncières et le prix du foncier par exemple),
	soit parce que les données existantes ne sont pas utilisables
	en tant que telles et qu'il est nécessaire de concevoir
	des services informationnels adaptés aux besoins des acteurs ;

\item les approches quantitatives, basées sur des indicateurs, permettent de donner
	une vue générale des dynamiques territoriales mais restent,
	en l'absence de modélisation, assez pauvres en informations.
	La modélisation, en explicitant les relations sémantiques et causales
	entre les composantes du système territorial, permettrait de déduire
	des systèmes d'indicateur plus pertinents.

% \item sciences sociales vs. sciences écologiques
%  Scientific knowledge is needed to enhance ef-
% forts to sustain SESs, but the ecological and social
% sciences have developed independently and do not
% combine easily (Ostrom, 2009)

\stopitemize

Si les données sur l'occupation du sol sont de plus en plus abondantes,
les bases de données administratives ne sont pas conçues pour
suivre les changements et sont largement incomplètes en zone périurbaine à
l'échelle des {\SCoT} et des {\PLUi} :
d'une part, l'artificialisation diffuse et le développement pavillonnaire
avec des extensions inférieures à 25 ha en surface sont majoritairement
comptabilisées comme espaces agricoles par Corine Land Cover \cite[aubry_les_2015],
et d'autre part, les systèmes agricoles atypiques caractéristiques de ces zones
sortent souvent du cadre de la {\PAC} et sont donc mal
observées par le registre parcellaire graphique ({\RPG}).

La difficulté à qualifier les fonctions remplies
par les espaces agricoles et naturels, et parce qu'il est aussi beaucoup plus facile
d'observer l'occupation du sol, conduit fréquemment à confondre occupation et usage du sol.
Occupation et usage du sol sont logiquement corrélés,
mais la distinction devient importante dès lors qu'on s'intéresse
à la relation fonctionnelle entre plusieurs espaces.