\section{Problème}

Il y a une contradiction (inévitable ?) entre l'objectif affiché de limiter
l'étalement urbain (modèle de la ville dense économe en espaces)
et la nouvelle demande sociétale pour le maintien d'une agriculture de proximité
et d'espaces naturels et agricoles {\it au sein} de la ville
(modèle de la ville étalée, polycentrique,
qui renvoie à l'utopie urbaine Howardienne).

Le sol, qui se distingue tout en se confondant avec le foncier
selon le point de vue adopté, est de plus en plus reconnu
comme un bien commun, c'est-à-dire une ressource territoriale
dont les choix d'utilisation et d'affectation devrait être une affaire collective,
tandis que la problématique foncière reste principalement déterminée
par un régime de propriété privée difficilement encadré par les
pouvoirs publics.


\section{Verrous scientifiques et méthodologiques}

En relation avec le problème tel que nous l'avons défini,
nous identifions plusieurs verrous qu'il conviendrait
de lever pour améliorer les connaissances des processus territoriaux
et développer des approches de gestion collective mieux informées :

\startitemize[n]

\item il est encore difficile passer de cas d'études descriptifs
	à des modèles explicatifs, en reliant les dynamiques d'occupation du sol
	(ce qui est observé/observable) aux processus de décision et
	au système de règles qui gouverne le système territorial ;

\item afin de mener des méta-analyses qui permettront de mieux comprendre
	les dynamiques territoriales et comment les acteurs peuvent faire évoluer
	ces dynamiques et les règles de gouvernance,
	il est nécessaire d'accumuler davantage de résultats de cas d'études
	en les organisant sous forme de données plus facilement comparables ;

\item les indicateurs existants, parce qu'ils sont controversés,
	ne suffisent pas à soutenir le processus adaptatif
	qui permettrait aux acteurs territoriaux
	d'adpater les règles pour (mieux) réaliser leurs objectifs ;

	la modélisation, en explicitant les relations sémantiques et causales
	entre les composantes du système territorial, permet de déduire
	des systèmes d'indicateur plus pertinents.

\item l'information utile et utilisable n'est pas toujours accessible,
	soit parce que certaines données importantes ne sont pas disponibles,
	soit parce que les données existantes ne sont pas utilisables
	en tant que telles et qu'il est nécessaire de concevoir
	des services informationnels adaptés aux besoins des acteurs.

% \item sciences sociales vs. sciences écologiques
%  Scientific knowledge is needed to enhance ef-
% forts to sustain SESs, but the ecological and social
% sciences have developed independently and do not
% combine easily (Ostrom, 2009)

\stopitemize

Si les données sur l'occupation du sol sont de plus en plus abondantes,
les bases de données administratives ne sont pas conçues pour
suivre les changements et sont largement incomplètes en zone périurbaine à
l'échelle à laquelle nous proposons de travailler :
d'une part, l'artificialisation diffuse et le développement pavillonnaire
avec des extensions inférieures à 25 ha en surface sont majoritairement
comptabilisées comme espaces agricoles par Corine Land Cover \cite[aubry_les_2015],
et d'autre part, les systèmes agricoles atypiques caractéristiques de ces zones
sortent souvent du cadre de la PAC et sont donc mal
observées par le Registre parcellaire graphique (RPG).