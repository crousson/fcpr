\page[yes]
\section{Cadres d'analyse
	des changements d'occupation du sol
	et de l'étalement urbain}

Le recours à un cadre heuristique facilite la lecture
et l'analyse des systèmes complexes.

Les sciences écologiques et les sciences sociales se sont développées
indépendamment jusqu'à encore récemment,
et malgré un appel récurrent pour davantage de transdisciplinarité,
ne se rencontrent pas facilement.

\startitemize

\item	Les sciences de la Terre et de l'environnement s'intéressent aux changements d'occupation
	du sol du point de vue de la biodiversité et de l'écologie du paysage
	et recourent fréquemment aux concepts de l'étude d'impact pour analyser les processus de changement,
	et font volontiers référence au modèle \DPSIR\ (\infull{DPSIR}) qui est largement adopté
	au niveau institutionnel et gestionnaire.

	Ce cadre heuristique offre un support robuste pour penser la relation entre environnement,
	société et politiques publiques \cite[tscherning_does_2012].
	Certains reprochent cependant à ce modèle une vision mécaniste
	et « occidentale » de la relation homme-nature.

\item \citet{ostrom_background_2011} propose la notion de système socioécologique
	qui met en exergue le rôle des institutions et les interactions multi-échelles
	dans la gestion des ressources naturelles.
	Le cadre heuristique \IAD\ (\infull{IAD}) formalise ces concepts
	pour analyser les interactions entre les acteurs, le système de gouvernance
	et l'environnement biophysique. Cette approche s'intéresse
	en particulier aux conditions dans lequelles les acteurs réussissent
	à s'auto-organiser et à faire évoluer les règles de gouvernance
	pour améliorer la gestion du système de ressources qu'ils ont
	collectivement intérêt à préserver.

\stopitemize

Ces cadres heuristiques ne doivent pas être compris comme des modèles
exprimant des relations causales, mais comme une conceptualisation très générale
des composantes en interaction et de leurs relations sémantiques.
Ils servent en quelque sort de guide ou de pense-bête pour aborder
la complexité du système territorial, selon le point de vue
qu'on souhaite privilégier.