% \page[yes]
\subsection{Cadres d'analyse
	des changements d'occupation du sol
	et de l'étalement urbain}

Le recours à un cadre heuristique facilite la lecture
et l'analyse des systèmes complexes.
Nous retenons a priori deux cadres heuristiques complémentaires,
fréquemment mis en avant dans la littérature sur les changements
d'occupation du sol :

% Les sciences écologiques et les sciences sociales se sont développées
% indépendamment jusqu'à encore récemment,
% et malgré un appel récurrent pour davantage de transdisciplinarité,
% ne se rencontrent pas facilement.

\startitemize

\item les études environnementales des changements d'occupation du sol
	recourent fréquemment aux concepts de l'étude d'impact pour analyser les processus de changement,
	et font volontiers référence au modèle \DPSIR\ (\infull{DPSIR}), largement adopté
	au niveau institutionnel et gestionnaire. 
	Ce cadre heuristique offre un support robuste pour penser la relation entre environnement,
	société et politiques publiques \cite[tscherning_does_2012],
	bien que certains reprochent à ce modèle une vision mécaniste
	et occidentale de la relation homme-nature.

\item le cadre heuristique \IAD\ (\infull{IAD}) proposé par \citet{ostrom_background_2011}
	formalise les interactions entre les acteurs, le système de gouvernance
	et l'environnement biophysique à l'intérieur d'un système socioécologique.
	Cette approche met en exergue le rôle des institutions et les interactions multi-échelles
	et s'intéresse en particulier aux conditions dans lesquelles les acteurs réussissent
	à s'auto-organiser et à faire évoluer les règles de gouvernance
	pour améliorer la gestion du système de ressources qu'ils ont
	collectivement intérêt à préserver.

\stopitemize

Ces cadres heuristiques ne doivent pas être compris comme des modèles
exprimant des relations causales, mais comme une conceptualisation très générale
des composantes en interaction et de leurs relations sémantiques.
Ils servent en quelque sorte de guide pour aborder
la complexité du système territorial et choisir différents points de vue
en fonction des questions étudiées.
L'adéquation de l'un et l'autre de ces cadres à la problématique
et aux cas d'études sera examinée pendant la thèse,
en proposant si nécessaire des adaptations.

% Les interactions spatiales entre ville et agriculture
% ne résultent pas seulement de la distance géographique
% mais aussi d'interactions à distance dans un contexte global / local \cite[seto_urban_2012].
% Cette proximité fonctionnelle s'inscrit
% dans la participation à des réseaux (filière, circuit de commercialisation, etc.)
% qui dépassent la seule échelle locale,
% et dépend aussi de facteurs institutionnels.
% De ce point de vue, la notion de circuit court peut s'entendre en terme de transport
% (kilomètres parcourus entre le lieu de production et le consommateur)
% mais aussi en nombre d'intermédiaires.