\subsection{Cadres d'analyse
	des changements d'occupation du sol
	et de l'étalement urbain}

Le recours à un cadre heuristique facilite l'analyse d'un système aussi complexe
que celui dans lequel se développent les dynamiques territoriales.

\startitemize

\item	Les sciences de la Terre et de l'environnement qui s'intéresse aux changements d'occupation
	du sol du point de vue de la biodiversité et de l'écologie du paysage
	recourt fréquemment aux concepts de l'étude d'impact pour analyser les processus de changement,
	et font volontiers référence au modèle \DPSIR\ (\infull{DPSIR}) qui est largement adopté
	au niveau institutionnel et gestionnaire.

	Ce cadre heuristique offre un support robuste pour penser la relation entre environnement,
	société et politiques publiques \cite[tscherning_does_2012].
	Certains reprochent à ce modèle une vision mécaniste
	et « occidentale » de la relation homme-nature.

\item	L'opposition entre organisation monocentrique et polycentrique
	s'entend également du point de vue
	de l'analyse institutionnelle \cite[chanteau_linstitutionnalisme_2013].
	\citet{ostrom_background_2011} propose la notion de système socioécologique
	qui met en exergue le rôle des institutions et les interactions multi-échelles
	dans la gestion des ressources naturelles.
	Le cadre heuristique \IAD\ (\infull{IAD}) formalise ces concepts
	pour analyser les interactions entre les acteurs, les institutions
	et l'environnement biophysique.

\stopitemize

Ces cadres heuristiques ne doivent pas être compris comme des modèles
exprimant des relations causales, mais comme une conceptualisation très générale
des composantes en interactions et de leurs relations sémantiques.
Ils servent en quelque sort de guide ou de pense-bête pour aborder
la complexité du système territorial, selon le point de vue
qu'on souhaite privilégier.

\page[yes]
\subsection{Analyse à travers le cadre heuristique IAD}

\startplacefigure[location=middle,
short={Analyse de l'étalement urbain à travers le cadre IAD},
title={Analyse de l'étalement urbain à travers le cadre IAD.
       Adapté de \citet{ostrom_background_2011} }]
  \externalfigure[../svg/iad_sprawl.svg][width=\textwidth]
\stopplacefigure

% \subject{Variables exogènes}

% \startitemize

% \item Occupation du sol
% \item Système (d'acteurs) territorial
% \item Régles d'urbanisme / Normes environnementales / Règles de gouvernance

% \stopitemize

\subsubject{Système territorial}

\startitemize[packed]

\item réseau d'acteurs
\item relations entre les différents espaces
\item relations entre ville et agriculture

\stopitemize

\subsubject{Acteurs}

Plusieurs échelles d'action et de décision
sont imbriquées les unes dans les autres :

\startcolumns[2]
\startitemize[a,packed]

\item Échelle individuelle :

	\startitemize[packed]
		\item propriétaire du foncier
		\item promoteur immobilier
		\item agriculteur
		\item habitant, citadin, usager
		\item porteur de projet agriculture urbaine / circuit court / \dots
	\stopitemize

\item Échelle locale : commune, \SCoT, inter-\SCoT, agglomération

	\startitemize[packed]

		\item collectivités locales
		\item Agence d'urbanisme
		\item services de l'État
		\item chambres professionnelles
		\item société civile : CoDev, associations, \dots

	\stopitemize

\item échelles méso : département, région

\item échelles globales : national, Europe, Monde

\stopitemize
\stopcolumns

\startcolumns[n=2]
\subsubject{Situations}


\startitemize[packed]

\item Révision des documents d'urbanisme
\item Intervention foncière / exercice du droit de préemption
\item Instruction des permis de construire
\item DUP

\stopitemize

Les CDCEA sont l'une des arènes de l'action,
et permettent à des acteurs qui n'avaient pas
ou plus difficilement la possibilité d'interagir
de se rencontrer pour exprimer leur point de vue
sur les choix d'utilisation et d'affectation du sol.

\column
\subsubject{Résultats}

\startitemize[packed]

\item évolution du zonage dans les documents d'urbanisme (PLU)
\item densification du bâti
\item étalement urbain
\item mesure de protection des sols et du foncier
\item soutien à des usages et à des activités favorables
  au maintien d'espaces agricoles et naturels
\item modification de la fiscalité ?

\stopitemize
\stopcolumns

\subsubject{Critères d'évaluation}

\startcolumns[2]
\startitemize[packed]

\item diagnostic territorial
\item changements d'occupation du sol (à qualifier et à quantifier)
\item évolution des usages (nécessite de mieux appréhender les fonctions
  des différents espaces, en particulier de l'agriculture urbaine)
\item cohérence inter-échelle
\item état des sols, en distinguant en fonction
  des différentes potentialités
\item état des continuités écologiques (TVB)

\stopitemize
\stopcolumns

\setcounter[subsection][4]