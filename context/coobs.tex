\section[coobs]{La démarche Co-Obs}

\startplacefigure[location=middle,title={La démarche Co-Obs}]
  \externalfigure[../svg/coobs.svg][width=0.9\textwidth]
\stopplacefigure

Dans ses réflexions sur les observatoires et les Systèmes d'informations,
\TETIS\ a développé des méthodes où l'ontologie sert à représenter une réalité
à la fois pour définir un modèle des dynamiques en jeu,
un modèle de l'action (un programme ou un projet d'action)
et un modèle de l'observation (qui accompagne l'action et en découle). 

La démarche accompagne et s'insère dans le cycle d'un projet d'observatoire territorial :
identification/diagnostic, programmation/planification; mise en œuvre/suivi; évaluation... 
La co-construction des modèles prépare la définition des besoins en informations et leur production.
Réciproquement, l’information produite par l’observatoire et sa mise en débat au sein de la communauté 
d’acteurs favorisent la production de nouvelles connaissances.