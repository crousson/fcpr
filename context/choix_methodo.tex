% \section
% {Choix méthodologiques}
% {recours à l'analyse spatiale et à la modélisation}


\subsection{Outils de modélisation}

Deux familles de modèles explicitement spatialisés
sont principalement utilisées pour modéliser et simuler
les changements d'occupation et d'utilisation du sol :
d'un côté les automates cellulaires, de l'autre les systèmes multi-agents.

Les modèles basés sur les automates cellulaires ({\CA}), comme le modèle {\sc clue}
\cite[veldkamp_clue_1996, verburg_modeling_2002],
permettent de rendre compte de manière tendancielle des processus d’évolution.
Ces modèles ne permettent pas d'expliciter simplement les processus socioéconomiques,
ce qui limite leur intérêt opérationnel pour l'aménagement
et les politiques publiques.

À la différence des automates cellulaires,
les systèmes multi-agents ({\SMA}) permettent
de mieux comprendre comment les choix individuels et hétérogènes des agents s’agrègent
pour expliquer les dynamiques à l'échelle du système environnemental ou territorial.

\citet{irwin_theory_2001,irwin_towards_2009} utilisent les {\SMA}
pour modéliser le processus d'urbanisation à travers les choix résidentiels
en utilisant la théorie économique des choix discrets
et la préférence hédonique des individus pour différents types de paysage
et de voisinage.
% \citet{ostrom_background_2011} souligne que les chercheurs en sciences sociales
% doivent davantage reconnaître que les comportements individuels
% sont fortement influencés par le contexte dans lequel se situent
% les interactions entre individus, davantage qu'ils ne sont le résultat
% de simples différence inter-individuelles.
La complexité des choix humains, difficilement représentable
dans des modèles analytiques, et l'idée que les comportements individuels
sont fortement influencés par le contexte dans lequel se situent
les interactions entre individus sont parmi les motivations
qui conduisent les chercheurs à utiliser de plus en plus
largement les simulations basées sur les {\SMA}
\cite[janssen_empirically_2006, ostrom_background_2011].
De fait, les {\SMA} sont fréquemment utilisés
comme support de la modélisation d'accompagnement
dans le cadre de la gestion participative des problématiques environnementales
\cite[le_page_agent_based_2013].

La plateforme Ocelet adopte une approche intermédiaire
entre la dynamique des systèmes et les {\SMA}.
Elle permet de simuler les dynamiques territoriales et paysagères
de manière spatialement explicite
en rendant compte des processus et des interactions qui interviennent à différentes échelles.
Cette plateforme se fonde sur le formalisme des graphes d'interactions pour représenter
les relations entre les acteurs et les objets du système territorial
\cite[degenne_design_2009, degenne_approche_2012, castets_integration_2014].
Ce formalisme a l'intérêt en particulier
de pouvoir représenter des processus multi-échelles.

% Nous proposons de recourir d'abord aux {\SMA} pour
% mettre en évidence, avec des hypothèses simplificatrices,
% les effets d'auto-organisation et de couplage
% qui peuvent résulter de la compétition pour l'occupation du sol
% entre plusieurs populations d'agents (citadins / agriculteurs)
% en fonction des tropismes (polarisation) de ces agents et des règles d'urbanisme.
% Puis, à partir de ces premiers résultats,
% nous proposons d'intégrer dans la modélisation les interactions multi-échelles,
% notamment les interactions entre l'échelle individuelle et l'échelle institutionnelle,
% en utilisant la plateforme Ocelet.

Les choix de modélisation seront précisés pendant la thèse,
en cherchant à mettre en relation l'échelle de décision des acteurs individuels
avec l'échelle de la planification territoriale.

% couplage {\SMA} / Ocelet ?