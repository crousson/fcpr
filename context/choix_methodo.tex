\section
{Choix méthodologiques :
recours à l'analyse spatiale et à la modélisation}

\subsection{Justification des apports de la modélisation}

La consommation d'espaces naturels et agricoles
est un processus complexe qui fait intervenir de nombreux facteurs,
avec des interactions multi-scalaires.

Les approches quantitatives, basées sur des indicateurs, permettent de donner
une vue générale des dynamiques territoriales mais restent,
en l'absence de modélisation, assez pauvres en informations.

Le modèle est un objet intermédiaire
qui aide à la lecture de la complexité du monde réel.
La modélisation nous paraît indispensable pour dépasser
une approche descriptive des dynamiques territoriales
et expliquer les processus en œuvre.
Le premier enjeu de la modélisation est d'expliciter les relations sémantiques
et causales entre les variables représentées par les indicateurs.

L'intégration des indicateurs dans une approche globale
permet de construire des systèmes d'indicateurs
plus pertinents pour le diagnostic territorial
et l'évaluation des politiques publiques.

L'apport du modèle est aussi :

\startitemize[packed]

\item de formuler et de tester des hypothèses sur des phénomènes
  qui sont difficilement observables directement
\item de permettre des changements d'échelle et de mettre en rapport dynamiques globales
  et changements locaux
\item d'identifier des régularités pour formuler des lois
  à partir d'observations empiriques
\item de comparer des situations différentes en explicitant
  ce qui est générique et en le distinguant du spécifique.

\stopitemize

Cadre heuristique, théorie et modèle sont liés et leur relation va du général au particulier.
Les cadres heuristiques « correspondent à la forme la plus générale d’analyse théorique ».
Les théories permettent de formuler des hypothèses de travail sur les relations entre
les éléments pertinents du cadre heuristique, tandis que le modèle permet
de tester ces hypothèses dans une situation réelle \cite[ostrom_background_2011].

Le modèle a donc une portée opérationnelle et spécifique.
C'est un dispositif expérimental qui rend possible l'exploration
de scénarios et permet d'apprendre à partir des écarts constatés entre
ce qui est prédit par le modèle et ce qui est observé.

La modélisation d’accompagnement, ou « companion modelling »,
exploite cette acception opérationnelle du modèle : c'est une méthode participative
basée sur les jeux de rôles et la simulation, utilisés pour servir de support
au dialogue entre les acteurs ;
elle encourage l’apprentissage collectif et organisationnel et
permet ainsi une construction itérative et adaptative des décisions
\cite[etienne_modelisation_2012].


\subsection{Outils de modélisation}

Deux familles de modèles explicitement spatialisés
sont principalement utilisées pour modéliser et simuler
les changements d'occupation et d'utilisation du sol :

\startitemize[packed]

\item les automates cellulaires ;
\item les systèmes multi-agents.

\stopitemize

Les modèles basés sur les automates cellulaires, comme le modèle CLUE
\cite[veldkamp_clue_1996, verburg_modeling_2002],
permettent de rendre compte de manière tendancielle des processus d’évolution.
Ces modèles ne permettent pas d'expliciter simplement les processus socioéconomiques,
ce qui limite leur intérêt opérationnel pour l'aménagement
et les politiques publiques.

Les systèmes multi-agents (SMA) ont l'intérêt de mettre en évidence par la simulation
les propriétés macroscopiques d'un système à partir des éléments microscopiques
qui le composent. Les SMA permettent par exemple
de mieux comprendre comment les choix individuels des agents s’agrègent
pour expliquer les dynamiques à l'échelle du système environnemental ou territorial.
Les SMA sont fréquemment utilisés
comme support de la modélisation d'accompagnement
dans le cadre de la gestion participative des problématiques environnementales
\cite[le_page_agent_based_2013].

À mi-chemin entre les automates cellulaires et les SMA,
la plateforme Ocelet a été développée pour simuler les dynamiques territoriales et paysagères
en rendant compte explicitement des processus et des interactions.
Cette plateforme se fonde sur le formalisme des graphes pour représenter
les relations entre les acteurs et les objets du système territorial
\cite[degenne_design_2009, degenne_approche_2012, castets_integration_2014].
Cette représentation permet de modéliser des processus multi-échelles.

Nous proposons de recourir d'abord aux SMA pour
mettre en évidence, avec des hypothèses simplificatrices,
les effets d'auto-organisation et de couplage
qui peuvent résulter de la compétition pour l'occupation du sol
entre deux populations d'agents (citadins / agriculteurs)
en fonction des tropismes (polarisation) de ces agents.
Puis, à partir de ces premiers résultats,
d'évoluer vers des modèles plus sophistiqués qui rendent compte
des interactions multi-échelles, notamment à travers les mécanismes institutionnels,
en utilisant la plateforme Ocelet.

Les choix de modélisation seront précisés pendant la thèse.

% couplage SMA / Ocelet ?

\subsection{Démarche proposée}

Notre démarche de modélisation suivra la méthode générale synthétisée par \citet{magliocca_metastudies_2015}
à partir de la littérature sur la modélisation des changements d'occupation du sol.

\startplacefigure[location=middle,title={Processus de modélisation}]
  \externalfigure[../svg/magliocca_methodo.svg][width=0.9\textwidth]
\stopplacefigure

Cette méthode divise le processus de modélisation en 5 étapes :

\startitemize[n]

\item le point de départ de tout processus de modélisation
   est la question de recherche et sa problématisation,
   qui définit le phénomène spatial auquel on s'intéresse
   ainsi que le contexte et l'échelle à laquelle on souhaite l'étudier,
   son éventuel rapport avec des processus d'échelle plus large ;

\item la description du problème permet d'identifier les limites du système,
   ses composants (acteurs, variables, processus)
   et les possibles relations qu'ils entretiennent.
   La littérature apporte
   un appui important pour identifier les composants et
   les décrire en même temps qu'elle peut apporter
   des éléments de théorie pour les mettre en relation.
   À cette étape, la cartographie à dire d'acteurs peut
   servir également à identifier ces éléments et à conceptualiser
   le système étudié.

\item le modèle conceptuel est ensuite traduit en modèle dynamique
   et implémenté dans un programme informatique
   en sélectionnant une représentation appropriée ;
   l'utilisation d'une plateforme de modélisation
   adaptée au type de modèle que l'on souhaite instancier
   facilite la réalisation de cette étape ;

\item le modèle est ensuite calibré et validé à l'aide des données
   du terrain d'études ; cette étape permet de tester la capacité
   du modèle à rendre compte des observations ainsi
   que sa sensibilité aux données en entrée ;

\item enfin, le modèle peut être exploité pour
   tester des hypothèses complémentaires, explorer
   différents scénarios ou être confronté à
   ses utilisateurs potentiels. Cette étape d'expérimentation
   permet de critiquer le modèle et de raffiner l'analyse du
   problème initial.

\stopitemize