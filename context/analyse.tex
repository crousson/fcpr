\enableregime[utf]
\input layout
\input metadata

\section{Problème}

Il y a une contradiction (inévitable ?) entre l'objectif affiché de limiter
l'étalement urbain (modèle de la ville dense économe en espaces)
et la nouvelle demande sociétale pour le maintien d'une agriculture de proximité
et d'espaces naturels et agricoles {\it au sein} de la ville (modèle de la ville polycentrique, étalée
qui renvoie à l'utopie urbaine Howardienne).

Foncier vs. Sol
= bien privé / bien commun
= ressource territoriale commune ?

\section{Analyse à travers le cadre heuristique IAD}

\startplacefigure[location=middle,
short={Analyse de l'étalement urbain à travers le cadre IAD},
title={Analyse de l'étalement urbain à travers le cadre IAD (d'après \cite[ostrom_background_2011])}]
  \externalfigure[../svg/iad_sprawl.svg][width=\textwidth]
\stopplacefigure


% \startcolumns[n=2]

% \subject{Variables exogènes}

% \startitemize

% \item Occupation du sol
% \item Système (d'acteurs) territorial
% \item Régles d'urbanisme / Normes environnementales / Règles de gouvernance

% \stopitemize

\subject{Système territorial}

\startitemize[packed]

\item réseau d'acteurs
\item relations entre les différents espaces
\item relations entre ville et agriculture

\stopitemize

\page[yes]

\subject{Acteurs}

Plusieurs échelles d'action et de décision
sont imbriquées les unes dans les autres :

\startcolumns[2]
\startitemize[a,packed]

\item Échelle individuelle :

	\startitemize[packed]
		\item propriétaire du foncier
		\item promoteur immobilier
		\item agriculteur
		\item habitant, citadin, usager
		\item porteur de projet agriculture urbaine / circuit court / \dots
	\stopitemize

\column
\item Échelle locale : commune, SCoT, inter-SCoT, agglomération

	\startitemize[packed]

		\item collectivités locales
		\item Agence d'urbanisme
		\item services de l'État
		\item chambres professionnelles
		\item société civile : CoDev, associations, \dots

	\stopitemize

\item échelles méso : département, région

\item échelles globales : national, Europe, Monde

\stopitemize
\stopcolumns

\startcolumns[n=2]
\subject{Situations}


\startitemize[packed]

\item Révision des documents d'urbanisme
\item Intervention foncière / exercice du droit de préemption
\item Instruction des permis de construire
\item DUP

\stopitemize

Les CDCEA sont l'une des arènes de l'action.

\column
\subject{Résultats}

\startitemize[packed]

\item évolution du zonage dans les documents d'urbanisme (PLU)
\item densification du bâti
\item étalement urbain
\item mesure de protection des sols et du foncier
\item soutien à des usages et à des activités favorables
  au maintien d'espaces agricoles et naturels
\item modification de la fiscalité ?

\stopitemize
\stopcolumns

\subject{Critères d'évaluation}

\startitemize[packed]

\item diagnostic territorial
\item changements d'occupation du sol (à qualifier et à quantifier)
\item évolution des usages (nécessite de mieux appréhender les fonctions
  des différents espaces, en particulier de l'agriculture urbaine)
\item cohérence inter-échelle
\item état des sols, en distinguant en fonction
  des différentes potentialités
\item état des continuités écologiques (TVB)

\stopitemize

% \stopcolumns

\page[yes]

\section{Questions}

\startitemize[n]

\item 	Comment les acteurs peuvent-ils
	volontairement modifier les règles du jeu,
	de manière incrémentale / adaptative,
	en fonction des résultats obtenus ?
	
	De quelles informations ont-ils besoin
	pour soutenir ce processus adaptatif ?
	% -> (R -> D,P; S,I -> D,P)

\item	Comment le résultat des interactions
	entre acteurs modifie-t-il le dynamique territoriale
	d'étalement urbain ?
	% -> (R -> S)
	
	Y a-t-il des effets de couplage (par ex. la protection d'espaces à un endroit
	entraîne un étalement augmenté à un autre endroit du système territorial) ?

\item	Comment les relations existantes entre ville et agriculture
	influencent-elles le processus de décision ?
	% -> (S,I->R; R->S,I)
	% -> à modéliser avec Ocelet ?

\stopitemize

\section{Verrous scientifiques et méthodologiques}

\startitemize[n]

\item Difficulté à passer de cas d'études descriptifs
	à des modèles explicatifs, en reliant les dynamiques d'occupation du sol
	(ce qui est observé/observable) aux processus de décision et
	au système de règles qui gouverne le système territorial

\item Besoin d'accumuler des résultats de cas d'études
	en les organisant sous forme de données comparables
	pour mener des méta-analyses qui permettront de mieux comprendre

\item Les indicateurs existants ne suffisent pas à soutenir
	le processus adaptatif qui permettrait aux acteurs territoriaux
	d'adpater les règles pour (mieux) réaliser leurs objectifs.

	Besoin de recourir à la modélisation pour déduire
	des systèmes d'indicateur plus pertinents (cf. démarche Co-Obs).

\item Il y a encore des difficultés d'accès à l'information,
	soit parce que certaines données utiles ne sont pas disponibles,
	mais aussi parce que les données disponibles ne sont pas utilisables
	en tant que telles et il est nécessaire de concevoir
	des services informationnels adaptés aux besoins des acteurs.

\stopitemize