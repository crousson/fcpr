\section{Hypothèses thématiques}

\startitemize[n]

\item	Si la proximité urbaine
	et l'étalement urbain ont un effet sur les usages agricoles
	périurbains, la localisation et l'adaptation de ces usages
	ont également un effet sur l'étalement urbain et les formes urbaines.
	Dans les deux sens de cette relation, les effets ne sont pas univoques ;
	ils sont tantôt positifs ou négatifs par rapport à l'objectif de bonne gouvernance
	du foncier et d'un développement urbain durable.

	Plusieurs sous-hypothèses peuvent être avancées
	pour expliquer ces interactions et les processus divergents
	qui en résultent :

	\startitemize[a,packed]

	\item	l'étalement urbain fragilise les usages agricoles
		en créant des contraintes d'exploitation ou
		en soustrayant le foncier au profit de l'urbanisation
		sous l'effet de la spéculation notamment ;
		la fragilisation de la fonction agricole
		due à la proximité urbaine favorise en retour l'étalement urbain
		en libérant le foncier ;

	\item	l'adaptation des productions agricoles d'une forme
		découplée de la consommation locale vers d'autres
		formes qui tirent parti de la proximité urbaine
		renforce la place de l'agriculture dans l'espace périurbain
		et contribue à contenir l'étalement urbain ;

	\item	l'étalement urbain et ses formes discontinues sont renforcés
		en présence d'aménités ou d'externalités négatives créées par les activités
		agricoles comme par les activités industrielles
		ou par le voisinage résidentiel ;

	\item	les politiques de protection des espaces agricoles, naturels
		et forestiers sont susceptibles de participer
		à la limitation de la dispersion morphologique de l'habitat
		mais aussi de favoriser l'étalement urbain

	\stopitemize

\item	Nous faisons l'hypothèse d'un couplage entre plusieurs processus territoriaux
	impliqués dans la consommation d'espaces naturels et agricoles ;
	par exemple, le maintien d'espaces ouverts
	interstitiels et la densité des parcelles bâties sont possiblement couplés.
	Ces couplages, tant qu'ils ne sont pas explicités,
	rendent difficiles la conception d'un modèle d'actions efficace.

\stopitemize

Des exercices de prospective (Westhoek et al., 2006; Gauvrit et Mora, 2009)
identifient plusieurs facteurs à prendre en compte pour expliquer
les dynamiques péri-urbaines futures et l'évolution du rapport
entre villes et campagnes :

\startitemize

\item	le renchérissement du coût des transports,
	comme impact de l'évolution du coût des énergies carbonées
	ou conséquences des politiques d'atténuation du changement climatique ;

\item 	les conséquences d'une moindre croissance économique ;

\item 	la transition vers une économie numérique
	qui diminue la dépendance (économique, culturelle, etc.)
	au centre (au grand pôle urbain) ;

\item	l'évolution de la mobilité et des modes de consommation
	à travers l'évolution des modes de vie ;

\item	l'évolution des politiques publiques agricoles
	et des échanges économiques globalisés.

\stopitemize