\section{Questions de recherche et objectifs}

L'adaptation au changement climatique et la nécessité
de mieux valoriser les ressources territoriales sur le plan économique
et social dans un contexte de crises récurrentes
(économique, sanitaire, foncière, etc.)
remettent en perspective les relations entre ville et agriculture.

Qu'est-ce que nous apprennent les changements d'occupation du sol
sur les processus d'adaptation des usages agricoles sous l'influence urbaine
et sur les interactions ville-agriculture~?

Y a-t-il des configurations spatiales plus favorables
que d'autres au maintien de l'agriculture~?
Comment ces configurations influencent-elles les fonctions
assignées à l'agriculture et son adaptation aux nouvelles demandes sociétales~?

Notre hypothèse centrale est que si la proximité urbaine
et l'étalement urbain ont un effet sur les usages agricoles
périurbains, la localisation et l'adaptation de ces usages
ont également un effet sur l'étalement urbain et les formes urbaines.
Dans les deux sens de cette relation, les effets ne sont pas univoques
et sont tantôt positifs ou négatifs par rapport à l'objectif de bonne gouvernance
du foncier et du développement urbain.

Plusieurs sous-hypothèses peuvent être avancées
pour expliquer ces interactions et les processus divergents
qui en résultent :

\startitemize

\item l'étalement urbain fragilise les usages agricoles
     en créant des contraintes d'exploitation ou
     en soustrayant le foncier au profit de l'urbanisation
     sous l'effet de la spéculation notamment ;
     la fragilisation de la fonction agricole
     due à la proximité urbaine favorise en retour l'étalement urbain
     en libérant le foncier ;

\item l'adaptation des productions agricoles d'une forme
     découplée de la consommation locale vers d'autres
     formes qui tirent parti de la proximité urbaine
     renforce la place de l'agriculture dans l'espace périurbain
     et contribue à contenir l'étalement urbain ;

\item l'étalement urbain et ses formes discontinues sont renforcés
     en présence d'aménités ou désaménités créées par les activités
     agricoles comme par les activités industrielles
     ou par le voisinage résidentiel ;

\item les politiques de protection des espaces agricoles, naturels
     et forestiers sont susceptibles de participer
     à la limitation de la dispersion morphologique de l'habitat
     mais aussi de favoriser l'étalement urbain.

\stopitemize

Notre objectif est d'identifier et de différencier ces interactions
en mettant statistiquement en relation les formes spatiales
de l'agriculture et de la ville entre elles
et avec leurs déterminants socio-économiques.

Cette caractérisation des processus étudiés doit permettre
de les représenter dans des modèles dynamiques explicitement
spatialisés et servir de support à l'exploration de scénarios
et à des exercices de prospective territoriale.


\subsection{Choix de l'échelle de travail}

Le choix de l'échelle d'analyse et de modélisation est guidé
par les considérations suivantes :

\startitemize

\item cette échelle doit être proche de l'échelle de décision
  pour permettre de représenter les processus sur lesquels il est possible
  d'agir ;

\item elle doit permettre une mise en évidence des incohérences
  qui résultent de décisions prises
  à des échelles inférieures.

\stopitemize

Nous proposons de travailler à l'échelle inter-SCoT dans un premier temps,
puis d'envisager une généralisation de notre démarche
à l'échelle départementale ou régionale.


\subsection
{Verrous scientifiques et méthodologiques}

Si les données sur l'occupation du sol sont de plus en plus abondantes,
les bases de données administratives ne sont pas conçues pour
suivre les changements et sont largement incomplètes en zone périurbaine à
l'échelle à laquelle nous proposons de travailler :
d'une part, l'artificialisation diffuse et le développement pavillonnaire
avec des extensions inférieures à 25 ha en surface sont majoritairement
comptabilisées comme espaces agricoles par Corine Land Cover, et d'autre part,
les systèmes agricoles atypiques caractéristiques de ces zones
sortent souvent du cadre de la PAC et sont donc mal
observées par le Registre parcellaire graphique (RPG).

Les données décrivant le marché foncier sont difficilement accessibles
en raison de leur coût, et il faudra éventuellement trouver
des variables de substitution pour approcher les effets d'anticipation
et de spéculation sur la disponibilité du foncier.

Distinguer entre les diverses formes d'agriculture
demande de qualifier et de quantifier les fonctions
remplies par ces différentes formes et de les hiérarchiser en fonction
des parties prenantes et de leurs attentes.

La caractérisation de changements qui impliquent
des processus avec des chronologies
différentes donne une importance particulière
à la prise en compte de la temporalité, à la fois
dans la structuration des données utilisées pour l'analyse
et dans la modélisation des processus. C'est une des limites actuelles
des outils de modélisation utilisés dans les études
sur les changements d'utilisation du sol.


\subsection{Terrains envisagés}

\subsubsection{Terrain initial : Bassin de Thau}

Le bassin de Thau est un territoire de 353 \unit{km2} autour de l’étang éponyme,
proche de Montpellier, bien connu de l’UMR TETIS,
pour lequel on dispose déjà de nombreuses données.

Il a connu une très forte urbanisation
au cours des 50 dernières années et regroupe 14 communes sur deux intercommunalités.
Le paysage agricole est composé pour partie de vignobles,
d’espaces agricoles « d’intérêt écologique » et d’espaces agricoles
productifs « à dynamiser » (céréales, monoculture de melons),
avec une forte diversification agricole déjà
engagée sur la commune de Villeveyrac.

Ayant adopté un SCoT début 2014, le syndicat mixte du bassin de Thau (SMBT) s’est donné à travers un
plan de gestion intégrée l’objectif de mieux maîtriser l’urbanisation
tout en « préservant les espaces agricoles et les paysages, menacés par une urbanisation
trop rapide pour conserver l’identité du territoire et le cadre de vie »
(SMBT, 2010 ; SMBT, 2014). La participation de la population, et la mise en place de nombreux
partenariats sont l’une des caractéristiques de fonctionnement revendiquée
tant par le syndicat que par ses partenaires \cite[maurel_apprentissage_2008].

Pour accompagner ce programme d'actions et évaluer en permanence
ses effets sur les dynamiques territoriales, le SMBT travaille avec l'appui de l'UMR TETIS
à la définition et à la mise en place d'un observatoire territorial basé
sur un ensemble d'indicateurs spatiaux.

\subsubsection{Terrain de validation}

Un deuxième terrain sera choisi rapidement en Rhône-Alpes pour reproduire
et valider la démarche, en utilisant les critères de sélection suivants :

\startitemize

\item dynamique urbaine différente ;
\item typologie différente d’agriculture ou de paysage ;
\item enjeux particuliers identifiés dans les documents de planification ;
\item implication des acteurs du monde agricole, de la société civile
  et des collectivités territoriales.

\stopitemize

Il sera possible d'utiliser un terrain déjà exploité par l'UMR EVS (bassin versant de l'Yzeron près de Lyon,
St Étienne Métropole) ou de choisir un terrain dans le cadre du projet FRUGAL.


\section{Méthodes}

Une méta-étude des cas portant sur les interactions ville-agriculture
publiés dans la littérature scientifique 
permettra d'abord d'identifier les éléments
de théorie pertinents et de sélectionner les processus à modéliser.

À partir de cette analyse bibliographique,
le processus de modélisation sera reproduit sur au moins deux études de cas
complémentaires qui permettront d'élaborer la méthode d'analyse des données
et d'enrichir la théorie expliquant les effets des interactions
ville-agriculture sur l'étalement urbain.

\subsection{Méthodes spatiales}

Nous proposons de recourir à l'analyse spatiale pour mobiliser
les données disponibles (données dérivées d'imagerie satellitaire,
données topographiques, données socioéconomiques) et caractériser qualitativement et quantitativement
les relations entre ville et agriculture à travers l’occupation du sol,
l’utilisation du sol et les potentialités d’usage.

La construction de modèles fait partie de la démarche de recherche en géographie
et permet de formuler des hypothèses
pour élaborer une information qui apporte un éclairage original et utile \cite[mathian_objets_2014].

Modéliser, du point de vue de la pratique géographique,
c'est « rechercher quelle composition de modèles
rend le mieux compte d'une organisation régionale ou locale, d'une configuration de champ
ou de réseau, ou d'une distribution spatiale » \cite[brunet_modeles_2000].

Les méthodes de l'analyse spatiale fournissent des outils de mesure en combinant systèmes
d’information géographique et outils statistiques \cite[pumain_les_2001, bavaud_handbook_2009, sanders_models_2010].
L’économétrie spatiale \cite[lesage_introduction_2009, chakir_spatial_2009, vignes_fiches_2013]
propose quant à elle des outils pour exploiter des données d’échelles différentes,
analyser les processus spatiaux et les flux mais
aussi étudier les préférences de localisation
ou encore quantifier les externalités.


\subsection{Processus de modélisation}

Notre démarche suivra la méthode générale synthétisée par \citet{magliocca_metastudies_2015}
à partir de la littérature sur la modélisation des changements d'occupation du sol.

Cette méthode divise le processus de modélisation en 5 étapes :

\startitemize[n]

\item le point de départ de tout processus de modélisation
   est la question de recherche et sa problématisation,
   qui définit le phénomène spatial auquel on s'intéresse
   ainsi que le contexte et l'échelle à laquelle on souhaite l'étudier,
   son éventuel rapport avec des processus d'échelle plus large ;

\item la description du problème permet d'identifier les limites du système,
   ses composants (acteurs, variables, processus)
   et les possibles relations qu'ils entretiennent.
   La littérature apporte
   un appui important pour identifier les composants et
   les décrire en même temps qu'elle peut apporter
   des éléments de théorie pour les mettre en relation.
   À cette étape, la cartographie à dire d'acteurs peut
   servir également à identifier ces éléments et à conceptualiser
   le système étudié.

\item le modèle conceptuel est ensuite traduit en modèle dynamique
   et implémenté dans un programme informatique
   en sélectionnant une représentation appropriée ;
   l'utilisation d'une plateforme de modélisation
   adaptée au type de modèle que l'on souhaite instancier
   facilite la réalisation de cette étape ;

\item le modèle est ensuite calibré et validé à l'aide des données
   du terrain d'études ; cette étape permet de tester la capacité
   du modèle à rendre compte des observations ainsi
   que sa sensibilité aux données en entrée ;

\item enfin, le modèle peut être exploité pour
   tester des hypothèses complémentaires, explorer
   différents scénarios ou être confronté à
   ses utilisateurs potentiels. Cette étape d'expérimentation
   permet de critiquer le modèle et de raffiner l'analyse du
   problème initial.

\stopitemize


\section{Résultats attendus}

Les résultats suivants sont attendus de ce projet de thèse :

\startitemize

\item adapter les méthodes de modélisation de la littérature sur les changements d'occupation du sol
  à l'analyse et la simulation des interactions ville-agriculture ;

\item proposer des indicateurs pertinents pour mesurer les effets de l'étalement urbain
  sur les usages agricoles et réciproquement ;

\item contribuer aux méthodes opérationnelles
  pour mobiliser les données disponibles 
  (bases de données topographiques et administratives, données socioéconomiques,
  imagerie satellitaire).

\stopitemize

\section{Encadrement et environnement de la thèse}

À reprendre de la version précédente du projet.