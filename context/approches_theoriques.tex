\section{Approches théoriques}

\subsection{Courants épistémiques}

Les sciences écologiques et les sciences sociales se sont développées
indépendamment jusqu'à encore récemment,
et malgré un appel récurrent pour davantage de transdisciplinarité,
ne se rencontrent pas facilement.

Les études urbaines et le courant de la {\em nouvelle économie géographique}
ont étudié les formes urbaines et le développement urbain
à travers des approches économiques et économétriques,
mais aussi sociologiques et anthropologiques.

La ville dense, économe en espace et supposée plus durable, s'oppose à la ville étalée,
résultat de la périurbanisation, de la recherche d'aménités environnementales
mais aussi d'évitement de déséconomies d'agglomération et d'externalités négatives
liées à la densité.

Sur un autre axe, deux modèles d'organisation urbaine sont couramment opposés :
a) le modèle monocentrique qui reconnaît la prépondérance
de la distance au centre urbain comme variable explicative de l'organisation spatiale ;
b) le modèle polycentrique qui rend compte de la hiérarchie des lieux.
Les deux modèles d'organisation se conjuguent à différentes échelles
\cite[pouyanne_theorie_2008].

L'opposition entre organisation monocentrique et polycentrique
s'entend également du point de vue
de l'analyse institutionnelle \cite[chanteau_linstitutionnalisme_2013].

% La cité-jardin de \citet{howard_tomorrow_2003} est une utopie,
% qui peut être considérée comme le modèle limite
% de l'organisation urbaine polycentrique et de la ville étalée.
% Elle correspond à une vision aspatiale (ou post-spatiale)
% de la société, où les contraintes liées à l'espace sont abolies ;
% son caractère utopique met en perspective
% l'importance de l'espace dans l'organisation urbaine
% et la structuration du territoire.

Les approches économiques, mais plus généralement
les approches en sciences sociales, mettent l'accent sur les comportements individuels
pour expliquer le processus d'urbanisation à travers les choix résidentiels
\cite[irwin_theory_2001,irwin_towards_2009].

\citet{ostrom_background_2011} souligne cependant que les chercheurs en sciences sociales
doivent davantage reconnaître que les comportements individuels
sont fortement influencés par le contexte dans lequel se situent
les interactions entre individus, davantage qu'ils ne sont le résultat
de simples différence inter-individuelles.

Les sciences de la Terre et de l'environnement s'intéressent
quant à elles à l'impact des activités humaines sur le milieu naturel,
en cherchant à caractériser l'occupation du sol, l'état des milieux
et à décrire les processus écologiques pour comprendre
le fonctionnement des écosystèmes et les perturbations qu'ils subissent.

Plus récemment, sous l'influence de nouvelles demandes sociétales,
l'agriculture urbaine s'est constituée en nouvel objet de recherche \cite[aubry_les_2015],
qui demande de qualifier et de quantifier les fonctions
remplies par ses différentes formes et de les hiérarchiser en fonction
des parties prenantes et de leurs attentes  \cite[ba_diversite_2011].

Le concept de services écosystémiques (\SES) s'est largement répandu
grâce à une importante médiatisation et renouvelle le concept de
multi-fonctionnalité en insistant sur les bénéfices sociaux
et économiques procurés par les écosystèmes.
Ces bénéfices sont distingués en 4 grandes catégories de services. La notion
de services est empruntée à une approche économique des relations homme-nature,
qui trouve sa justification dans des mécanismes comme les compensations environnementales
et les paiements aux services environnementaux (\PSE).
\citet{barnaud_vers_2011} relèvent les incertitudes scientifiques et sociétale de ce concept,
tandis que \citet{mathevet_solidarite_2010}
proposent de recourir au concept de solidarité écologique pour penser les
interdépendances sociales, économiques et écologiques à différentes échelles.

Les interactions entre ville et agriculture, en tant que composantes des dynamiques spatiales périurbaines,
restent peu étudiées.