\section{Objets de recherche}

Dans le contexte d'une volonté affichée de mieux intégrer les fonctions agricoles
dans le projet urbain,
et de limiter la consommation d'espaces agricoles et naturels,
nous proposons d'étudier les objets suivants et leurs relations :

\startitemize[packed]

\item l'étalement urbain et son pendant, la consommation d'espaces agricoles
\item l'agriculture urbaine et les fonctions qu'elle remplit
\item la localisation des usages agricoles
  et les interactions spatiales entre ville et agriculture
\item les modèles d'action, c'est-à-dire l'ensemble
  des projets et des actions que souhaitent (peuvent ?) mettre en œuvre
  les acteurs pour modifier les dynamiques territoriales.

\stopitemize

\section{Approches théoriques}

\subsection{Courants épistémiques}

Les sciences écologiques et les sciences sociales se sont développées
indépendamment jusqu'à encore récemment,
et malgré un appel récurrent pour davantage de transdisciplinarité,
ne se rencontrent pas facilement.

Les études urbaines et le courant de la {\em nouvelle économie géographique}
ont étudié les formes urbaines et le développement urbain en
mettant souvent l'accent sur les comportements individuels
à travers des approches économiques et économétriques,
mais parfois aussi sociologiques.

\citet{ostrom_background_2011} note cependant que les chercheurs en sciences sociales
doivent davantage reconnaître que les comportements individuels
sont fortement influencés par le contexte dans lequel se situent
les interactions entre individus, davantage qu'ils ne sont le résultat
de simples différence inter-individuelles.

Les sciences de la Terre et de l'environnement s'intéressent
quant à elles aux questions écologiques
et à l'impact des activités humaines sur le milieu naturel,
en cherchant à caractériser l'occupation du sol, l'état des milieux
et à décrire les processus écologiques pour comprendre
le fonctionnement des écosystèmes et les perturbations qu'ils subissent.

Plus récemment, sous l'influence de nouvelles demandes sociétales,
l'agriculture urbaine s'est constituée en nouvel objet de recherche \cite[aubry_les_2015],
qui demande de qualifier et de quantifier les fonctions
remplies par ses différentes formes et de les hiérarchiser en fonction
des parties prenantes et de leurs attentes  \cite[ba_diversite_2011].

Le concept de services écosystémiques (\SES) s'est largement répandu
grâce à une importante médiatisation et renouvelle l'approche de la
multi-fonctionnalité en insistant sur les bénéfices sociaux
et économiques procurés par les écosystèmes et en catégorisant
ces bénéfices en 4 grandes catégories de services. La notion
de services est empruntée à une approche économique des relations homme-nature,
qui sous-tend le principe des compensations environnementales
et des paiements aux services environnementaux (PSE).
\citet{barnaud_vers_2011} relève les incertitudes scientifiques et sociétale de ce concept,
tandis que \citet{mathevet_solidarite_2010}
proposent de recourir au concept de solidarité écologique pour penser les
interdépendances sociales, économiques et écologiques à différentes échelles.

Les interactions entre ville et agriculture, en tant que moteur de dynamiques spatiales,
restent peu étudiées.


\input theorie_organisation_urbaine

\input cadre_analyse

\input hypotheses_theoriques