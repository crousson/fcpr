\section{Approches théoriques}

\subsection{Théories de l'organisation urbaine
   et cadre d'analyse des changements d'occupation du sol}

Il existe une abondante littérature issue des sciences régionales d'une part,
de l'écologie et des sciences de la Terre d'autre part, traitant
des dynamiques spatiales, de l'organisation urbaine, de l'allocation et
des changements d'occupation du sol.
Nous nous limitons à en souligner
quelques aspects théoriques qui nous paraissent utiles par rapport
à la problématique et aux objets de recherche choisis.

\startitemize[n]

\item	\citet{von_thunen_isolated_1966} propose en 1842 un modèle théorique
	d'organisation spatiale de l'utilisation du sol sous l'influence de la ville.
	Les lieux se différencient en productions spécialisées qui forment
	des anneaux concentriques en raison
	d'une rente foncière déterminée par la nature de la production
	et le coût de transport vers le marché central, coût qui est fonction de la distance
	au centre ville.

	Ce modèle monocentrique s'appuie sur la théorie des utilités et des
	avantages comparatifs. Il met en évidence l'auto-organisation spatiale
	de la répartition des cultures et le rôle important
	de la distance au centre-ville.

	Von Thünen développe des idées qui sont, longtemps en avance,
	précurseurs de la théorie des lieux centraux de Christaller et Lösch
	puis des théories qui seront développées par la {\em Nouvelle géographie économique}
	\cite[fujita_thunen_2012].

	Parmi ces idées, il y a celle, contre-intuitive,
	que le développement des transports favorise la concentration urbaine
	et modifie dramatiquement la hiérarchie des lieux centraux.

\item	La densité démographique est une mesure possible de l'étalement urbain.
	De nombreuses études urbaines se sont intéressées
	à la densité et sa distribution spatiale
	en cherchant à la formaliser sous forme de loi mathématique
	\cite[bailly_les_1973, wegener_operational_1994].
	Ces travaux ont relié l'évolution des densités selon un gradient centre-périphérie
	à différents facteurs socioéconomiques.

\item	Les théories des lieux centraux de Christaller (1933) et Lösch (1940)
	décrivent le principe d'une hiérarchie dans les systèmes de ville,
	qui s'explique par la répartition des aires de marché, le coût des infrastructures
	de communications et l'organisation des fonctions administratives.
	Ces théories formalisent l'idée d'une organisation spatiale hiérarchisée et
	multi-échelle.

\item	Les sciences régionales et les études urbaines
	opposent de manière dialogique
	un modèle monocentrique et un modèle polycentrique de l'organisation urbaine.
	Le modèle monocentrique est le modèle de la ville dense
	mais aussi celui de l'opposition centre-périphérie et des effets d'agglomération ;
	tandis que le modèle polycentrique est le modèle de la ville discontinue et
	de la ville étalée \cite[pouyanne_theorie_2008].

\item	Cette opposition mono-polycentrique
	s'entend également du point de vue
	de l'analyse institutionnelle \cite[chanteau_linstitutionnalisme_2013].
	\citet{ostrom_background_2011} propose la notion de système socioécologique
	qui met en exergue le rôle des institutions et les interactions multi-échelles
	dans la gestion des ressources naturelles.
	Le cadre conceptuel (heuristique ?) IAD (Institutional Analysis and Developement framework)
	formalise ces concepts.

\item	Les formes d'organisation urbaine dépendent de l'échelle
	à laquelle on les considère \cite[wegener_polycentric_2013] :

	\startitemize[packed]

	\item à l'échelle régionale : modèle monocentrique
	  dans lequel les grands pôles urbains ont accru leur influence
	  aux dépends des pôles secondaires,
	  sous l'effet du développement des transports

	\item à l'échelle locale : modèle polycentrique
	  ou transition d'un modèle monocentrique vers un modèle polycentrique.

	\stopitemize

\item	Le courant scientifique qui s'intéresse aux changements d'occupation
	du sol du point de vue de la biodiversité et de l'écologie du paysage
	recourt fréquemment aux concepts de l'étude d'impact pour analyser les processus de changement.
	Le modèle DPSIR (Driver Pressure State Impact Response) fournit un cadre conceptuel (heuristique ?)
	est largement adopté et offre un support robuste pour penser la relation entre environnement,
	société et politiques publiques \cite[tscherning_does_2012].
	Certains reprochent à ce modèle une vision mécaniste
	et « occidentale » de la relation homme-nature.

\stopitemize


\subsection{Hypothèses théoriques :
   importance de l'espace, phénomènes de polarisation
   et interactions globales-locales}

\startitemize[n]

\item	L'utopie urbaine de \citet{howard_tomorrow_2003} fait de la cité-jardin ou de la ville-campagne
	l'organisation urbaine idéale.
	Cette utopie, qui peut être considérée comme le modèle limite de l'organisation urbaine polycentrique,
	correspond à une vision aspatiale (ou post-spatiale)
	de la société, où les contraintes liées à l'espace sont abolies ;
	son caractère utopique met en perspective
	l'importance de l'espace dans l'organisation urbaine
	et la structuration du territoire.

\item 	Les forces économiques, à travers les mécanismes
	du marché foncier et de la propriété privée,
	sont importantes et déterminent en grande partie
	les configurations spatiales de l'utilisation du sol
	dans la frange périurbaine ;
	mais elles ne suffisent pas à rendre compte
	de la complexité des interactions qui sont en jeu.

\item	L'organisation urbaine comme l'organisation
	des usages agricoles résultent
	de forces centrifuges (par ex. externalités négatives
	liées à la concentration urbaine) et centripètes
	(modèle gravitaire, modèle centre-périphérie)

	On peut rendre compte, au moins en partie, de ces forces
	par une double polarisation :

	\startitemize[a,packed]

    \item 	polarisation centre-périphérie
     	dans les préférences des habitants de la ville,
		qui préfèrent pour certains la ville dense,
		pour d'autres une ville plus étalée, en interface avec des
		espaces agricoles et naturels pourvoyeurs d'aménités ;

    \item	polarisation local-global dans les usages agricoles,
		en fonction du niveau d'adaptation de ces usages
		à l'influence urbaine, par ex. des cultures maraîchères
		destinées à des circuits de proximité ou une parcelle
		de grande culture destiné au marché mondial.

	\stopitemize

	Une manière de représenter cette polarisation comme variable
	des dynamiques spatiales
	est d'utiliser les champs moyens \cite[sanders_models_2010].

\item	Les interactions spatiales entre ville et agriculture
	ne résultent pas seulement de la distance géographique
	mais aussi d'interactions globales/locales \cite[seto_urban_2012].
	Cette proximité fonctionnelle s'inscrit
	dans la participation à des réseaux (filière, circuit de commercialisation, etc.)
	qui dépassent la seule échelle locale,
	et dépend aussi de facteurs institutionnels.
	De ce point de vue, la notion de circuit court peut s'entendre en terme de transport
	(kilomètres parcourus entre le lieu de production et le consommateur)
	mais aussi en nombre d'intermédiaires.

\item	comportements individuels des agents : comment les aborder ?
	explicitement ou uniquement par leurs effets spatiaux,
	à travers des individus « moyens » ?

\stopitemize