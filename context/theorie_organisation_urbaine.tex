\section{Théories de l'organisation urbaine}

\startitemize[n]

\item	\citet{von_thunen_isolated_1966} propose en 1842 un modèle théorique
	d'organisation spatiale de l'utilisation du sol sous l'influence de la ville.
	Les lieux se différencient en productions spécialisées qui forment
	des anneaux concentriques en raison
	d'une rente foncière déterminée par la nature de la production
	et le coût de transport vers le marché central, coût qui est fonction de la distance
	au centre ville.

	Ce modèle monocentrique s'appuie sur la théorie des utilités et des
	avantages comparatifs. Il met en évidence l'auto-organisation spatiale
	de la répartition des cultures et le rôle important
	de la distance au centre-ville.

	Von Thünen développe des idées qui sont, longtemps en avance,
	précurseurs de la théorie des lieux centraux de Christaller et Lösch
	puis des théories qui seront développées par la {\em nouvelle géographie économique}
	\cite[fujita_thunen_2012].

	Parmi ces idées, il y a celle, contre-intuitive,
	que le développement des transports favorise la concentration urbaine
	et modifie dramatiquement la hiérarchie des lieux centraux.

\item	La densité démographique est une mesure possible de l'étalement urbain.
	De nombreuses études urbaines se sont intéressées
	à la densité et sa distribution spatiale
	en cherchant à la formaliser sous forme de loi mathématique
	\cite[bailly_les_1973, wegener_operational_1994].
	Ces travaux ont relié l'évolution des densités selon un gradient centre-périphérie
	à différents facteurs socioéconomiques.

\item	Les théories des lieux centraux de Christaller (1933) et Lösch (1940)
	décrivent le principe d'une hiérarchie dans les systèmes de ville,
	qui s'explique par la répartition des aires de marché, le coût des infrastructures
	de communications et l'organisation des fonctions administratives.
	Ces théories formalisent l'idée d'une organisation spatiale hiérarchisée et
	multi-échelle.

\item	Les sciences régionales et les études urbaines
	opposent de manière dialogique
	un modèle monocentrique et un modèle polycentrique de l'organisation urbaine.
	Le modèle monocentrique est le modèle de la ville dense
	mais aussi celui de l'opposition centre-périphérie et des effets d'agglomération ;
	tandis que le modèle polycentrique est le modèle de la ville discontinue et
	de la ville étalée \cite[pouyanne_theorie_2008].

\item	Les formes d'organisation urbaine dépendent de l'échelle
	à laquelle on les considère \cite[wegener_polycentric_2013] :

	\startitemize[packed]

	\item à l'échelle régionale : modèle monocentrique
	  dans lequel les grands pôles urbains ont accru leur influence
	  aux dépends des pôles secondaires,
	  sous l'effet du développement des transports

	\item à l'échelle locale : modèle polycentrique
	  ou transition d'un modèle monocentrique vers un modèle polycentrique.

	\stopitemize

\stopitemize