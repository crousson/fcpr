\section[constats]
{Stratégie d'information géographique et nécessité de développer
les usages : une analyse des difficultés au ministère chargé de l'agriculture}


\subsection[constats:strategie]
{Une stratégie d'information géographique difficile à définir}


Le rapport à l'information géographique des services du ministère
pourrait se caractériser par :

\startitemize

\item des usages de l'information géographique
      qui reste principalement orienté vers la cartographie thématique,
      plutôt que le développement d'un système d'information territorial ;

\item une ambition néanmoins transversale des SIG qui questionne
      le maintien d'une fonction SI au niveau local,
      dans les services déconcentrés, alors que celle-ci est découragée
      au profit d'une recentralisation des systèmes d'information ;      

\item une volonté de mutualisation des infrastructures au niveau central et interministériel
      qui se justifie par la recherche d'économies d'échelle,
      mais qui donne des résultats décevants
      et apparaît en pratique difficile à gouverner ;

\item la difficile émergence du collaboratif,
      qui semble pourtant de plus en plus être une voie à développer
      pour collecter et enrichir l'information spatiale,
      mais aussi pour exploiter les données ;

\item une faible appropriation de la dimension spatiale
      et des enjeux des SI environnementaux
      par les directions générales,
      qui délèguent volontiers un sujet perçu comme essentiellement
      technique à leurs opérateurs (ASP, IGN).

\stopitemize

La SDSI, qui souhaite se concentrer sur les moyens informatiques,
la gestion de projet et les applications de gestion
s'est désengagée de l'accompagnement méthodologique et technique
qu'elle procurait depuis 2002 sur les outils de l'information géographique.

Le schéma directeur national des systèmes d'information de 2012
affiche des ambitions très limitées en matière d'information géographique
et l'idée que l'information géographique est une problématique
métier s'est imposée (ou a été imposée),
déplaçant la responsabilité de définir une stratégie
pour développer les usages de l'information géographique
vers les directions métiers et leurs maîtrises d'ouvrage,
ce pour quoi elles ne sont pas bien armées.

Malgré ces difficultés, le contexte évolue favorablement
et des progrès ont été réalisés.


\subsection[constats:progres]
{Des progrès dans l'accès et le partage de l'information géographique pour les
services déconcentrés}


En 2006, les principaux freins au développement
des usages de l'information géographique à l'ONF étaient :

\startitemize

\item la difficulté à acquérir les données nécessaires,
      et en particulier le coût de ces acquisitions

\item un retard important dans l'informatisation et l'outillage
      des processus métiers qui empêchait de faire
      des données spatialisées le support transversal
      du système d'information métier ;

\item un cloisonnement organisationnel défavorable
      à la circulation de l'information en rendant difficiles
      les échanges internes et interinstitutionnels de
      données, et plus généralement défavorable à l'innovation,
      à la transformation des processus.

\stopitemize

En 2015, presque dix ans plus tard, ce constat ne serait plus exactement le
même ; la problématique d'accès aux données a favorablement évolué :
i) les données de référence sont beaucoup plus facilement disponibles et
accessibles, et ii) l'idée que les données publiques sont un bien commun a
progressé avec la culture de l'open data, ce qui bénéficie d'abord à
l'administration elle-même. Mais les aspects méthodologiques et
organisationnels restent sinon à peu près les mêmes.

Entre 2007 et 2015, on peut néanmoins noter que des progrès significatifs ont
été enregistrés :

\startitemize

\item les infrastructures de données géographiques ont acquis en maturité et
réussissent à entretenir des partenariats importants entre acteurs publics, en
particulier au niveau régional, bien qu'elles restent principalement portées
par les services de l'État ;

\item d'autres plateformes se mettent en place et permettent d'accéder dans de
meilleures conditions à un bouquet de plus en plus large de données
spatialisées et d'observation de la Terre.

\stopitemize

À l'échelle ministérielle et interministérielle, ces progrès se sont traduits
par des avancées principalement techniques, mais aussi par des évolutions
institutionnelles, notamment en ce qui concerne l'accès et le partage des
données :

\startitemize

\item les deux ministères techniques en charge du territoire ont mis en place,
malgré les difficultés propres à l’interministérialité, une infrastructure de
données géographiques nationale commune (Géo-IDE) pour répondre, quoique
encore imparfaitement, aux exigences de la directive INSPIRE et de l'open data ;

\item à la SDSI, nous avons aussi beaucoup investi dans la maîtrise des risques et
des coûts liés aux projets informatiques, pour prendre en compte l'information
spatiale dès la conception des projets, avec un travail important sur les cas
d'usage, le recensement des patrons applicatifs, l'ergonomie applicative et
l'urbanisation du système d'information ;

\item la gratuité des données du référentiel géographique à grande échelle (RGE)
pour un usage dans le cadre d'une mission de service public sécurise l'accès à
des données de base et garantit l'utilisation de référentiels communs entre
acteurs du service public ;

\item le GéoPortail s'est transformé pour faciliter l'accès aux données de l'IGN et
leur utilisation dans les systèmes d'information métier, en même temps que le
ministère réussit mieux à collaborer avec l'IGN, ce qui lui permet de
bénéficier de son savoir-faire.

\stopitemize

Malgré ces progrès, et dans un contexte où la problématique d'acquisition et
d'accès aux données a significativement évolué, il reste des difficultés
d'ordre organisationnel autant que technique.


\subsection[constats:paradoxe_donnees]
{Le paradoxe de la sous-utilisation des données}

Le paradoxe est que les acteurs publics disposent aujourd'hui de beaucoup de données dont
ils ne savent souvent pas quoi faire, alors qu'ils ne
réussissent pas à obtenir l'information utile qui leur permettrait d'agir
efficacement dans un monde complexe. L'efficacité doit être
ici comprise comme la capacité à adapter l'action à ses objectifs, à apprendre
des erreurs et trouver des modes d'action organisée plus résilients.

Des investissements importants ont été réalisés pour acquérir
et produire des données d'observation de la Terre.
Cela se traduit par l'accroissement régulier du
nombre de capteurs en opération et une augmentation quasi exponentielle
de la quantité de données disponibles.

Néanmoins, cet accroissement de l'offre n'induit proportionnellement qu'un
faible accroissement des usages, ce qui traduit les limites d'un progrès
technologique insuffisamment adossé à une réflexion sur les usages et les
conditions du transfert opérationnel.

L'intérêt pour les données, en particulier dans les applications
environnementales et de connaissance du territoire, existe cependant chez les
acteurs publics sans qu'ils soient pour autant prêts à payer pour utiliser des
données qu'ils ont en pratique beaucoup de mal à exploiter. D'où un décalage
important entre le potentiel de ces données et leur valeur réelle d’usage.

Les plateformes, comme c'est le cas dans la démarche GéoSUD/Théia, poursuivent
l'objectif de :

\startitemize

\item lever le verrou de l'accès aux données ;

\item créer des « écosystèmes » données-utilisateurs-infrastructures à l'intérieur
desquels l'expérimentation et l'innovation sont facilitées et permettent de
développer les usages.

\stopitemize

Cette attention apportée aux usages est devenue nécessaire pour identifier de
nouveaux gisements de valeur et améliorer la production de connaissance
dans la chaîne aval du traitement de l'information, qui reste
insuffisamment développée et ne prend pas assez en compte les utilisateurs
finaux.


\subsection[constats:difficultes]
{Des difficultés persistantes qui demandent à être levés par des approches
innovantes}

On peut analyser les difficultés qui subsistent
autour de trois enjeux principaux :

\startitemize

\item l'enjeu de la transversalité, qui est de mieux faire circuler l'information en
se libérant des contraintes des « silos » existants ;

\item l'enjeu de l'accompagnement méthodologique, qui est d'homogénéiser le niveau
des pratiques et d'assurer un transfert suffisant vers l'opérationnel ;

\item l'enjeu des usages de l'information, qui est de créer davantage de valeur en
répondant à des usages pertinents adaptés aux besoins des utilisateurs.

\stopitemize

En ce qui concerne les deux premiers enjeux, les plateformes qui se
développent à différents niveaux ont considérablement amélioré les conditions
d'accès aux données, tout en créant des écosystèmes de réutilisation
favorables à l'accompagnement et aux développements méthodologiques.
Même s'il reste des difficultés de mise en œuvre qui sont principalement
organisationnelles, la voie est tracée.

L'enjeu des usages de l'information pose d'autres difficultés parce que les
utilisateurs ont en réalité beaucoup de mal à dépasser leurs pratiques
existantes et à revisiter leurs processus métiers pour imaginer comment ils
pourraient davantage tirer parti des données
et exprimer des besoins concrets et réalistes,
c'est-à-dire auxquels on puisse répondre.

\subsection[constats:circulations]
{Faire circuler l'information en se libérant des contraintes des silos
existants (enjeu de transversalité)}

{\it Constats :

\startitemize[a,packed]

\item des investissements centrés sur les outils et les exigences techniques au
détriment de la réponse organisationnelle

\item un cloisonnement qui reste important

\item un manque de structuration du patrimoine de données et d'organisation des
processus

\item une circulation insuffisante de l'information entre services.

\stopitemize}

La directive INSPIRE a été un moyen, un prétexte utile pour développer les
échanges de données, mais elle a trop été le seul horizon de travail de la
maîtrise d'ouvrage interministérielle.

Si les infrastructure de données géographiques constituent un progrès majeur,
elles ne résolvent pas en tant que solutions techniques les problèmes
organisationnels (Pornon, 2004). Cela s'est vérifié dans le programme
interministériel d'infrastructure de données géographiques Géo-IDE, qui s'est
trop concentré sur les outils au détriment de la prise en compte des besoins
organisationnels.

Les relations parfois difficiles entre la tutelle et ses opérateurs créent un
cloisonnement artificiel préjudiciable à l'efficacité collective. Le SI
géographique de la PAC opéré par l'ASP en est un exemple, alors même que la
mise en place de ce système d'information en 2007 était précurseur et a été un
succès technique. Paradoxalement, ce succès a justifié par ailleurs un
désengagement du ministère des sujets liés à l'information géographique.

DRAAF et DREAL ont réussi dans la plupart des régions à
développer une collaboration étroite autour de l'information géographique et
des plateformes régionales.

En DRAAF, à la différence des DREAL qui ont réussi à mettre en place des pôles
dédiés à la connaissance du territoire dotés de moyens importants, les SRISE
n'assument pas assez en général une fonction transversale qui leur permettrait
de mettre leurs compétences au service de l'ensemble de la DRAAF et des DDI.

Les tentatives de structurer le patrimoine de données des services
déconcentrés au niveau national donnent des résultats mitigés. Dans une
première période, de 2004 à 2007, où le dispositif de standardisation était
principalement à l'initiative du niveau local, les résultats ont été
encourageants. Avec l'élargissement du dispositif en 2008 dans le cadre de la
mutualisation interministérielle, la commission chargée de la standardisation
des données, la COVADIS, a adopté un fonctionnement beaucoup plus descendant
et ne répond plus en réalité qu'aux besoins de modélisation des maîtrises
d'ouvrage d'applications métiers.

Il nous faut donc aussi répondre au besoin interne de transversalité et de
structuration du patrimoine de données, en même temps que nous avons besoin de
rendre plus opérationnel les échanges entre l'échelon local et l'échelon
national/européen d'une part et entre l'État, ses opérateurs et les
collectivités territoriales d'autre part pour renforcer les collaborations
existantes.

\subsection[constats:homogeneisation]
{Homogénéiser les pratiques et assurer un transfert suffisant vers
l'opérationnel (enjeu de l'accompagnement méthodologique)}

{\it Constats :

\startitemize[a,packed]

\item diffusion limitée des outils et des méthodes

\item un écart qui se creuse entre services

\item une collaboration entre services qui peine à s'organiser

\item des difficultés à faire face à l'afflux de données à traiter

\stopitemize}

La « démocratisation » des outils de
l'information géographique et la diffusion des méthodes n' a pas assez progressé,
bien qu'aujourd'hui le logiciel libre, les dispositifs de formation à distance, les ressources
pédagogiques en licence ouverte mais aussi l'abondance de données plus
facilement accessibles y aident.

La diminution des moyens d'accompagnement et d'animation crée le risque de
creuser encore un peu plus l'écart entre les services les plus avancés et ceux
qui ont historiquement moins investi et qui risquent de « décrocher ».

Bien qu'une culture d'entraide existe et se développe autour de support comme
les forums internes, la collaboration entre services a du mal à se structurer.

Cela peut s'illustrer par exemple par le programme VALOR : ce programme mis en
place pour faire travailler les SRISE en réseau autour de la valorisation et
de la diffusion des données peine à entretenir sa dynamique dans le temps,
faute de mobiliser de manière pertinente les outils collaboratifs qui
permettraient de partager et d'enrichir collectivement les communs
méthodologiques produits par les services déconcentrés sur le modèle de l'open
source, et y incorporer aussi les apports de la recherche trop souvent
ignorés.

L'enjeu de l'accompagnement méthodologique est aussi de tirer parti
efficacement de la quantité de données disponibles qui s’est considérablement
accrue en l'espace d'une dizaine d'années.

\subsection[constats:creation_valeur]
{Créer davantage de valeur en répondant à des usages pertinents adaptés aux
besoins des utilisateurs et de la société (enjeu de la valorisation des
données)}

{\it Constats :

\startitemize[a,packed]

\item manque d'accompagnement des directions métiers et des MOA

\item alignement insuffisant entre le métier (la mission) et la stratégie de
systèmes d'information

\stopitemize}

La culture organisationnelle du ministère, du moins telle que j'ai pu la vivre
dans le contexte des projets de système d'information, reste marquée par :

\startitemize

\item une forte aversion pour le risque et le manque d'expérience dans la gestion
des projets d'innovation qui rend difficile de mener des expérimentations et
d'intégrer des nouveautés dans le cadre technique du ministère ;

\item la difficulté à s'ouvrir sur l'extérieur et à travailler en coopération avec
d'autres partenaires, notamment avec les organismes de recherche et les
collectivités territoriales.

\stopitemize

Les directions métiers ne sont pas assez accompagnées par les services supports
pour les aider à mieux prendre en
compte les enjeux du numérique, et plus spécifiquement les enjeux de
l'information géographique, dans leur stratégie de transformation et
d'amélioration de leurs processus métier.

Le développement de nouveaux services à valeur ajoutée demande de travailler
sur l'alignement entre le métier et la stratégie de système d'information, et
plus spécifiquement sur les usages de l'information.

L'objectif sous-jacent est de favoriser la valorisation des données et un
usage analytique et décisionnel de l'information géographique.