\section
{Quelques constats tirés de l’expérience}

Il reste des verrous organisationnels et méthodologiques à l’accès, au partage
et à la circulation de l’information dans les services de l’État et ses
établissements. On peut analyser ces verrous autour de trois enjeux principaux :

\startitemize

\item l'enjeu de la transversalité, qui est de mieux faire circuler
      l'information en se libérant des contraintes des « silos » existants ;

\item l'enjeu de l'accompagnement méthodologique, qui est d'homogénéiser
      le niveau des pratiques et d'assurer un transfert suffisant
      vers l'opérationnel ;

\item l'enjeu de valorisation des données, qui est de créer davantage de valeur en
répondant à des usages pertinents adaptés aux besoins des utilisateurs et de
la société.

\stopitemize

Cette annexe détaille les constats que je peux tirer de mon expérience à l'ONF
et au ministère chargé de l'agriculture.

\subsection
{Faire circuler l'information en se libérant des contraintes des silos
existants (enjeu de transversalité)}

{\ss\bf Constats} :

\startitemize

\item des investissements centrés sur les outils et les exigences techniques au
détriment de la réponse organisationnelle

\item un cloisonnement qui reste important

\item un manque de structuration du patrimoine de données et d'organisation des
processus

\item une circulation insuffisante de l'information entre services.

\stopitemize

La directive INSPIRE a été un moyen, un prétexte utile pour développer les
échanges de données, mais elle a trop été le seul horizon de travail de la
maîtrise d'ouvrage interministérielle.

Si les infrastructure de données géographiques constituent un progrès majeur,
elles ne résolvent pas en tant que solutions techniques les problèmes
organisationnels (Pornon, 2004). Cela s'est vérifié dans le programme
interministériel d'infrastructure de données géographiques Géo-IDE, qui s'est
trop concentré sur les outils au détriment de la prise en compte des besoins
organisationnels.

Les relations parfois difficiles entre la tutelle et ses opérateurs créent un
cloisonnement artificiel préjudiciable à l'efficacité collective. Le SI
géographique de la PAC opéré par l'ASP en est un exemple, alors même que la
mise en place de ce système d'information en 2007 était précurseur et a été un
succès technique. Paradoxalement, ce succès a justifié par ailleurs un
désengagement du ministère des sujets liés à l'information géographique.

Heureusement, DRAAF et DREAL ont réussi dans la plupart des régions à
développer une collaboration étroite autour de l'information géographique et
des plateformes régionales.

En DRAAF, à la différence des DREAL qui ont su mettre en place des pôles
dédiés à la connaissance du territoire dotés de moyens importants, les SRISE
n'assument pas assez en général une fonction transversale qui leur permettrait
de mettre leurs compétences au service de l'ensemble de la DRAAF et des DDI.

Les tentatives de structurer le patrimoine de données des services
déconcentrés au niveau national donnent des résultats mitigés. Dans une
première période, de 2004 à 2007, où le dispositif de standardisation était
principalement à l'initiative du niveau local, les résultats ont été
encourageants. Avec l'élargissement du dispositif en 2008 dans le cadre de la
mutualisation interministérielle, la commission chargée de la standardisation
des données, la COVADIS, a adopté un fonctionnement beaucoup plus descendant
et ne répond plus en réalité qu'aux besoins de modélisation des maîtrises
d'ouvrage d'applications métiers.

Il nous faut donc aussi répondre au besoin interne de transversalité et de
structuration du patrimoine de données, en même temps que nous avons besoin de
rendre plus opérationnel les échanges entre l'échelon local et l'échelon
national/européen d'une part et entre l'État, ses opérateurs et les
collectivités territoriales d'autre part pour renforcer les collaborations
existantes.

\subsection
{Homogénéiser les pratiques et assurer un transfert suffisant vers
l'opérationnel (enjeu de l'accompagnement méthodologique)}

{\ss\bf Constats} :

\startitemize

\item diffusion limitée des outils et des méthodes

\item un écart qui se creuse entre services

\item une collaboration entre services qui peine à s'organiser

\item des difficultés à faire face à l'afflux de données à traiter

\stopitemize

Nous n'avons pas assez progressé dans la « démocratisation » des outils de
l'information géographique et la diffusion des méthodes, bien qu'aujourd'hui
le logiciel libre, les dispositifs de formation à distance, les ressources
pédagogiques en licence ouverte mais aussi l'abondance de données plus
facilement accessibles nous y aident.

La diminution des moyens d'accompagnement et d'animation crée le risque de
creuser encore un peu plus l'écart entre les services les plus avancés et ceux
qui ont historiquement moins investi et qui risquent de « décrocher ».

Bien qu'une culture d'entraide existe et se développe autour de support comme
les forums internes, la collaboration entre services a du mal à se structurer.

Cela peut s'illustrer par exemple par le programme VALOR : ce programme mis en
place pour faire travailler les SRISE en réseau autour de la valorisation et
de la diffusion des données peine à entretenir sa dynamique dans le temps,
faute de mobiliser de manière pertinente les outils collaboratifs qui
permettraient de partager et d'enrichir collectivement les communs
méthodologiques produits par les services déconcentrés sur le modèle de l'open
source, et y incorporer aussi les apports de la recherche trop souvent
ignorés.

L'enjeu de l'accompagnement méthodologique est aussi de tirer parti
efficacement de la quantité de données disponibles qui s’est considérablement
accrue en l'espace d'une dizaine d'années.

\subsection
{Créer davantage de valeur en répondant à des usages pertinents adaptés aux
besoins des utilisateurs et de la société (enjeu de la valorisation des
données)}

{\ss\bf Constats} :

\startitemize

\item manque d'accompagnement des directions métiers et des MOA

\item alignement insuffisant entre le métier (la mission) et la stratégie de
systèmes d'information

\stopitemize

La culture organisationnelle du ministère, du moins telle que j'ai pu la vivre
dans le contexte des projets de système d'information, reste marquée par :

une forte aversion pour le risque et le manque d'expérience dans la gestion
des projets d'innovation qui rend difficile de mener des expérimentations et
d'intégrer des nouveautés dans le cadre technique du ministère ;

la difficulté à s'ouvrir sur l'extérieur et à travailler en coopération avec
d'autres partenaires, notamment avec les organismes de recherche et les
collectivités territoriales.

Nous ne réussissons pas à accompagner efficacement les décideurs, les
maîtrises d'ouvrage et les utilisateurs pour les aider à mieux prendre en
compte les enjeux du numérique, et plus spécifiquement les enjeux de
l'information géographique, dans leur stratégie de transformation et
d'amélioration de leurs processus métier.

Le développement de nouveaux services à valeur ajoutée demande de travailler
sur l'alignement entre le métier et la stratégie de système d'information, et
plus spécifiquement sur les usages de l'information.

L'objectif sous-jacent est de favoriser la valorisation des données et un
usage analytique et décisionnel de l'information géographique.
