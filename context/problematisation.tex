\section{}

L'utilisation et les changements d'occupation du sol
sont le support de nombreuses problématiques sociales,
économiques et environnementales
et mobilisent des communautés épistémiques variées
avec des approches théoriques différentes.

- d'un côté, les sciences régionales et les études urbaines
  s'y sont intéressés du point de vue social,
  et plus particulièrement du point de vue économique

- de l'autre, les sciences de la Terre et l'étude des surfaces continentales
  ont adopté les méthodes des sciences expérimentales
  pour observer et décrire l'occupation du sol,
  l'état et le fonctionnement des écosystèmes.

Ricardo (18xx) analyse les rentes foncières à partir
de la fertilité des sols

von Thünen (18xx) formule une théorie de la localisation
qui rend compte de l'influence de la ville
sur l'organisation spatiale des productions agricoles