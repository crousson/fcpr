\section
{Contexte}

\subsection
{Une prise en compte encore récente de l’étalement urbain dans les politiques
publiques}

L’étalement urbain, sans être un phénomène nouveau, s’est accéléré dans la
deuxième moitié du XXème siècle. Il n’a cependant bénéficié d’une prise en
compte au niveau européen et d’une forte médiatisation que depuis une dizaine
d’années.

Les préoccupations liées à l'étalement urbain ont motivé la commande en 2008
par la DRAAF Languedoc-Roussillon d’une étude sur la mesure de la consommation
de terres agricoles liée à l’artificialisation \cite[balestrat_systeme_2011].

Ces préoccupations ont été traduites en France à partir de 2000 dans le droit
rural, le droit de l’urbanisme et aujourd’hui le droit de l’environnement (SRU
2000, DTR 2005, ENE 2010, LMA 2010), faisant de la maîtrise de la consommation
des terres agricoles un enjeu important et partagé, et l’élevant au rang de
nouvelles normes \cite[bertrand_quelle_2006,bertrand_terres_2013].

La consommation d’espaces agricoles et naturels est un phénomène complexe, qui
évolue dans le temps et dans l’espace, et sa problématisation fait bien moins
consensus que certains discours globalisants ne le laissent croire. «
Étalement urbain », « consommation d'espaces agricoles et naturels », «
artificialisation des terres ou du sol » renvoient à une même problématique
sans être synonymes et traduisent des points de vue différents mais
complémentaires.

\subsection
{Les enjeux de l'étalement urbain}

L’ « étalement urbain » (en anglais « urban sprawl ») définit une
problématique d’aménagement du territoire caractérisée par un changement
d’occupation du sol dû à l’urbanisation à la fois irréversible et plus rapide
que la croissance démographique \cite[certu_consommation_2010].

À l'époque contemporaine, en France, d’autres dynamiques ont pour conséquence
des changements importants d'occupation du sol : on peut citer a) la déprise
agricole qui restitue des terres autrefois cultivées à l'espace naturel et
forestier, b) l'évolution des pratiques agricoles sous l’influence conjuguée
du marché, des mesures agri-environnementales et des politiques de trame verte
et bleue, et enfin c) le changement climatique dont l'impact reste beaucoup
plus difficile à mesurer.

Toutefois, l’étalement urbain a des conséquence sur le quotidien du plus grand
nombre et c’est une dynamique rapide : de 1992 à 2004, ce sont entre 50 000 et
60 000 ha de terres qui ont été artificialisées par an en France \cite[chakir_analyse_2006].
L’artificialisation progresse quatre fois plus rapidement que la
croissance démographique \cite[balny_proteger_2009].
Ces chiffres ont marqué les autorités
publiques comme l’opinion.

En France, le territoire reste globalement peu artificialisé comparé à
d’autres régions européennes : entre 5~\% et 8~\% de la surface totale en 2006
selon les sources contre 37~\% aux Pays-Bas par exemple.
C’est bien la dynamique de l’artificialisation qui interpelle les politiques publiques.

Le terme « étalement » traduit non seulement l'extension spatiale de la ville
mais sous-entend encore un envahissement non désiré du territoire, l’entropie
d’une urbanisation dysfonctionnelle et un changement qualitatif du tissu
urbain, comme si celui-ci se diluait dans l'espace et faisait tâche d’huile.

Le dynamisme nouveau des territoires ruraux renouvelle une population rurale
de moins en moins agricole, demandeuse d’aménités parfois contradictoires avec
l’usage agricole et l’évolution des systèmes agronomiques (Pistre, 2012). La
périurbanisation n’est plus seulement l’extension périphérique de la ville,
c’est aussi la ville qui s’invite à la campagne.

La problématique de l’étalement urbain reflète, d’une part, les préoccupations
des collectivités pour lesquelles l’étalement urbain a un coût social
éventuellement important et, d’autre part, celles du monde agricole pour qui
l’artificialisation représente une perte irréversible de foncier productif.
Elle reflète aussi, directement ou indirectement, les préoccupations des
citadins soucieux de la qualité de leur cadre de vie et davantage sensibilisés
au développement durable.

Si l'étalement urbain est d'abord une dynamique de changement d'occupation du
sol, sa problématisation ventriloque des préoccupations variées : sociales,
économiques, écologiques, foncières, urbanistiques. C’est pourquoi il peut
être un objet pertinent pour aborder de manière systémique le fonctionnement
d’un territoire et servir de support à une réflexion sur son évolution
(Dodane, 2014).

\subsection
{La consommation d'espaces agricoles et naturels non raisonnée n’est pas
durable}

Le corollaire de l'étalement urbain est la consommation d'espaces agricoles,
naturels et forestiers, qui est la conséquence en négatif du même phénomène de
différenciation et d'organisation de l'espace au cours du processus
d’urbanisation. Cette diversité de points de vue est importante pour rendre
compte de la variété des enjeux.

L'étalement urbain est indissociable de la consommation d'espaces : pour que
la ville s'étale, il faut qu'il y ait consommation d’espaces agricoles ou
naturels, et cette consommation est d'autant plus facile qu'il y a des espaces
disponibles.

L’espace agricole est encore perçu par une majorité d’acteurs comme un tiers
espace, associé à de faibles enjeux et disponible pour des usages à plus forte
valeur ajoutée. L’étalement urbain pose donc la question de la place allouée à
l’agriculture dans les projets de territoire.

Le choix du terme « consommation », qui s'est imposé dans le discours public
et a été repris dans les dernières lois agricoles, est significatif. Il
désigne la transformation et par conséquent la disparition d'une ressource, le
sol, qui n'est pas renouvelable, et le risque d'épuisement de cette ressource.
La consommation d'espaces agricoles touche en effet pour un tiers des sols de
très bonne qualité agronomique \cite[antoni_lartificialisation_2011].

Cette perte est mise en relation avec les réflexions sur une pénurie future de
ressources alimentaires et de terres cultivées, dans un contexte d’échanges
mondialisés, d’interdépendances multi-échelles et de changement climatique,
avec l’apparition depuis quelques années d’une plus grande volatilité sur les
marchés agricoles et de tensions sur l’accès au foncier agricole (« land
grabbing »).

Le mot consommation pourrait également référer à une critique post-
productiviste de la société de consommation, perçue comme essentiellement non
durable. L’artificialisation des terres contrevient en effet aux principes du
développement durable, qui préconisent d’éviter les irréversibilités, de
découpler le développement des ressources naturelles et des facteurs primaires
de production, et de payer les vrais coûts \cite[sainteny_letalement_2008].

L’étalement urbain n’est que l’une des causes de l’artificialisation du sol

L'étalement urbain consomme des espaces agricoles et naturels en les
artificialisant. Pour autant, l'artificialisation des sols ne se limite pas à
l'étalement urbain.

Est artificiel ce qui est produit par une technique humaine et non par la
nature, mais également ce qui se substitue à un élément naturel. Ainsi,
l’étalement urbain dénature les espaces agricoles et naturels en leur
substituant de manière peut-être irréversible une nouvelle occupation du sol
beaucoup plus fortement anthropisée.

Est artificiel aussi ce qui manque d’authenticité. On ne peut pas dans le
débat sur l'artificialisation des terres et l'étalement urbain ignorer
complètement les perceptions et les représentations des acteurs, qui sont
éventuellement contradictoires, conflictuelles et qui évoluent.

L'artificialisation a des conséquences concrètes sur le fonctionnement
hydrologique et écologique d’un territoire et conduit à des pertes
irréversibles de potentialités. C'est le cas notamment quand
l'artificialisation a pour conséquences l'imperméabilisation des surfaces,
mais aussi lorsque les pratiques agricoles viennent perturber le
fonctionnement biologique des sols ou les équilibres naturels.

L'artificialisation n'est pas le propre de l'espace urbain, et au delà de la
dualité urbain/rural, ville/campagne, artificiel/naturel, elle n'est pas qu'un
problème quantitatif qui se mesurerait en hectares consommés mais c'est
également un problème qualitatif qui demande d'évaluer le degré
d'artificialisation des milieux qui, du moins en France et en Europe, sont
tous soumis à une influence anthropique plus ou ou moins forte.

\subsection
{L’enjeu social d’une agriculture de proximité}

Les paysages de l’agriculture qui côtoient la ville étalée d’aujourd’hui ne
répondent pas aux attentes des citadins, et les agriculteurs, de moins en
moins nombreux, ne suffisent plus à tisser un lien social qui existaient
autrefois entre ville et campagne voisine \cite[vidal_entre_2011].
On peut supposer que
ce constat tiré en Île-de-France se généralise dans la plupart des métropoles
françaises.

Le besoin de renouer avec la proximité entre ville et agriculture est alimenté
par trois tendances qui remettent en cause le productivisme et pourrait faire
advenir « nouveau paradigme autour d’une ville et d’une agriculture
réconciliées », une « agriculture de reliance » \cite[poulot_agriculture_2010] :

\startitemize

\item les préoccupation environnementales et la sensibilisation au développement
durable font refuser la transformation irréversibles des milieux ;

\item l’étalement urbain, en amenant la ville à la campagne, a rendu inévitable le
partage de l’espace, contrecarre la logique de ségrégation et de
spécialisation et oblige à un nouveau dialogue ;

\item les conséquences des scandales sanitaires répétés ont profondément détérioré
la confiance dans les filières agro-alimentaires conventionnelles et
expliquent en partie, avec l’importance accordée en France en particulier à
l’alimentation, le succès des productions biologiques , des AMAP et des
circuits courts.

\stopitemize

Dans cette proximité, ce n’est pas tant la distance qu’une relation de
confiance qui compte. Celle-ci peut très bien se construire entre des
agriculteurs et des citadins qui ne seront pas forcément des voisins
\cite[vidal_entre_2011].

L’engouement actuel pour les jardins partagés et l’agriculture urbaine, qui
existe depuis toujours dans les pays émergents où elle joue parfois un rôle
majeur dans l’alimentation des populations, fait s’interroger sur leur
pertinence économique en France et se demander s’il ne s’agit pas que d’un
caprice des « bobos ». Plus qu’à une fonction nourricière et à des impératifs
économiques, ce lien renouvelé entre agriculture et citadins semble répondre à
des besoins d’aménités environnementales, de paysage et au désir de nature des
citadins, et au delà de ces dimensions esthétiques, il remplit une fonction
sociale \cite[torre_quand_2013] et identitaire \cite[bredif_reevaluer_2012].

Cette agriculture de proximité fournit un espace propice pour les projets
d’économie sociale et solidaire rendus davantage nécessaires par le
prolongement de la crise économique, stimulés par le soutien des collectivités
territoriales et les défis du développement durable. Entre les jardins
collectifs et la grande exploitation industrielle, la proximité peut être le
lieu de formes intermédiaires et innovantes d’agriculture qui dans leur
rapport à la terre ne sont « pas seulement un mode de production économique
mais aussi un modèle social et culturel » \cite[wolton_territoire_2015].

\subsection
{La planification comme principale réponse de l’État}

La planification est le principal levier mis en avant par l’État et ses
services pour lutter contre la consommation excessive de foncier agricole.
Notons que celui-ci est aujourd’hui davantage protégé par le code de
l’urbanisme, à travers une prise en compte des enjeux dans les documents
d’urbanisme, que par le code rural dont certains dispositifs, comme les zones
agricoles protégées (ZAP), ne rencontrent pas l’adhésion des acteurs qui
pourraient les mettre en œuvre.

Cependant, alors que la foi dans les vertus de la planification reste
culturellement bien ancrée en France, les intérêts contradictoires des acteurs
locaux compromettent parfois la cohérence voulue dans les SCoT, tandis que
l’accumulation des textes et la « furia normative » traduisent paradoxalement
la difficulté à répondre aux enjeux \cite[souchard_sortie_2013].

Les documents
d’urbanisme n’assurent pas une protection efficace de la biodiversité
ordinaire, des espaces agricoles et des milieux naturels \cite[delattre_ecologiser_2011].
Le ralentissement de l’artificialisation constatée à partir de 2008 est
principalement imputable au ralentissement économique \cite[_panorama_2014].

L’importance donnée à la propriété privée dans le droit français complique
singulièrement la gouvernance foncière et les possibilités d’intervention de
la puissance publique \cite[bertrand_terres_2013,renard_les_2013].
La fiscalité en vigueur
encourage quant à elle les mauvaises pratiques \cite[sainteny_letalement_2008].

Une troisième voie, qui s’appuierait sur les principes de l’économie du bien
commun et l’émergence d’une gouvernance innovante, apparaît aujourd’hui comme
un complément nécessaire à l’aménagement planifié et à l’intervention publique
sur les marchés fonciers, dans un contexte de forte évolution des institutions
\cite[bertrand_quelle_2006].

\subsection
{L’évolution du contexte institutionnel}

Le contexte du débat sur la consommation d’espaces agricoles et naturels est
aussi celui de l’affaiblissement de la représentation de la profession
agricole, qui compte aujourd’hui moins de 2~\% de la population active.
C’est aussi, avec la réforme territoriale, celui de la remise en cause d’une
organisation institutionnelle favorable au monde agricole, tandis que l’État
aménageur disparaît au profit de « l’État-plateforme » \cite[viard_dire_2015]. Le
transfert fin 2014 de la gestion du FEADER aux régions est à ce titre
significatif. Malgré cette évolution et de manière inattendue, les services de
l’État se retrouvent de plus en plus souvent en position d’arbitres entre des
acteurs locaux aux positions contradictoires.

Les collectivités territoriales, à qui incombe désormais la coordination des
politiques d’aménagement du territoire, ont besoin de rapprocher les acteurs,
de décloisonner les démarches et de garantir la solidarité interrégionale et
intrarégionale. Pour ce faire, elles réclament davantage d’outils
d’observation du territoire \cite[rousset_roles_2015]

\subsection
{Prise en compte de la consommation d’espaces agricoles et naturels par le
ministère chargé de l’agriculture}

La protection des espaces agricoles ou à vocation agricole, naturels ou
forestiers est l’une des 14 mesures clés de la loi d’avenir pour l’agriculture
adoptée fin 2014.

La loi de modernisation agricole de 2010 a créé l’observatoire national de la
consommation des espaces agricoles (ONCEA) et les commissions départementales
de consommation des espaces agricoles (CDCEA), renforcés depuis par la loi
d’avenir pour l’agriculture de 2014.

Dans ce cadre, le ministère chargé de l’agriculture a produit un panorama des
méthodes de quantification de la consommation d’espaces agricoles. Ce rapport
\cite[_panorama_2014] suggère de développer les liens entre observatoires à
différentes échelles et d’élargir la réflexion aux espaces dits naturels et
forestiers.

Il préconise également d’approfondir les questions concernant :

\startitemize

\item l’avenir de l’agriculture périurbaine, dans une logique d’intégration ville-
campagne prenant en compte les enjeux alimentaires et énergétiques ;

\item la qualité des espaces agricoles préservés et à préserver, en tenant davantage
compte de leur valeur agronomique ;

\item les conditions qui permettent d’assurer la viabilité de l’activité agricole et
le bon fonctionnement des fonctionnalités écologiques ;

\item et enfin les compensations environnementales et agricoles.

\stopitemize

\subsection
{Les instruments techniques et les observatoires territoriaux}

Les observatoires territoriaux se sont multipliés depuis quelques années, à
différentes échelles (nationale, régionale, départementale, locale), de
manière parfois désordonnée ; ils peuvent couvrir des thématiques très
diverses et ont encore besoin de mieux se structurer pour remplir leurs
fonctions \cite[feyt_les_2011].

En matière d’information géographique, l’instauration d’observatoires
territoriaux constitue l’évolution naturelle des plateformes régionales
d’information géographique qui se sont mises en place avec succès dans la
plupart des régions au cours de la dernière décennie. Ces plateformes se sont
développées d’abord sous l’impulsion des services de l’État, dans leur rôle de
« porter à connaissance » et pour favoriser les échanges de données entre
administrations et avec les collectivités. Cependant, si l’implication des
collectivités territoriales dans ces systèmes d’information progresse, elle
reste faible.

Pourtant, les nouvelles exigences du cadre législatif contraignent de plus en
plus les collectivités territoriales et les établissements publics à
surveiller les évolutions de l’utilisation du sol sur leur territoire (Dodane,
2014). La mise en place de dispositif d’observation et la recherche de
méthodes de mesure et de suivi adaptées en découlent logiquement mais se
heurtent à de nombreuses difficultés dont la première pourrait être
l’identification précise des enjeux et la formulation même du problème.

\section
{Mesure de la consommation d’espaces agricoles, naturels et forestiers}

\subsection
{Des indicateurs en profusion qui font débat}

Plusieurs approches de la consommation d’espaces agricoles par la mesure et
les indicateurs ont été proposées. De nombreux travaux se sont concentrés sur
la mesure de l’étalement urbain à partir de l’observation de l’occupation du
sol. Le CGDD recense 82 indicateurs regroupés en 23 familles « pour explorer
les différentes facettes du sujet de l’étalement urbain » (CGDD, 2012).       % TODO missing ref

D’autres chercheurs relèvent l’importance des mécanismes fonciers et ont
proposé par exemple l’indice de perturbation du marché des terres agricoles
comme indicateur de la pression sur le foncier agricole \cite[briquel_indicateurs_2013].

Face à cette profusion d’indicateurs, la construction de systèmes
d’indicateurs cohérents et adaptés aux questions posées demande encore à être
approfondie. \cite[balestrat_reconnaissance_2011] a proposé d’objectiver le débat sur
l’artificialisation par une méthode de mesure de la tâche artificialisée à
partir de données satellitaires haute résolution couplée à un indicateur de
qualité agronomique des sols.

Néanmoins, les nombreuses controverses que suscitent ces instruments de mesure
\cite[charmes_artificialisation_2013,martin-scholz_quand_2013] rappellent que les indicateurs sont
sujets à interprétation et que leur utilisation pose elle-même problème ; ils
doivent avant tout « nourrir un débat sur les enjeux du territoire, sur les
questions et les difficultés qu’ils soulèvent et la façon d’y répondre »
\cite[briquel_indicateurs_2013].

Enfin, l’évaluation de la qualité des données spatiales et de leur adéquation
aux besoins des utilisateurs (« fitness for use ») reste une question en soi
et reçoit de plus en plus d’attention de la communauté scientifique
\cite[devillers_thirty_2010].

\subsection
{Données sur l’occupation des sols}

Des sources de données de plus en plus nombreuses sont aujourd’hui disponibles
pour produire des indicateurs. Malheureusement, ces différentes sources de
données donnent une mesure divergente du phénomène de consommation d’espaces
agricoles \cite[_panorama_2014]. À l’échelle infrarégionale, les données spatiales
disponibles dans les bases de données existantes ne donnent pas de réponse
fiable pour un pas de temps décennal \cite[bousquet_les_2013].

Parmi les bases de données disponibles ou bientôt disponibles à l’échelle
nationale, citons :

\startitemize

\item Corine Land Cover

\item l’enquête TERUTI-LUCAS

\item le registre parcellaire graphique (RPG)

\item les couvertures satellitaires haute résolution annuelles GéoSUD

\item le projet de cartographie de l’occupation des sols à grande échelle de l’IGN
(OCS GE).

\stopitemize

Le rapprochement de données sur l’occupation du sol d’échelles différentes
dans une base de données nationale pose encore de nombreux problèmes d’ordre
méthodologique et pratique.

L’apport des outils collaboratifs à une connaissance plus précise de
l’occupation et de l’utilisation du sol n’a pas encore été exploré. Ces outils
permettraient d’enrichir et de valider la connaissance existante dans une
approche de « community sourced geographic information » ou de « volunteered
geographic information », mais aussi de développer des modes de production de
l’information plus innovants qui tirent parti de l’implication grandissante
des utilisateurs tandis que les chaînes de production traditionnelles
affichent des limites en terme de coût et de capacité à traiter une grande
quantité de données hétérogènes.

\section
{Méthodes : analyse spatiale, modélisation et prospective}

\subsection
{De la prospective territoriale à la géoprospective}

Les indicateurs d’occupation du sol et de suivi de l’étalement urbain sont
utiles dans la mesure où ils nourrissent le « débat sur les enjeux du
territoire, sur les questions et les difficultés qu’ils soulèvent et la façon
d’y répondre » \cite[briquel_indicateurs_2013].
Ils devraient donc avant tout être mobilisés
dans une démarche prospective, bien qu’ils puissent aussi, une fois les enjeux
correctement identifiés, servir à évaluer l’efficacité des actions et des
politiques publiques.

La prospective a émergé en France à la fin des années 1950 et consiste à
porter un regard sur l’avenir destiné à éclairer l’action présente à partir de
l’étude des facteurs historiques (la rétrospective), des forces en jeu et de
leurs inter-relations (De Jouvenel, 1999). L’objectif n’est pas de prédire
l’avenir mais d’explorer des scénarios possibles pour identifier les risques
et les opportunités, prendre des décisions adaptées qui préparent un avenir
désiré. La prospective territoriale, application de la prospective au devenir
d’un territoire, est devenue un outil communément utilisé par les
scientifiques et les professionnels de l’aménagement.

L’analyse territoriale et la cartographie à dire d’acteurs ont largement été
mobilisées depuis une vingtaine d’années dans les démarches de territoire et
la planification, jusqu’à inventer un nouveau champ de compétences qu’on
désigne aujourd’hui par ingénierie territoriale (Lardon, Piveteau). Cependant,   %% TODO Missing Ref
sans le support d’une objectivation des enjeux et des processus, il est
souvent difficile de prendre en compte les effets d’échelle et de dépasser la
simple confrontation de points de vue (Maurel, 2012).                            %% TODO Missing Ref

La géoprospective est née plus récemment du souhait d’expliciter davantage la
dimension spatiale par rapport à la simple prospective territoriale. La
géoprospective a pour objectif spécifique d’intégrer de la dimension spatiale
aux différents stades du processus prospectif, de chercher à comprendre et
prendre en compte des dynamiques et interactions spatiales dans les scénarios
prospectifs (Gourmelon, 2012). Pour ce faire, la géoprospective introduit des   %% TODO Missing Ref
méthodes de modélisation et de simulation spatiale dans la prospective
territoriale.

\citet{dodane_simuler_2014} ont observé que la géoprospective pouvait être « utile
pour faciliter les échanges entre les experts de différents secteurs sur un
même territoire ou les experts de différents territoires contigus ou sécant ».
Cette approche présente un intérêt pour expérimenter des idées et tester des
hypothèses sur les interactions spatiales et les trajectoires territoriales
afin « d’aider les élus à prendre conscience de l’impact des politiques
d’aménagement sur lesquelles ils réfléchissent ». Toutefois, la complexité
actuelle des outils de simulation et la difficulté à les manipuler pour des
non-experts limitent encore le public auquel la géoprospective peut
s’adresser.

\subsection
{La modélisation comme processus de recherche}

Les géographes se sont depuis longtemps attachés à décrire le vocabulaire et
la syntaxe des organisations spatiales et ont proposé des représentations des
dynamiques territoriales sous la forme de chorèmes (Brunet, 1980). Sous         %% TODO Missing Ref
l’angle de la théorie des systèmes, ils ont proposé des modèles qui expliquent
la production de l’espace par des éléments (ressources, moyens de productions,
populations, capital, information) qui interagissent et qui, en interagissant,
font émerger des structures qui donnent formes aux territoires et aux paysages
\cite[brunet_dechiffrement_2001].

La cartographie à dire d’acteurs permet de construire une représentation
partagée du territoire à partir des perceptions qu’ont les acteurs des
hétérogénéités fonctionnelles et des potentialités (Benoît, 2006). Elle
s’appuie largement sur les chorèmes et c’est un exemple de modélisation
iconique (ou graphique) dont le but est d’expliquer et de représenter la
structure du territoire, ses enjeux et ses dynamiques.

Modéliser, du point de vue de la pratique géographique, c'est « rechercher
quelle composition de modèles rend le mieux compte d'une organisation
régionale ou locale, d'une configuration de champ ou de réseau, ou d'une
distribution spatiale » \cite[brunet_modeles_2000]. La construction de modèles permet de
formuler des hypothèses pour construire une information qui apporte un
éclairage original et utile \cite[mathian_objets_2014].

La géographie quantitative fournit des outils de mesure basés sur l’analyse
spatiale en combinant systèmes d’information géographique et outils
statistiques \cite[pumain_les_2001,bavaud_handbook_2009,sanders_models_2010].
Les méthodes de
l’économétrie spatiale \cite[lesage_introduction_2009,vignes_fiches_2013] permettent d’exploiter
des données d’échelles différentes et d’analyser les processus spatiaux et les
flux mais aussi de répondre à des questions d’optimisation liée à la
localisation ou encore de quantifier des externalités.

Les modèles prédictifs permettent de simuler des phénomènes territoriaux. Les
modèles basés sur les automates cellulaires, dont un exemple pour la
simulation de l’évolution de l’occupation du sol est le modèle CLUE \cite[verburg_modeling_2002],
permettent de rendre compte de manière tendancielle des processus
d’évolution sans toutefois les expliciter. Ils peuvent être construits à
l’aide des méthodes d’intelligence artificielle ou de fouille de données.

D’autres formalismes ont pour but de mettre en évidence les propriétés du
système territorial à partir des comportements individuels des acteurs
(systèmes multi-agents) ou de rendre compte explicitement des processus et des
interactions. Pour ce dernier besoin, l’UMR TETIS a développé une plateforme
de simulation des dynamiques territoriales qui se fonde sur le formalisme des
graphes et propose un langage spécifique de modélisation destiné pour
représenter et simuler des dynamiques territoriales explicitement
spatialisées. Ce formalisme rend ainsi possible d’expliciter les processus et
les relations entre les acteurs et les objets du système territorial
\cite[degenne_approche_2012,castets_integration_2014].

\subsection
{Des modèles comme support à la prospective : la modélisation d’accompagnement}

Le modèle support de l’analyse et du questionnement scientifique peut aussi
devenir le support à la concertation et aux démarches participatives.

C’est ce que propose la modélisation d’accompagnement pour aborder les
problèmes complexes de gestion environnementale et de gestion collective des
ressources renouvelables \cite[etienne_modelisation_2010, etienne_modelisation_2012].
Elle exploite les modèles comme des
objets intermédiaires qui servent de support au dialogue entre les acteurs et
à l’apprentissage collectif et organisationnel ; elle permet ainsi une
construction itérative et adaptative des décisions.

Les outils numériques, et en particulier les outils collaboratifs, pourraient
être davantage mobilisés pour construire une connaissance partagée du
territoire. Ils permettraient de dépasser les limites actuelles des démarches
d’observatoire et d’impliquer les acteurs dans la réponse aux enjeux. Ils
aideraient les acteurs historiques et nouveaux à se positionner et à
interagir. Cependant, l’extension de l’utilisation d’outils mathématiques et
informatiques complexes à des arènes participatives intégrant des élus et des
citoyens demande prudence, car l’usage des techniques de simulation et la
diffusion des résultats peuvent s’avérer délicats \cite[dodane_simuler_2014].
