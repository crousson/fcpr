\section
{Contexte}

\subsection
{La consommation d'espaces agricoles :
 une prise en compte récente dans les politiques publiques}

« Consommation d'espaces agricoles », « étalement urbain »,
« artificialisation des terres ou du sol » renvoient à une même problématique
sans être synonymes et traduisent des points de vue différents mais
complémentaires.

L’« étalement urbain » (en anglais « urban sprawl ») définit une
problématique d’aménagement du territoire caractérisée par un changement
d’occupation du sol dû à l’urbanisation à la fois irréversible et plus rapide
que la croissance démographique \cite[certu_consommation_2010].

L’étalement urbain, sans être un phénomène nouveau, s’est accéléré dans la
deuxième moitié du XXème siècle. Il n’a cependant bénéficié d’une prise en
compte au niveau européen et d’une forte médiatisation que depuis une dizaine
d’années.

De 1992 à 2004, ce sont entre 50 000 et 60 000 ha de terres
qui ont été artificialisées par an en France \cite[chakir_analyse_2006].
L’artificialisation progresse quatre fois plus rapidement que la
croissance démographique \cite[balny_proteger_2009].
La consommation d'espaces agricoles touche pour un tiers des sols de
très bonne qualité agronomique \cite[antoni_lartificialisation_2011].
Ces chiffres ont marqué les autorités publiques comme l’opinion.

Cette perte est mise en relation avec les réflexions sur une pénurie future de
ressources alimentaires et de terres cultivées, dans un contexte d’échanges
mondialisés, d’interdépendances multi-échelles et de changement climatique,
avec l’apparition depuis quelques années d’une plus grande volatilité sur les
marchés agricoles et de tensions sur l’accès au foncier agricole (« land
grabbing »).

En France, le territoire reste globalement peu artificialisé comparé à
d’autres régions européennes : entre 5~\% et 8~\% de la surface totale en 2006
selon les sources contre 37~\% aux Pays-Bas par exemple.
C’est bien la dynamique de l’artificialisation qui interpelle les politiques publiques.

Ces préoccupations ont été traduites en France à partir de 2000 dans le droit
rural, le droit de l’urbanisme et aujourd’hui le droit de l’environnement (SRU
2000, DTR 2005, ENE 2010), élevant la maîtrise de la consommation
des terres agricoles au rang de nouvelle norme environnementale
\cite[bertrand_quelle_2006, bertrand_terres_2013].

La préservation des espaces naturels, agricoles et forestiers
contre les effets de l'étalement urbain et de l'artificialisation
a été inscrite dans la Loi de modernisation agricole en 2010
et dans la Loi d'avenir pour l'agriculture en 2014.


\subsection
{L’enjeu d’une agriculture de proximité}

Les paysages de l’agriculture qui côtoient la ville étalée d’aujourd’hui ne
répondent pas aux attentes des citadins, et les agriculteurs, de moins en
moins nombreux, ne suffisent plus à tisser un lien social qui existaient
autrefois entre ville et campagne voisine \cite[vidal_entre_2011].
%On peut supposer que
%ce constat tiré en Île-de-France se généralise dans la plupart des métropoles
%françaises.

L’engouement actuel pour l’agriculture urbaine, qui
existe depuis toujours dans les pays émergents où elle joue parfois un rôle
majeur dans l’alimentation des populations, fait s’interroger sur sa
pertinence économique en France et se demander s’il ne s’agit pas que d’un
caprice des « bobos ».

Plus qu’à une fonction nourricière et à des impératifs
économiques, ce lien renouvelé entre agriculture et citadins
ne se réduit pas à une demande pour plus nature et d'aménités environnementales.
Il remplit également une fonction sociale \cite[torre_quand_2013]
et identitaire \cite[bredif_reevaluer_2012].

Le besoin de renouer avec la proximité entre ville et agriculture est alimenté
par trois tendances qui mettent en cause le modèle productiviste et pourrait faire
advenir « nouveau paradigme autour d’une ville et d’une agriculture
réconciliées », une « agriculture de reliance » \cite[poulot_agriculture_2010] :

\startitemize

\item les préoccupation environnementales et la sensibilisation au développement
      durable font refuser la transformation irréversibles des milieux ;

\item l’étalement urbain, en amenant la ville à la campagne, a rendu inévitable le
      partage de l’espace, contrecarre la logique de ségrégation et de
      spécialisation et oblige à un nouveau dialogue ;

\item les conséquences des scandales sanitaires répétés ont profondément détérioré
      la confiance dans les filières agro-alimentaires conventionnelles et
      expliquent en partie, avec l’importance accordée en France en particulier à
      l’alimentation, le succès des productions biologiques , des AMAP et des
      circuits courts.

\stopitemize

Cette agriculture de proximité fournit un espace propice pour les projets
d’économie sociale et solidaire rendus davantage nécessaires par le
prolongement de la crise économique, stimulés par le soutien des collectivités
territoriales et les défis du développement durable. Entre les jardins
collectifs et la grande exploitation industrielle, la proximité peut être le
lieu de formes intermédiaires et innovantes d’agriculture qui dans leur
rapport à la terre ne sont « pas seulement un mode de production économique
mais aussi un modèle social et culturel » \cite[wolton_territoire_2015].


\subsection
{La planification comme principale réponse de l’État}

La planification est le principal levier mis en avant par l’État et ses
services pour lutter contre la consommation excessive de foncier agricole.
Celui-ci est paradoxalement davantage protégé par le code de
l’urbanisme, à travers une prise en compte des enjeux dans les documents
d’urbanisme, que par le code rural dont certains dispositifs, comme les zones
agricoles protégées (ZAP), ne rencontrent pas l’adhésion des acteurs qui
pourraient les mettre en œuvre.

Cependant, alors que la foi dans les vertus de la planification reste
culturellement bien ancrée en France, les intérêts contradictoires des acteurs
locaux compromettent parfois la cohérence voulue dans les SCoT, tandis que
l’accumulation des textes et la « furia normative » traduisent paradoxalement
la difficulté à répondre aux enjeux \cite[souchard_sortie_2013].

Les documents
d’urbanisme n’assurent pas une protection efficace de la biodiversité
ordinaire, des espaces agricoles et des milieux naturels \cite[delattre_ecologiser_2011].
Le ralentissement de l’artificialisation constatée à partir de 2008 serait
principalement imputable au ralentissement économique \cite[_panorama_2014].

L’importance donnée à la propriété privée dans le droit français complique
singulièrement la gouvernance foncière et les possibilités d’intervention de
la puissance publique \cite[bertrand_terres_2013,renard_les_2013].
La fiscalité en vigueur
encourage quant à elle les mauvaises pratiques \cite[sainteny_letalement_2008].

Une troisième voie, qui s’appuierait sur les principes de l’économie des biens
communs et l’émergence d’une gouvernance innovante, apparaît aujourd’hui comme
un complément nécessaire à l’aménagement planifié et à l’intervention publique
sur les marchés fonciers, dans un contexte de forte évolution des institutions
\cite[bertrand_quelle_2006].


\subsection
{L’évolution du contexte institutionnel}

Le contexte du débat sur la consommation d’espaces agricoles et naturels est
aussi celui de l’affaiblissement de la représentation de la profession
agricole, qui compte aujourd’hui moins de 2~\% de la population active.

C’est aussi, avec la réforme territoriale, celui de la remise en cause d’une
organisation institutionnelle favorable au monde agricole, tandis que l’État
aménageur disparaît au profit de « l’État-plateforme » \cite[viard_dire_2015]. Le
transfert fin 2014 de la gestion du FEADER aux régions est à ce titre
significatif.

Malgré cette évolution et de manière inattendue, les services de
l’État se retrouvent de plus en plus souvent en position d’arbitres entre des
acteurs locaux aux positions contradictoires.

Les collectivités territoriales, à qui incombe désormais la coordination des
politiques d’aménagement du territoire, ont besoin de rapprocher les acteurs,
de décloisonner les démarches et de garantir la solidarité interrégionale et
intrarégionale. Pour ce faire, elles réclament davantage d’outils
d’observation du territoire \cite[rousset_roles_2015].

Les nouvelles exigences du cadre législatif contraignent de plus en
plus les collectivités territoriales et les établissements publics à
surveiller les évolutions de l’utilisation du sol sur leur territoire
\cite[dodane_simuler_2014].
La mise en place de dispositif d’observation et la recherche de
méthodes de mesure et de suivi adaptées en découlent logiquement mais se
heurtent à de nombreuses difficultés comme
l’identification précise des enjeux ou la mise en place de moyens techniques
adaptés.


\subsection
{Les instruments techniques et les observatoires territoriaux}

Les observatoires territoriaux se sont multipliés depuis quelques années, à
différentes échelles (nationale, régionale, départementale, locale), de
manière parfois désordonnée ; ils peuvent couvrir des thématiques très
diverses et ont encore besoin de mieux se structurer pour remplir leurs
fonctions \cite[feyt_les_2011].

L’instauration d’observatoires
territoriaux constitue l’évolution naturelle des plateformes régionales
d’information géographique qui se sont mises en place avec succès dans la
plupart des régions au cours de la dernière décennie.

Ces plateformes se sont
développées d’abord sous l’impulsion des services de l’État, dans leur rôle de
« porter à connaissance » et pour favoriser les échanges de données entre
administrations et avec les collectivités. Cependant, si l’implication des
collectivités territoriales dans ces systèmes d’information progresse, elle
reste faible.

La loi de modernisation agricole de 2010 a créé l’observatoire national de la
consommation des espaces agricoles (ONCEA) et les commissions départementales
de consommation des espaces agricoles (CDCEA), renforcés depuis par la loi
d’avenir pour l’agriculture de 2014.

Dans ce cadre, le ministère chargé de l’agriculture a produit un panorama des
méthodes de quantification de la consommation d’espaces agricoles. Ce rapport
\cite[_panorama_2014] suggère de développer les liens entre observatoires à
différentes échelles et d’élargir la réflexion aux espaces dits naturels et
forestiers.

Il préconise également d’approfondir les questions concernant :

\startitemize

\item l’avenir de l’agriculture périurbaine, dans une logique d’intégration ville-
campagne prenant en compte les enjeux alimentaires et énergétiques ;

\item la qualité des espaces agricoles préservés et à préserver, en tenant davantage
compte de leur valeur agronomique ;

\item les conditions qui permettent d’assurer la viabilité de l’activité agricole et
le bon fonctionnement des fonctionnalités écologiques ;

\item et enfin les compensations environnementales et agricoles.

\stopitemize



\subsection
{Des indicateurs en profusion qui font débat}

De nombreux travaux se sont concentrés sur
la mesure de l’étalement urbain à partir de l’observation de l’occupation du
sol. Le CGDD recense 82 indicateurs regroupés en 23 familles « pour explorer
les différentes facettes du sujet de l’étalement urbain » (CGDD, 2012).       % TODO missing ref

D’autres chercheurs relèvent l’importance des mécanismes fonciers et ont
proposé par exemple l’indice de perturbation du marché des terres agricoles
comme indicateur de la pression sur le foncier agricole \cite[briquel_indicateurs_2013].

Face à cette profusion d’indicateurs, la construction de systèmes
d’indicateurs cohérents et adaptés aux questions posées demande encore à être
approfondie. \cite[balestrat_reconnaissance_2011] a proposé d’objectiver le débat sur
l’artificialisation par une méthode de mesure de la tâche artificialisée à
partir de données satellitaires haute résolution couplée à un indicateur de
qualité agronomique des sols.

Les indicateurs sont sujets à interprétation et leur utilisation pose elle-même problème ;
ils doivent avant tout « nourrir un débat sur les enjeux du territoire, sur les
questions et les difficultés qu’ils soulèvent et la façon d’y répondre »
\cite[briquel_indicateurs_2013].

Enfin, l’évaluation de la qualité des données spatiales et de leur adéquation
aux besoins des utilisateurs (« fitness for use ») reste une question en soi
et reçoit de plus en plus d’attention de la communauté scientifique
\cite[devillers_thirty_2010].