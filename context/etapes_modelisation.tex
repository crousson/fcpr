\section{Processus de modélisation : du problème à l'expérimentation}

\citet{magliocca_metastudies_2015} propose une démarche générale de modélisation,
synthétisée à partir de la littérature sur les changements d'occupation du sol.

\startplacefigure[location=middle,title={Processus de modélisation}]
  \externalfigure[../svg/magliocca_methodo.svg][width=0.9\textwidth]
\stopplacefigure

Cette démarche identifie les différentes étapes d'un effort de modélisation :

\startitemize[n]

\item le point de départ de tout processus de modélisation
   est la question de recherche et sa problématisation,
   qui définit le phénomène spatial auquel on s'intéresse
   ainsi que le contexte et l'échelle à laquelle on souhaite l'étudier,
   son éventuel rapport avec des processus d'échelle plus large ;

\item la description du problème permet d'identifier les limites du système,
   ses composants (acteurs, variables, processus)
   et les possibles relations qu'ils entretiennent.
   La littérature apporte
   un appui important pour identifier les composants et
   les décrire en même temps qu'elle peut apporter
   des éléments de théorie pour les mettre en relation.
   À cette étape, la cartographie à dire d'acteurs peut
   servir également à identifier ces éléments et à conceptualiser
   le système étudié.

\item le modèle conceptuel est ensuite traduit en modèle dynamique
   et implémenté dans un programme informatique
   en sélectionnant une représentation appropriée ;
   l'utilisation d'une plateforme de modélisation
   adaptée au type de modèle que l'on souhaite instancier
   facilite la réalisation de cette étape ;

\item le modèle est ensuite calibré et validé à l'aide des données
   du terrain d'études ; cette étape permet de tester la capacité
   du modèle à rendre compte des observations ainsi
   que sa sensibilité aux données en entrée ;

\item enfin, le modèle peut être exploité pour
   tester des hypothèses complémentaires, explorer
   différents scénarios ou être confronté à
   ses utilisateurs potentiels. Cette étape d'expérimentation
   permet de critiquer le modèle et de raffiner l'analyse du
   problème initial.

\stopitemize