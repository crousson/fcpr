\section
{Conclusion}

Nous avons présenté un projet de recherche sur les dynamiques périurbaines et
la place de l’agriculture dans ces dynamiques. Si l’agriculture périurbaine et
la problématique de l’étalement urbain ont reçu de plus en plus d’attention
récemment, nous pensons que l’originalité de notre proposition réside en deux
endroits :

\startitemize

\item d’abord, dans la tentative de combiner une approche qualitative, celle de la
cartographie à dire d’acteurs, avec une analyse quantitative pour confronter
l’approche constructiviste aux données et indicateurs disponibles, et dépasser
la simple confrontation des points de vue dans un débat davantage objectivé
(tout étant conscient des problèmes de traduction et des limites de
l’objectivation) ;

\item ensuite, dans le recours à la modélisation comme procédé de construction d’une
représentation partagée des processus territoriaux et comme support aux
démarches participatives.

\stopitemize

Notre projet met en avant les apports des outils numériques à la médiation
territoriale à travers la géoprospective. Il s'inscrit plus généralement dans
la perspective du développement des observatoires territoriaux à l’échelle
locale, régionale et nationale. En restant critiques sur le discours de la «
smart city », notre propos est davantage de reconnaître l’émergence de la
société numérique comme une évolution importante et de tirer parti au mieux de
cette évolution et des données toujours plus nombreuses dont nous disposons.

Ce projet est encore perfectible mais nous espérons néanmoins qu’il apportera
une contribution utile à la réflexion sur la consommation d’espaces agricoles
et naturels et sur la construction du territoire en zones périurbaines, alors
que la place et la fonction de l’agriculture apparaissent comme une
problématique structurante, en obligeant à dépasser les approches sectorielles
et l’opposition des choix individuels.