\section{Apports de la modélisation}

\citet{lemoisson_cadre_2012} ont formalisé dans une démarche baptisée Co-Obs 
un processus dans lequel la modélisation des dynamiques spatiales à partir d'un diagnostic
de territoire permet de produire un modèle d'action (un programme ou un projet d'action)
et un modèle d'observation (qui accompagne l'action et en découle).

La démarche accompagne et s'insère dans le cycle de mise en place et de fonctionnement d'un observatoire territorial.
La modélisation opérationnalise le diagnostic de territoire pour en faire un outil de prospective
et d'évaluation des actions mises en œuvre.
La co-construction des modèles fournit un cadre réflexif pour déduire un système d'indicateurs adaptés à la problématique
territoriale et d'identifier les services informationnels qui sont utiles aux acteurs.
Réciproquement, l’information produite par l’observatoire et sa mise en débat au sein de la communauté 
d’acteurs favorisent la production de nouvelles connaissances.

La modélisation nous paraît en effet indispensable pour dépasser
une approche descriptive des dynamiques territoriales
et expliquer les processus en œuvre.
Le premier enjeu de la modélisation est d'expliciter les relations sémantiques
et causales entre les variables représentées par les indicateurs.

Le modèle devient alors un objet intermédiaire
qui aide à la lecture de la complexité du monde réel,
en permettant :

\startitemize[packed]

\item de formuler et de tester des hypothèses sur des phénomènes
  qui sont difficilement observables directement
\item de permettre des changements d'échelle et de mettre en rapport dynamiques globales
  et changements locaux
\item d'identifier des régularités pour formuler des lois
  à partir d'observations empiriques
\item de comparer des situations différentes en explicitant
  ce qui est générique et en le distinguant du spécifique.

\stopitemize

Cadre heuristique, théorie et modèle sont liés par une relation qui va du général au particulier.
Les cadres heuristiques correspondent à la forme la plus générale d’analyse théorique.
Les théories permettent de formuler des hypothèses de travail sur les relations entre
les éléments pertinents du cadre heuristique, tandis que le modèle permet
de tester ces hypothèses dans une situation réelle \cite[ostrom_background_2011].

Le modèle a donc une portée opérationnelle et spécifique.
C'est un dispositif expérimental qui rend possible l'exploration
de scénarios et permet d'apprendre à partir des écarts constatés entre
ce qui est prédit par le modèle et ce qui est observé.

La modélisation d’accompagnement, ou « companion modelling »,
exploite cette acception opérationnelle du modèle : c'est une méthode participative
basée sur les jeux de rôles et la simulation, utilisés pour servir de support
au dialogue entre les acteurs ;
elle encourage l’apprentissage collectif et organisationnel et
permet ainsi une construction itérative et adaptative des décisions
\cite[etienne_modelisation_2012].

% L'intégration des indicateurs dans une approche globale
% permet de construire des systèmes d'indicateurs
% plus pertinents pour le diagnostic territorial
% et l'évaluation des politiques publiques.