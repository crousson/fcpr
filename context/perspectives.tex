\section[suites]
{Mise en perspective du projet de thèse dans le projet professionnel}

\subsection
{Compétences acquises grâce à la thèse et positionnement}

À travers ce projet, mon ambition est de relier recherche et opérationnel dans
un aller-retour vertueux où les questions de l’action publique interpellent la
recherche et où en retour la recherche permet l’expérimentation et contribue
au développement de l’action publique et à l'innovation sociale.

Dans mon domaine spécifique, il s'agit d'élargir mes compétences à des
disciplines complémentaires, en particulier en sciences sociales, pour être
davantage en capacité d'intervenir à l'interface entre la production de
connaissances nouvelles, les démarches d'innovation et de transformation et la
réponse aux besoins opérationnels d'évaluation des politiques publiques et de
prospective territoriale.

Dans une ambition d'excellence technique et méthodologique,
la thèse sera également le moyen de consolider ma connaissance
des méthodes statistiques et spatiales, de la télédétection
et de l'exploitation des données d'observation de la Terre,
et des outils de modélisation appliqués à l'ingénierie territoriale.

Les compétences nouvellement acquises me permettront d'exercer
ce rôle d'interface, qui reste la caractéristique des {\IPEF}
et qui est aujourd'hui indispensable pour dépasser les limites
de projets essentiellement techniques qui, quoique pertinents, n'aboutissent
pas. Ce positionnement pluridisciplinaire est un atout pour conduire des
projets qui potentialisent les apports du numérique et de
l'information spatialisée dans des usages qui soutiennent la gestion durable
des territoires et des ressources naturelles.

\subsection
{Intérêt pour l'employeur et pour le Corps des IPEF}

Ce projet crée immédiatement une opportunité de renforcer la collaboration
entre plusieurs services du ministère ({\DGPE}, {\SSP}, {\SDSI})
impliqués dans le suivi de la consommation
d’espaces agricoles et naturels, et de contribuer à une dynamique commune.
Il permettra d’accompagner la mise en place de 
l’observatoire de la consommation d’espaces naturels, agricoles et forestiers,
et de mettre en perspective les travaux nationaux par rapport aux
échelles locales, mais aussi d'avoir une réflexion sur le volet
valorisation de la nouvelle enquête {\sc teruti}
et les conditions de cette valorisation.

Ce projet participe également de la logique d'un
parcours d'expert et constitue un moyen de concrétiser et de
faire reconnaître une spécialisation dans le domaine
des systèmes d'information environnementaux
à un niveau d'expert. Cette démarche est cohérente avec
la note de service {\sc sg/srh/smec/2015-658} qui a inscrit les systèmes
d'information géographique dans la liste des thématiques d'expertise prioritaires
pour l'exercice des missions du ministère chargé de l'agriculture.
La thèse est un passage obligé vers la reconnaissance de cette expertise
en même temps qu'elle confère une légitimité scientifique
qui est aujourd'hui nécessaire aux {\IPEF} pour dialoguer
avec d'autres experts à l'international,
les collectivités territoriales ou la société civile.

Enfin, il me paraît important de mettre ce projet de thèse en perspective
du défi majeur que représente la transition énergétique 
pour notre société pour les prochaines décennies.
Les {\IPEF}, de par leurs compétences techniques
et scientifiques, sont appelés à y prendre part en étant capables d'accompagner
l'évolution des modes de vie et de consommation,
le renouvellement des institutions et le mouvement de la société civile,
qu'ils participent à la transformation de l'État vers l'État-plateforme
pour orchestrer et soutenir les initiatives locales,
qu'ils apportent leurs compétences aux collectivités territoriales
ou qu'ils s'impliquent dans la société civile.
Les technologies de l'information et du numérique seront dans cet effort
un levier important de transformation
et le support des nouveaux modes d'action organisée
qui tirent parti de l'intelligence collective
et du polycentrisme institutionnel.
Ces évolutions sont en même temps un défi pour les {\IPEF} eux-mêmes,
et ce projet de thèse, qui s'inscrit complètement
dans ces préoccupations, est une manière concrète de m'y préparer.


\subsection
{Trajectoire professionnelle envisagée}

La suite de mon parcours privilégiera l'appui aux politiques publiques
et sera réfléchie en fonction des résultats
acquis au cours de ce travail de recherche et de mon positionnement
à l'interface entre la communauté scientifique et les utilisateurs
des connaissances et méthodes que j'aurais développées.

J'envisage d'abord l'option d'être affecté à un service régional
de connaissance du territoire et d'évaluation des politiques publiques,
pour y mettre en pratique le savoir-faire acquis pendant la thèse
et avec un souci de diffusion méthodologique auprès des services d'autres régions.

Une autre option serait de rejoindre une collectivité territoriale. Cette option
s'inscrirait dans la politique d'essaimage encouragé par le Corps et serait
pertinent par rapport aux renforcements des compétences des collectivités,
en particulier des régions, dans le domaine agricole et de l'aménagement
du territoire.

Les compétences acquises pourraient également être valorisées :

\startitemize

\item en administration centrale dans des activités de prospective
	ou chez un opérateur chargé de la production de données environnementales
	et de leur valorisation~;

\item au sein d'un bureau métier sur une mission de maîtrise d'ouvrage du {\SI}
	ou de conduite de projet de transformation et d'innovation~;

\item dans un organisme international qui coordonne des politiques publiques ou opère des
	dispositifs d'appui à ces politiques.

\stopitemize

Ce projet devrait donc déboucher sur de nombreuses opportunités de valoriser
l'expertise acquise dans les domaines de compétences des collectivités
territoriales et des ministères techniques en charge de l'aménagement du territoire.