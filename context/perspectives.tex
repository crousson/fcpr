\section[suites]
{Mise en perspective du projet de thèse dans le projet professionnel}

\subsection
{Compétences acquises grâce à la thèse et positionnement}

À travers ce projet, mon ambition est de relier recherche et opérationnel dans
un aller-retour vertueux où les questions de l’action publique interpellent la
recherche et où en retour la recherche permet l’expérimentation et contribue
au développement de l’action publique et de modes de gouvernance adaptés aux
enjeux du développement durable.

Dans mon domaine spécifique, il s'agit d'élargir mes compétences à des
disciplines complémentaires, en particulier en sciences sociales, pour être
davantage en capacité d'intervenir à l'interface entre la production de
connaissances nouvelles, les démarches d'innovation et de transformation et la
réponse aux besoins opérationnels d'évaluation des politiques publiques et de
prospective territoriale.

Les compétences nouvellement acquises pendant la thèse me permettront d'exercer
ce rôle d'interface, qui est aujourd'hui nécessaire pour dépasser les limites
de projets essentiellement techniques qui, quoique pertinents, n'aboutissent
pas. Ce positionnement pluridisciplinaire est nécessaire pour conduire des
projets innovants qui potentialisent les apports du numérique et de
l'information spatialisée dans des usages qui soutiennent la gestion durable
du territoire et des ressources naturelles.

\subsection
{Intérêt pour l'employeur et pour le corps des IPEF}

[\dots]

\subsection
{Trajectoire professionnelle envisagée}

[\dots]