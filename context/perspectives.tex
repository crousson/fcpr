\section
{Mise en perspective du projet de thèse dans le projet professionnel}

\subsection
{Intérêt pour l'employeur et pour le corps des IPEF}

Le choix d'une question de recherche qui traite de la place de l'agriculture
dans le territoire est en phase avec un regain d'intérêt de l'opinion publique
mais aussi des institutions internationales pour l'agriculture et
l'alimentation ; celles-ci retrouvent en effet leur importance comme enjeu
stratégique à mesure que se développe la spéculation sur le foncier agricole
et que l'instabilité des marchés agricoles interroge la résilience de nos
systèmes de production.

Comme je l'ai rappelé en introduction, le contexte institutionnel, social et
technologique est aujourd'hui très favorable au développement des usages des
données d’observation de la Terre en appui aux politiques publiques agricoles,
d’aménagement du territoire et de développement durable ; en effet, une offre
de données abondantes et des solutions technologiques matures rencontrent un
besoin d'innovation tiré par les enjeux économiques et sociaux du
développement durable et des transformations institutionnelles importantes
(État-plateforme, réforme territoriale).

C'est pourquoi ce projet devrait déboucher sur de nombreuses opportunités de
valoriser les compétences acquises grâce à la thèse dans les domaines de
compétences des collectivités territoriales et des grands ministères
techniques en charge de l'aménagement du territoire.

Comme je l'ai aussi indiqué au début de ce document, ce projet participe
également de la logique d'un parcours d'expert et constitue un moyen de
concrétiser et de faire reconnaître une spécialisation dans le domaine des
systèmes d'information géographique à un niveau d'expert. Cette démarche est
cohérente avec la note de service SG/SRH/SMEC/2015-658 qui a inscrit les
systèmes d'information géographique dans la liste des thématiques d'expertise
prioritaires pour l'exercice des missions du ministère chargé de
l'agriculture.

La thèse est un passage obligé vers cette reconnaissance en même temps qu'elle
confère une légitimité au niveau international et me permettrait de rejoindre
à un moment ou un autre de ma carrière un organisme international qui
coordonne des politiques publiques ou opère des dispositifs d'appui à ces
politiques.

\subsection
{Compétences acquises grâce à la thèse et positionnement}

À travers ce projet, mon ambition est de relier recherche et opérationnel dans
un aller-retour vertueux où les questions de l’action publique interpellent la
recherche et où en retour la recherche permet l’expérimentation et contribue
au développement de l’action publique et de modes de gouvernance adaptés aux
enjeux du développement durable.

Dans mon domaine spécifique, il s'agit d'élargir mes compétences à des
disciplines complémentaires, en particulier en sciences sociales, pour être
davantage en capacité d'intervenir à l'interface entre la production de
connaissances nouvelles, les démarches d'innovation et de transformation et la
réponse aux besoins opérationnels d'évaluation des politiques publiques et de
prospective territoriale.

Les compétences nouvelles acquises pendant la thèse me permettront d'exercer
ce rôle d'interface, qui est aujourd'hui nécessaire pour dépasser les limites
de projets essentiellement techniques qui, quoique pertinents, n'aboutissent
pas. Ce positionnement pluridisciplinaire est nécessaire pour conduire des
projets innovants qui potentialisent les apports du numérique et de
l'information spatialisée dans des usages qui soutiennent la gestion durable
du territoire et des ressources naturelles.

\subsection
{Trajectoire professionnelle envisagée}

Une première option, après la thèse, serait de poursuivre dans un institut de
recherche appliquée qui exercent une mission d’appui aux politiques publiques
et intègre cet appui dans leur stratégie scientifique. Cette option sera
réfléchie, en fonction des résultats acquis au cours de ce travail de
recherche et de mon positionnement à l’interface entre la communauté
scientifique et les utilisateurs publics des connaissances et méthodes que
j’aurais développées. Il s’agirait d’approfondir le travail de recherche
réalisé et sa diffusion auprès des parties prenantes, précédemment
identifiées, autour de trois activités complémentaires :

\startitemize

\item l'appui aux politiques publiques, principalement par du transfert
méthodologique qui doit permettre de créer de la valeur en innovant et en
travaillant sur les usages grâce à l'apport de compétences pluridisciplinaires

\item la production de nouvelles connaissances sur les processus territoriaux et
leur représentation dans des modèles utiles à l'action organisée et à la
gouvernance collective

\item l'enseignement et le développement de ressources pédagogiques.

\stopitemize

Une deuxième option serait d’être affecté à un service régional de
connaissance du territoire et d'évaluation des politiques, pour y mettre en
pratique le savoir-faire acquis pendant la thèse et avec un souci de diffusion
méthodologique auprès des services d’autres régions.

Après la thèse ou après son prolongement dans un EPST ou un service régional,
je pourrai également valoriser les compétences acquises dans différents
contextes opérationnels :

\startitemize

\item en administration centrale dans des activités de prospective ou chez un
opérateur chargé de la production de données environnementales et de leur
valorisation ;

\item au sein d'un bureau métier sur une mission de maîtrise d'ouvrage du SI ou de
conduite de projet de transformation et d’innovation ;

\item au sein d'une collectivité territoriale.

\stopitemize

Rejoindre une collectivité territoriale s'inscrirait dans la politique
d'essaimage encouragé par le corps et serait pertinent par rapport aux
renforcements des compétences des collectivités, en particulier des régions,
dans le domaine agricole et de l'aménagement du territoire.
