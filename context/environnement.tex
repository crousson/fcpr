\section[environnement]
{Présentation de l'encadrement et de l'environnement de thèse}

\subsection{Laboratoires d'accueil}

Le projet de thèse sera accueilli à l'{\ENS} Lyon au sein l'{\UMR\ \EVS} et à la Maison de la Télédétection
à Montpellier dans l'équipe {\AMoS} (Analyse et modélisation spatiale) de l'{\UMR\ \TETIS}.

L'{\UMR} « Environnement, Ville, Société » a pour objectif scientifique la production de connaissance sur les modalités
par lesquelles les sociétés contemporaines - fortement urbanisées - constituent, instituent et utilisent leurs environnements.
À travers une approche pluridisciplinaire des relations entre nature, technique et société, les travaux de l'unité
portent sur les formes d'organisation des sociétés en lien avec leurs environnements.

L'{\UMR} « Territoires, Environnement, Télédétection et Information Spatiale » conduit des projets de recherche
sur les méthodes d'informations spatiales appliquées à l'environnement et au développement durable.
Le fil directeur des travaux de l'{\UMR} est l'élaboration et la valorisation d'une information spatialisée « utile, utilisable et utilisée »,
en traitant de bout en bout la chaîne de l'information spatiale, de l'observation de la Terre et
l'acquisition de données jusqu'à leur utilisation par les acteurs impliqués dans les démarches territoriales.
La stratégie de l'{\UMR} s'articule autour d'un équilibre entre recherche, formation initiale et continue,
et accompagnement des politiques publiques par l'expertise.


\subsection{Comité de thèse}

Le travail de thèse sera dirigé par \person{Anne Rivière-Honneger}. Géographe, directrice de recherche {\CNRS} à l'{\UMR\ \EVS},
Anne Rivière-Honneger étudie les interfaces homme-nature, les dynamiques des territoires,
les pratiques collectives de gestion de l'eau, et s'intéresse à la connaissance et à la gestion du paysage.

Le projet sera également supervisé par un comité de thèse auquel ont accepté de participer~:

\person{Christiane Weber} : directrice de recherche {\CNRS}, % directrice de l'{\UMR \TETIS}.
Christiane Weber a encadré de nombreux travaux de recherche portant sur les méthodes de l'observation spatiale
appliquées à l'exploration de la ville et l'étude de l'écologie humaine,
traitant de sujets aussi variés que les énergies renouvelables, la pollution de l'air, la santé humaine,
les formes urbaines et les mobilités.

\person{Thierry Joliveau} : géographe, directeur de recherche à {\EVS}. Ses recherches portent sur la mise en œuvre des
systèmes d'information géographique et des bases de données géographiques et géohistoriques,
leur utilisation pour la planification territoriale et environnementale.
Il s'intéresse aux implications sociales et culturelles de la « néogographie » et des changements
de représentation géographique induits par les technologies numériques.
Il anime avec Hervé Piégay l'équipe {\sc isig} « Imagerie et systèmes d'information géographique »
qui apporte une expertise géomatique aux étudiants et chercheurs de {\EVS}.

\person{Nathalie Bertrand} : économiste, directrice de recherche au sein de l'unité
Développement des Territoires Montagnards ({\sc dtm}) d'{\IRSTEA}.
Elle coordonne des travaux de recherche sur les dynamiques périurbaines, la régulation institutionnelle et la gouvernance foncière
des espaces agricoles, naturels et périurbains. Elle a notamment piloté le projet {\sc diva3 passages} sur la contribution de l'agriculture
à la production de la trame verte.

\person{Jean-Philippe Tonneau} : géographe, directeur de recherche au {\CIRAD}. Spécialiste de la gouvernance et du développement territorial,
il a participé à la conception de l'Observatoire des agricultures du monde et encadré de nombreux travaux
sur le foncier agricole et l'étalement urbain.

\person{Danny Lo Seen} : chercheur au {\CIRAD} en analyse et modélisation spatiale.
Il étudie à {\TETIS} les dynamiques territoriales à travers la modélisation.
Il est à l'origine avec Pascal Degenne de la plateforme de modélisation Ocelet, développée dans le cadre du projet {\ANR} Descartes.
Ocelet est un outil de simulation cartographique qui permet de tester les « conséquences et cohérences »
de différents scénarios d'affectation de l'usage des sols dans un projet de territoire.

\person{Pierre Maurel} : directeur adjoint scientifique du département Territoires d'{\IRSTEA}, chargé du thème de recherche {\sc synergie}.
Il coordonne depuis 2013 le projet Équipex {\sc geosud}. Ses travaux de recherche portent sur les usages et les effets
des représentations spatiales dans les dispositifs d'information et de communication en appui aux processus de gouvernance territoriale.

\person{Olivier Barreteau} : directeur de l'{\UMR\ G-Eau}.
Spécialiste en modélisation d'accompagnement,
ses recherches portent sur une meilleure gestion des ressources renouvelables
et une approche territoriale de la ressource en eau.

% Liaisons internes à l'{\TETIS}
% Éric Barbe, Danny Lo Seen, Pascal Degenne, Pierre Maurel, Maxime Lenormand, Philippe Lemoisson, Jean-Pierre Chéry, Flavie Cernesson

\subsection{Projets en relation avec le travail de thèse}

% Ce projet s'inscrit dans la continuité des travaux de {\TETIS} sur la cartographie de la tâche artificialisée, de la thèse de Maud Balestrat (2011) et de plusieurs thèses en cours :
% usages et effets des informations sur la consommation des terres agricoles par l'urbanisation par les acteurs de l'aménagement territorial en Languedoc-Roussillon (Anja Martin-Scholtz)
% cohérence multi-échelle de la trame verte et bleue (Julie Chaurand)
% représentations sociales des espaces agricoles péri-urbains en Australie et en France (Laure-Élise Ruoso)
% incertitude dans les données, application à la tâche artificialisée (post-doc GeoSUD)
% Il s'appuiera sur la plateforme de modélisation Ocelet, qui continue d'être activement développée par l'équipe AMoS.

Ce travail de recherche bénéficiera des projets et des collaborations portés par l'Équipex {\sc geosud} et le Labex {\sc imu},
apportera une contribution au projet {\sc frugal}.

L'Équipex {\sc geosud} (2010-2017) réunit 14 partenaires institutionnels et a permis de mettre en place une plateforme
scientifique et technique pour faciliter l'usage des données d'observation de la Terre et répondre aux besoins
de la communauté scientifique et des politiques publiques.
Cet effort se prolonge dans le pôle thématique sur les surfaces continentales Théia
pour produire des données sur étagères et des services à destination des acteurs scientifiques.
% La méthode de cartographie de la tâche artificialisée développée en Languedoc-Roussillon (Dupuy et al,  2012)
% est en cours de généralisation à la France entière à partir des couvertures nationales annuelles GeoSUD SPOT 6-7
% et {\TETIS} poursuit dans ce cadre des projets de recherche sur le suivi de la dynamique de la tâche artificialisée
% par télédétection.

Le Labex {\sc imu} « Intelligence des mondes urbains » a pour objectif de stimuler la recherche scientifique
sur les mondes urbains passés, présents et futurs, et de créer un contexte favorable à l'innovation sociale
en application de ces recherches. Il regroupe 500 chercheurs et 28 laboratoires des universités de Lyon et de Saint-Étienne
autour de nombreuses thématiques de la ville contemporaine :
la nature dans la ville, les mobilités urbaines, la ville numérique, les risques urbains et environnementaux.

Le projet {\sc frugal}  s'intéresse à  l’analyse   des   enjeux   systémiques   liés   à
l’approvisionnement alimentaire de métropoles du Grand Ouest français et en région Rhône-
Alpes. Dans un contexte de crises récurrentes (économique, sanitaire, foncière, etc.), l'objectif du projet
est d'analyser les scénarios possibles qui conduiraient vers une autonomie alimentaire accrue dans une perspective de ville-territoire
post-carbone et de meilleure valorisation économique et sociale des ressources de ces
territoires.