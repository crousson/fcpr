\section{Analyse à travers le cadre heuristique IAD}

\startplacefigure[location=middle,
short={Analyse de l'étalement urbain à travers le cadre IAD},
title={Analyse de l'étalement urbain à travers le cadre IAD.
       Adapté de \citet{ostrom_background_2011} }]
  \externalfigure[../svg/iad_sprawl.svg][width=\textwidth]
\stopplacefigure

% \subject{Variables exogènes}

% \startitemize

% \item Occupation du sol
% \item Système (d'acteurs) territorial
% \item Régles d'urbanisme / Normes environnementales / Règles de gouvernance

% \stopitemize

\subsubject{Système territorial}

\startitemize[packed]

\item réseau d'acteurs
\item relations entre les différents espaces
\item relations entre ville et agriculture

\stopitemize

\subsubject{Acteurs}

Plusieurs échelles d'action et de décision
sont imbriquées les unes dans les autres :

\startcolumns[2]
\startitemize[a,packed]

\item Échelle individuelle :

	\startitemize[packed]
		\item propriétaire du foncier
		\item promoteur immobilier
		\item agriculteur
		\item habitant, citadin, usager
		\item porteur de projet agriculture urbaine / circuit court / \dots
	\stopitemize

\item Échelle locale : commune, \SCoT, inter-\SCoT, agglomération

	\startitemize[packed]

		\item collectivités locales
		\item Agence d'urbanisme
		\item services de l'État
		\item chambres professionnelles
		\item société civile : CoDev, associations, \dots

	\stopitemize

\item échelles méso : département, région

\item échelles globales : national, Europe, Monde

\stopitemize
\stopcolumns

\startcolumns[n=2]
\subsubject{Situations}


\startitemize[packed]

\item Révision des documents d'urbanisme
\item Intervention foncière / exercice du droit de préemption
\item Instruction des permis de construire
\item DUP

\stopitemize

Les CDCEA sont l'une des arènes de l'action,
et permettent à des acteurs qui n'avaient pas
ou plus difficilement la possibilité d'interagir
de se rencontrer pour exprimer leur point de vue
sur les choix d'utilisation et d'affectation du sol.

\column
\subsubject{Résultats}

\startitemize[packed]

\item évolution du zonage dans les documents d'urbanisme (PLU)
\item densification du bâti
\item étalement urbain
\item mesure de protection des sols et du foncier
\item soutien à des usages et à des activités favorables
  au maintien d'espaces agricoles et naturels
\item modification de la fiscalité ?

\stopitemize
\stopcolumns

\subsubject{Critères d'évaluation}

\startcolumns[2]
\startitemize[packed]

\item diagnostic territorial
\item changements d'occupation du sol (à qualifier et à quantifier)
\item évolution des usages (nécessite de mieux appréhender les fonctions
  des différents espaces, en particulier de l'agriculture urbaine)
\item cohérence inter-échelle
\item état des sols, en distinguant en fonction
  des différentes potentialités
\item état des continuités écologiques (TVB)

\stopitemize
\stopcolumns

\setcounter[subsection][4]