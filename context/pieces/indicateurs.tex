\subsection
{Des indicateurs contestés ou peu utiles}

De nombreux travaux se sont concentrés sur
la mesure de l’étalement urbain à partir de l’observation de l’occupation du
sol. Le CGDD recense 82 indicateurs regroupés en 23 familles « pour explorer
les différentes facettes du sujet de l’étalement urbain » (CGDD, 2012).       % TODO missing ref

\citet{briquel_indicateurs_2013} relèvent l’importance des mécanismes fonciers et ont
proposé par exemple l’indice de perturbation du marché des terres agricoles
comme indicateur de la pression sur le foncier agricole.

Face à cette profusion d’indicateurs, la construction de systèmes
d’indicateurs cohérents et adaptés aux questions posées demande encore à être
approfondie. \cite[balestrat_reconnaissance_2011] a proposé de coupler la mesure
de la tâche artificialisée à partir de données satellitaires haute résolution
à un indicateur de qualité agronomique des sols pour prendre
en compte les potentialités.

Les indicateurs sont sujets à interprétation et leur utilisation pose elle-même problème ;
ils doivent avant tout « nourrir un débat sur les enjeux du territoire, sur les
questions et les difficultés qu’ils soulèvent et la façon d’y répondre »
\cite[briquel_indicateurs_2013].

Enfin, l’évaluation de la qualité des données spatiales et de leur adéquation
aux besoins des utilisateurs (« fitness for use ») reste une question en soi
et reçoit de plus en plus d’attention de la communauté scientifique
\cite[devillers_thirty_2010].


\subsection
{Outils de modélisation}

Deux familles de modèles explicitement spatialisés
sont principalement utilisées pour modéliser et simuler
les changements d'occupation et d'utilisation du sol :

\startitemize

\item les automates cellulaires
\item les systèmes multi-agent

\stopitemize

Les modèles basés sur les automates cellulaires, comme le modèle CLUE \cite[verburg_modeling_2002],
permettent de rendre compte de manière tendancielle des processus d’évolution sans
toutefois les expliciter. Ils peuvent être construits et calibrés à l’aide des méthodes d’intelligence
artificielle ou de fouille de données.

Les systèmes multi-agents (SMA) cherchent à mettre en évidence
les propriétés du système territorial à partir des comportements individuels des acteurs.

Entre les deux, la plateforme Ocelet a été développée pour simuler les dynamiques territoriales et paysagères
en rendant compte explicitement des processus et des interactions.
Cette plateforme se fonde sur le formalisme des graphes pour représenter
les relations entre les acteurs et les objets du système territorial \cite[degenne_approche_2012].
Cette représentation permet de modéliser des processus multi-échelles
et combine des avantages des automates cellulaires avec des avantages
des systèmes multi-agents.

\subsection
{Des modèles comme support à la prospective : la modélisation d’accompagnement}

Le modèle support de l’analyse et du questionnement scientifique peut aussi
devenir le support à la concertation et aux démarches participatives.

C’est ce que propose la modélisation d’accompagnement pour aborder les
problèmes complexes de gestion environnementale et de gestion collective des
ressources renouvelables \cite[etienne_modelisation_2010, etienne_modelisation_2012].
Elle exploite les modèles comme des
objets intermédiaires qui servent de support au dialogue entre les acteurs et
à l’apprentissage collectif et organisationnel ; elle permet ainsi une
construction itérative et adaptative des décisions.

Les outils numériques, et en particulier les outils collaboratifs, pourraient
être davantage mobilisés pour construire une connaissance partagée du
territoire. Ils permettraient de dépasser les limites actuelles des démarches
d’observatoire et d’impliquer les acteurs dans la réponse aux enjeux. Ils
aideraient les acteurs historiques et nouveaux à se positionner et à
interagir. Cependant, l’extension de l’utilisation d’outils mathématiques et
informatiques complexes à des arènes participatives intégrant des élus et des
citoyens demande prudence, car l’usage des techniques de simulation et la
diffusion des résultats peuvent s’avérer délicats \cite[dodane_simuler_2014].