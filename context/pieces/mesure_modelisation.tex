\section
{Mesure de la consommation d’espaces agricoles, naturels et forestiers}


\subsection
{Données sur l’occupation des sols}

Des sources de données de plus en plus nombreuses sont aujourd’hui disponibles
pour produire des indicateurs. Malheureusement, ces différentes sources de
données donnent une mesure divergente du phénomène de consommation d’espaces
agricoles \cite[_panorama_2014]. À l’échelle infrarégionale, les données spatiales
disponibles dans les bases de données existantes ne donnent pas de réponse
fiable pour un pas de temps décennal \cite[bousquet_les_2013].

Parmi les bases de données disponibles ou bientôt disponibles à l’échelle
nationale, citons :

\startitemize

\item Corine Land Cover

\item l’enquête TERUTI-LUCAS

\item le registre parcellaire graphique (RPG)

\item les couvertures satellitaires haute résolution annuelles GéoSUD

\item le projet de cartographie de l’occupation des sols à grande échelle de l’IGN
(OCS GE).

\stopitemize

Le rapprochement de données sur l’occupation du sol d’échelles différentes
dans une base de données nationale pose encore de nombreux problèmes d’ordre
méthodologique et pratique.

L’apport des outils collaboratifs à une connaissance plus précise de
l’occupation et de l’utilisation du sol n’a pas encore été exploré. Ces outils
permettraient d’enrichir et de valider la connaissance existante dans une
approche de « community sourced geographic information » ou de « volunteered
geographic information », mais aussi de développer des modes de production de
l’information plus innovants qui tirent parti de l’implication grandissante
des utilisateurs tandis que les chaînes de production traditionnelles
affichent des limites en terme de coût et de capacité à traiter une grande
quantité de données hétérogènes.

\section
{Méthodes : analyse spatiale, modélisation et prospective}

\subsection
{De la prospective territoriale à la géoprospective}

Les indicateurs d’occupation du sol et de suivi de l’étalement urbain sont
utiles dans la mesure où ils nourrissent le « débat sur les enjeux du
territoire, sur les questions et les difficultés qu’ils soulèvent et la façon
d’y répondre » \cite[briquel_indicateurs_2013].
Ils devraient donc avant tout être mobilisés
dans une démarche prospective, bien qu’ils puissent aussi, une fois les enjeux
correctement identifiés, servir à évaluer l’efficacité des actions et des
politiques publiques.

La prospective a émergé en France à la fin des années 1950 et consiste à
porter un regard sur l’avenir destiné à éclairer l’action présente à partir de
l’étude des facteurs historiques (la rétrospective), des forces en jeu et de
leurs inter-relations (De Jouvenel, 1999). L’objectif n’est pas de prédire
l’avenir mais d’explorer des scénarios possibles pour identifier les risques
et les opportunités, prendre des décisions adaptées qui préparent un avenir
désiré. La prospective territoriale, application de la prospective au devenir
d’un territoire, est devenue un outil communément utilisé par les
scientifiques et les professionnels de l’aménagement.

L’analyse territoriale et la cartographie à dire d’acteurs ont largement été
mobilisées depuis une vingtaine d’années dans les démarches de territoire et
la planification, jusqu’à inventer un nouveau champ de compétences qu’on
désigne aujourd’hui par ingénierie territoriale (Lardon, Piveteau). Cependant,   %% TODO Missing Ref
sans le support d’une objectivation des enjeux et des processus, il est
souvent difficile de prendre en compte les effets d’échelle et de dépasser la
simple confrontation de points de vue (Maurel, 2012).                            %% TODO Missing Ref

La géoprospective est née plus récemment du souhait d’expliciter davantage la
dimension spatiale par rapport à la simple prospective territoriale. La
géoprospective a pour objectif spécifique d’intégrer de la dimension spatiale
aux différents stades du processus prospectif, de chercher à comprendre et
prendre en compte des dynamiques et interactions spatiales dans les scénarios
prospectifs (Gourmelon, 2012). Pour ce faire, la géoprospective introduit des   %% TODO Missing Ref
méthodes de modélisation et de simulation spatiale dans la prospective
territoriale.

\citet{dodane_simuler_2014} ont observé que la géoprospective pouvait être « utile
pour faciliter les échanges entre les experts de différents secteurs sur un
même territoire ou les experts de différents territoires contigus ou sécant ».
Cette approche présente un intérêt pour expérimenter des idées et tester des
hypothèses sur les interactions spatiales et les trajectoires territoriales
afin « d’aider les élus à prendre conscience de l’impact des politiques
d’aménagement sur lesquelles ils réfléchissent ». Toutefois, la complexité
actuelle des outils de simulation et la difficulté à les manipuler pour des
non-experts limitent encore le public auquel la géoprospective peut
s’adresser.

\subsection
{La modélisation comme processus de recherche}

Les géographes se sont depuis longtemps attachés à décrire le vocabulaire et
la syntaxe des organisations spatiales et ont proposé des représentations des
dynamiques territoriales sous la forme de chorèmes (Brunet, 1980). Sous         %% TODO Missing Ref
l’angle de la théorie des systèmes, ils ont proposé des modèles qui expliquent
la production de l’espace par des éléments (ressources, moyens de productions,
populations, capital, information) qui interagissent et qui, en interagissant,
font émerger des structures qui donnent formes aux territoires et aux paysages
\cite[brunet_dechiffrement_2001].

La cartographie à dire d’acteurs permet de construire une représentation
partagée du territoire à partir des perceptions qu’ont les acteurs des
hétérogénéités fonctionnelles et des potentialités (Benoît, 2006). Elle
s’appuie largement sur les chorèmes et c’est un exemple de modélisation
iconique (ou graphique) dont le but est d’expliquer et de représenter la
structure du territoire, ses enjeux et ses dynamiques.

Modéliser, du point de vue de la pratique géographique, c'est « rechercher
quelle composition de modèles rend le mieux compte d'une organisation
régionale ou locale, d'une configuration de champ ou de réseau, ou d'une
distribution spatiale » \cite[brunet_modeles_2000]. La construction de modèles permet de
formuler des hypothèses pour construire une information qui apporte un
éclairage original et utile \cite[mathian_objets_2014].

La géographie quantitative fournit des outils de mesure basés sur l’analyse
spatiale en combinant systèmes d’information géographique et outils
statistiques \cite[pumain_les_2001,bavaud_handbook_2009,sanders_models_2010].
Les méthodes de
l’économétrie spatiale \cite[lesage_introduction_2009,vignes_fiches_2013] permettent d’exploiter
des données d’échelles différentes et d’analyser les processus spatiaux et les
flux mais aussi de répondre à des questions d’optimisation liée à la
localisation ou encore de quantifier des externalités.

Les modèles prédictifs permettent de simuler des phénomènes territoriaux. Les
modèles basés sur les automates cellulaires, dont un exemple pour la
simulation de l’évolution de l’occupation du sol est le modèle CLUE \cite[verburg_modeling_2002],
permettent de rendre compte de manière tendancielle des processus
d’évolution sans toutefois les expliciter. Ils peuvent être construits à
l’aide des méthodes d’intelligence artificielle ou de fouille de données.

D’autres formalismes ont pour but de mettre en évidence les propriétés du
système territorial à partir des comportements individuels des acteurs
(systèmes multi-agents) ou de rendre compte explicitement des processus et des
interactions. Pour ce dernier besoin, l’UMR TETIS a développé une plateforme
de simulation des dynamiques territoriales qui se fonde sur le formalisme des
graphes et propose un langage spécifique de modélisation destiné pour
représenter et simuler des dynamiques territoriales explicitement
spatialisées. Ce formalisme rend ainsi possible d’expliciter les processus et
les relations entre les acteurs et les objets du système territorial
\cite[degenne_approche_2012,castets_integration_2014].

