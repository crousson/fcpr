\subsection{
La consommation d'espaces agricoles et naturels non raisonnée n’est pas
durable}

Le corollaire de l'étalement urbain est la consommation d'espaces agricoles,
naturels et forestiers, qui est la conséquence en négatif du même phénomène de
différenciation et d'organisation de l'espace au cours du processus
d’urbanisation. Cette diversité de points de vue est importante pour rendre
compte de la variété des enjeux.

L'étalement urbain est indissociable de la consommation d'espaces : pour que
la ville s'étale, il faut qu'il y ait consommation d’espaces agricoles ou
naturels, et cette consommation est d'autant plus facile qu'il y a des espaces
disponibles.

L’espace agricole est encore perçu par une majorité d’acteurs comme un tiers
espace, associé à de faibles enjeux et disponible pour des usages à plus forte
valeur ajoutée. L’étalement urbain pose donc la question de la place allouée à
l’agriculture dans les projets de territoire.

Le choix du terme « consommation », qui s'est imposé dans le discours public
et a été repris dans les dernières lois agricoles, est significatif. Il
désigne la transformation et par conséquent la disparition d'une ressource, le
sol, qui n'est pas renouvelable, et le risque d'épuisement de cette ressource.

Le mot consommation pourrait également référer à une critique post-
productiviste de la société de consommation, perçue comme essentiellement non
durable. L’artificialisation des terres contrevient en effet aux principes du
développement durable, qui préconisent d’éviter les irréversibilités, de
découpler le développement des ressources naturelles et des facteurs primaires
de production, et de payer les vrais coûts \cite[sainteny_letalement_2008].

\subsection{
	L’étalement urbain n’est que l’une des causes de l’artificialisation du sol
}

L'étalement urbain consomme des espaces agricoles et naturels en les
artificialisant. Pour autant, l'artificialisation des sols ne se limite pas à
l'étalement urbain.

Est artificiel ce qui est produit par une technique humaine et non par la
nature, mais également ce qui se substitue à un élément naturel. Ainsi,
l’étalement urbain dénature les espaces agricoles et naturels en leur
substituant de manière peut-être irréversible une nouvelle occupation du sol
beaucoup plus fortement anthropisée.

Est artificiel aussi ce qui manque d’authenticité. On ne peut pas dans le
débat sur l'artificialisation des terres et l'étalement urbain ignorer
complètement les perceptions et les représentations des acteurs, qui sont
éventuellement contradictoires, conflictuelles et qui évoluent.

L'artificialisation a des conséquences concrètes sur le fonctionnement
hydrologique et écologique d’un territoire et conduit à des pertes
irréversibles de potentialités. C'est le cas notamment quand
l'artificialisation a pour conséquences l'imperméabilisation des surfaces,
mais aussi lorsque les pratiques agricoles viennent perturber le
fonctionnement biologique des sols ou les équilibres naturels.

L'artificialisation n'est pas le propre de l'espace urbain, et au delà de la
dualité urbain/rural, ville/campagne, artificiel/naturel, elle n'est pas qu'un
problème quantitatif qui se mesurerait en hectares consommés mais c'est
également un problème qualitatif qui demande d'évaluer le degré
d'artificialisation des milieux qui, du moins en France et en Europe, sont
tous soumis à une influence anthropique plus ou ou moins forte.