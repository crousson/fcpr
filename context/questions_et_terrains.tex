\section{Sujet de thèse}

\subsection{Questions de recherche}

Compte-tenu des contradictions des politiques d'aménagement
en faveur de la multifonctionnalité et du maintien
d'usages agricoles à proximité des villes dont nous faisons l'hypothèse,
nous proposons d'examiner les questions suivantes :

\startitemize[n]

% \item 	Dans l'hypothèse où les acteurs peuvent de manière incrémentale / adaptative
% 	% volontairement ?
% 	modifier les règles du jeu en fonction des résultats obtenus,
% 	de quelles informations ont-ils besoin
% 	pour soutenir ce processus adaptatif ?
% 	% -> (R -> D,P; S,I -> D,P)

% \item 	Comment les acteurs peuvent-ils volontairement modifier les règles du jeu,
% 	de manière incrémentale / adaptative, en fonction des résultats obtenus ?
	
% 	De quelles informations ont-ils besoin
% 	pour soutenir ce processus adaptatif ?
	% -> (R -> D,P; S,I -> D,P)

\item	Comment les relations existantes entre ville et agriculture
	influencent-elles les interactions entre acteurs ?
	% -> (S,I->R; R->S,I)
	% -> à modéliser avec Ocelet ?

\item	Comment le résultat des interactions
	entre acteurs modifie-t-il la dynamique territoriale
	d'étalement urbain ?
	% -> (R -> S)
	
	Y a-t-il en particulier des effets de couplage (par ex. la protection d'espaces à un endroit
	entraîne un étalement plus important à un autre endroit du système territorial) ?

\item	Y a-t-il des relations spatiales et fonctionnelles entre les usages agricoles
	et la ville, concourant à des formes urbaines différentes,
	plus favorables que d'autres à la maîtrise de l'artificialisation
	et de l'étalement urbain ?

\item	Comment ces configurations spatiales influencent-elles les fonctions
	assignées à l'agriculture et son adaptation aux nouvelles demandes sociétales ?
	

\stopitemize

\subsection{Objectifs}

Le premier objectif de la thèse sera d'établir
un modèle conceptuel des interactions entre ville et agriculture.
Il s'agira d'identifier les processus territoriaux
qui interviennent dans la consommation d'espaces agricoles et naturels
et d'analyser les composantes de ces processus pour les expliciter.

Le deuxième objectif consistera à mettre en évidence
les influences réciproques de la ville et des usages agricoles
en confrontant ce modèle conceptuel aux changements d'occupation
du sol en zone périurbaine observés sur plusieurs cas d'études.
La modélisation spatialement explicite des interactions ville-agriculture
à l'échelle de décision des acteurs individuels
et à l'échelle du territoire servira à formaliser les relations causales
et à mettre en relation les deux échelles.

Enfin, le troisième objectif sera de déduire de la modélisation
des dynamiques territoriales un modèle
d'observation afin de proposer un système d'indicateurs
pour le suivi de la consommation d'espaces naturels et agricoles
en zone périurbaine.


\subsection{Choix de l'échelle de travail}

Le choix de l'échelle d'analyse et de modélisation est guidé
par les considérations suivantes :

\startitemize

\item cette échelle doit être proche de l'échelle de décision
  pour permettre de représenter les processus sur lesquels il est possible
  d'agir ;

\item elle doit permettre une mise en évidence des incohérences
  qui résultent de décisions prises
  à des échelles inférieures et supérieures.

\stopitemize

Nous proposons de travailler à l'échelle du territoire de projet,
qui peut être le \SCoT, l'inter-\SCoT\ ou le \PLUi.

À l'échelle du \SCoT, dans le cadre d'un territoire sous influence d'un pôle
urbain important, les interactions peuvent être modélisées
à l'intérieur du système plus large de l'agglomération
et d'éventuelles contradictions entre le \SCoT\ et les \PLU\ 
peuvent être mises en évidence.

L'échelle inter-\SCoT\ est d'une autre manière intéressante
pour examiner de possibles incohérences à l'échelle supérieure,
notamment par rapport à la fixation d'objectifs de limitation
de l'étalement urbain et l'examen de l'efficacité
d'une politique normative \cite[dodane_simuler_2014].