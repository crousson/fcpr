\section{Sujet de thèse}

\subsection{Questions de recherche et objectifs}

Compte-tenu des contradictions des politiques d'aménagement
en faveur de la multifonctionnalité et du maintien
d'usages agricoles à proximité des villes dont nous faisons l'hypothèse,
quels sont les impacts possibles de ces politiques
sur la consommation d'espaces naturels et agricoles ?

\startitemize[n]

\item	Y a-t-il des relations spatiales et fonctionnelles entre les usages agricoles
	et la ville, concourant à des formes urbaines différentes,
	plus favorables que d'autres à la maîtrise de l'artificialisation
	et de l'étalement urbain ?

\item	Comment ces configurations spatiales influencent-elles les fonctions
	assignées à l'agriculture et son adaptation aux nouvelles demandes sociétales ?

\item	Quels sont les couplages entre le maintien d'usages agricoles
	à proximité des villes ou en relation avec l'espace urbain
	et d'autres processus de l'étalement urbain et de la consommation
	d'espaces agricoles et naturels ?

\stopitemize

Le premier objectif de la thèse sera d'établir
un modèle conceptuel des interactions entre ville et agriculture.
Il s'agira d'identifier les processus territoriaux
qui interviennent dans la consommation d'espaces agricoles et naturels
et d'analyser les composantes de ces processus pour les expliciter.

Le deuxième objectif consistera à mettre en évidence
les influences réciproques de la ville et des usages agricoles
en confrontant ce modèle conceptuel aux changements d'occupation
du sol en zone périurbaine observés sur plusieurs cas d'études.
La modélisation spatialement explicite des interactions ville-agriculture
à l'échelle de décision des acteurs individuels
et à l'échelle du territoire servira à formaliser les relations causales
et à mettre en relation les deux échelles.

Enfin, le troisième objectif sera de déduire de la modélisation
des dynamiques territoriales un modèle
d'observation afin de proposer un système d'indicateurs
pour le suivi de la consommation d'espaces naturels et agricoles
en zone périurbaine.


\subsection{Choix de l'échelle de travail}

Le choix de l'échelle d'analyse et de modélisation est guidé
par les considérations suivantes :

\startitemize

\item cette échelle doit être proche de l'échelle de décision
  pour permettre de représenter les processus sur lesquels il est possible
  d'agir ;

\item elle doit permettre une mise en évidence des incohérences
  qui résultent de décisions prises
  à des échelles inférieures et supérieures.

\stopitemize

Nous proposons de travailler à l'échelle du territoire de projet,
qui peut être le SCoT, l'inter-SCoT ou le PLUi.

À l'échelle du SCoT, dans le cadre d'un territoire sous influence d'un pôle
urbain important, les interactions peuvent être modélisées
à l'intérieur du système plus large de l'agglomération
et d'éventuelles contradictions entre le SCoT et les PLU
peuvent être mises en évidence.

L'échelle inter-SCoT est d'une autre manière intéressante
pour examiner de possibles incohérences à l'échelle supérieure,
notamment par rapport à la fixation d'objectifs de limitation
de l'étalement urbain et l'examen de l'efficacité
d'une politique normative \cite[dodane_simuler_2014].


\subsection{Terrains envisagés}

Deux terrains sont envisagés afin de i) vérifier la reproductibilité de la démarche,
en particulier par rapport aux données disponibles,
ii) permettre des comparaisons entre des situations différentes et identifier
d'éventuelles régularités et iii) tester l'opérationnalité des modèles proposés.

\subsubsection{Terrain initial : Bassin de Thau}

Le bassin de Thau est un territoire de 353 km^2 autour de l’étang éponyme,
proche de Montpellier, bien connu de l’UMR TETIS,
pour lequel on dispose déjà de nombreuses données.

Il a connu une très forte urbanisation
au cours des 50 dernières années et regroupe 14 communes sur deux intercommunalités.
Le paysage agricole est composé pour partie de vignobles,
d’espaces agricoles « d’intérêt écologique » et d’espaces agricoles
productifs « à dynamiser » (céréales, monoculture de melons),
avec une forte diversification agricole déjà
engagée sur la commune de Villeveyrac.

Ayant adopté un SCoT début 2014, le syndicat mixte du bassin de Thau (SMBT) s’est donné à travers un
plan de gestion intégrée l’objectif de mieux maîtriser l’urbanisation
tout en « préservant les espaces agricoles et les paysages, menacés par une urbanisation
trop rapide pour conserver l’identité du territoire et le cadre de vie »
(SMBT, 2010 ; SMBT, 2014). La participation de la population, et la mise en place de nombreux
partenariats sont l’une des caractéristiques de fonctionnement revendiquée
tant par le syndicat que par ses partenaires \cite[maurel_apprentissage_2008].

Pour accompagner ce programme d'actions et évaluer en permanence
ses effets sur les dynamiques territoriales, le SMBT travaille avec l'appui de l'UMR TETIS
à la définition et à la mise en place d'un observatoire territorial basé
sur un ensemble d'indicateurs spatiaux.


\subsubsection{Terrain secondaire}

Un deuxième terrain sera choisi rapidement en Rhône-Alpes pour reproduire
et valider la démarche, en utilisant les critères de sélection suivants :

\startitemize[packed]

\item dynamique urbaine différente ;
\item typologie différente d’agriculture ou de paysage ;
\item enjeux particuliers identifiés dans les documents de planification ;
\item implication des acteurs du monde agricole, de la société civile
  et des collectivités territoriales.

\stopitemize

Il sera possible d'utiliser un terrain déjà exploité par l'UMR EVS (bassin versant de l'Yzeron,
St Étienne Métropole) ou de choisir un terrain dans le cadre du projet FRUGAL.