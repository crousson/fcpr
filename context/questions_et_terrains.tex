\section{Sujet de thèse}

\subsection{Objets de recherche}

Dans le contexte d'une volonté affichée de mieux intégrer les fonctions agricoles
dans le projet urbain,
et de limiter la consommation d'espaces agricoles et naturels,
nous proposons d'étudier les objets suivants et leurs relations :

\startitemize[packed]

\item l'étalement urbain et son pendant, la consommation d'espaces agricoles
\item l'agriculture urbaine et les fonctions qu'elle remplit
\item la localisation des usages agricoles
  et les interactions spatiales entre ville et agriculture
\item les modèles d'action, c'est-à-dire l'ensemble
  des projets et des actions que souhaitent (peuvent ?) mettre en œuvre
  les acteurs pour modifier les dynamiques territoriales.

\stopitemize

\subsection{Questions}

% Compte-tenu des contradictions des politiques d'aménagement
% en faveur de la multifonctionnalité et du maintien
% d'usages agricoles à proximité des villes dont nous faisons l'hypothèse,
% nous proposons d'examiner les questions suivantes :

Si la proximité urbaine
et l'étalement urbain ont un effet sur les usages agricoles
périurbains, la localisation et l'adaptation de ces usages
aux nouvelles demandes de proximité
ont également un effet sur l'étalement urbain et les formes urbaines.
Dans les deux sens de cette relation, les effets ne sont pas univoques ;
ils sont tantôt positifs ou négatifs par rapport à l'objectif de bonne gouvernance
du foncier et d'un développement urbain durable.

% \startitemize[n]

% \item 	Dans l'hypothèse où les acteurs peuvent de manière incrémentale / adaptative
% 	% volontairement ?
% 	modifier les règles du jeu en fonction des résultats obtenus,
% 	de quelles informations ont-ils besoin
% 	pour soutenir ce processus adaptatif ?
% 	% -> (R -> D,P; S,I -> D,P)

% \item 	Comment les acteurs peuvent-ils volontairement modifier les règles du jeu,
% 	de manière incrémentale / adaptative, en fonction des résultats obtenus ?
	
% 	De quelles informations ont-ils besoin
% 	pour soutenir ce processus adaptatif ?
	% -> (R -> D,P; S,I -> D,P)

% \item	Comment les relations existantes entre ville et agriculture
% 	influencent-elles les interactions entre acteurs ?
	% -> (S,I->R; R->S,I)
	% -> à modéliser avec Ocelet ?

% \item	Comment le résultat des interactions
% 	entre acteurs modifie-t-il la dynamique territoriale
% 	d'étalement urbain ?
	% -> (R -> S)

Y a-t-il des relations spatiales et fonctionnelles entre les usages agricoles
et la ville, concourant à des formes urbaines différentes,
plus favorables que d'autres à la maîtrise de l'artificialisation
et de l'étalement urbain ?

Comment ces configurations spatiales influencent-elles les fonctions
assignées à l'agriculture et son adaptation aux nouvelles demandes de proximité ?

% Quels sont les effets de couplage entre la préservation d'espaces agricoles
% et naturels et l'étalement urbain ?

% Nous faisons existe l'hypothèse d'un couplage entre plusieurs processus territoriaux
% impliqués dans la consommation d'espaces naturels et agricoles ;
% par exemple, le maintien d'espaces ouverts
% interstitiels et la densité des parcelles bâties sont possiblement couplés.
% Ces couplages, tant qu'ils ne sont pas explicités,
% rendent difficiles la conception d'un modèle d'actions efficace.

Les contradictions spatiales entre le besoin de maîtriser l'étalement urbain
et la demande de préserver des espaces ouverts et agricoles à proximité de la ville
se traduisent par un couplage entre plusieurs processus territoriaux.
Les effets de ces couplages peuvent être spatialement dissociés
et nous faisons l'hypothèse qu'ils sont la cause d'incohérence à l'échelle supérieure.

On peut imaginer aussi que les acteurs de la planification
introduisent volontairement, sous forme de compensation par exemple, un couplage
entre la consommation d'espaces au profit de l'artificialisation
et la protection d'espaces naturels et agricoles. Quel est alors
l'effet sur la dynamique territoriale ? Ce couplage
à travers les règles permet-il de mieux atteindre les objectifs ?

Enfin, nous proposons, comme hypothèse simplificatrice,
de représenter par un effet de polarisation la préférence
i) des acteurs individuels pour une forme urbaine plus ou moins dense
et ii) des acteurs agricoles pour une adaptation plus ou moins forte
à la proximité urbaine. Nous pouvons supposer que
cette double polarisation est influencée
par le contexte et par les interactions des acteurs entre eux.	

% \stopitemize

\subsection{Objectifs}

Le premier objectif de la thèse sera d'établir
un modèle conceptuel des processus territoriaux
qui interviennent dans la consommation d'espaces agricoles périurbains
et d'analyser les composantes de ces processus pour les expliciter.
Ce modèle conceptuel sera établi à partir d'une revue de la littérature
et d'un diagnostic territorial des relations ville-agriculture
sur les cas d'études sélectionnés.

Le deuxième objectif consistera à confronter empiriquement
ce modèle conceptuel aux changements d'occupation
du sol en zone périurbaine observés sur plusieurs cas d'études.
La modélisation spatialement explicite des changements d'utilisation du sol
à l'échelle de décision des acteurs individuels
et à l'échelle du territoire servira à formaliser les relations causales
et à mettre en relation les deux échelles.
À travers la modélisation, nous chercherons à mettre en évidence
les couplages et les effets de polarisation dont nous faisons l'hypothèse
mais aussi l'influence des règles d'urbanisme et des normes environnementales.

Enfin, le troisième objectif sera de déduire de la modélisation
des dynamiques territoriales un modèle
d'observation afin de proposer un système d'indicateurs
pour le suivi de la consommation d'espaces naturels et agricoles
en zone périurbaine.


\subsection{Choix de l'échelle de travail}

Le choix de l'échelle d'analyse et de modélisation est guidé
par les considérations suivantes :

\startitemize

\item cette échelle doit être proche de l'échelle de décision
  pour permettre de représenter les processus sur lesquels il est possible
  d'agir ;

\item elle doit permettre une mise en évidence des incohérences
  qui résultent de décisions prises
  à des échelles inférieures et supérieures.

\stopitemize

Nous proposons de travailler à l'échelle du territoire de projet,
qui peut être le \SCoT, l'inter-\SCoT\ ou le \PLUi.

À l'échelle du \SCoT, dans le cadre d'un territoire sous influence d'un pôle
urbain important, les interactions peuvent être modélisées
à l'intérieur du système plus large de l'agglomération
et d'éventuelles contradictions entre le \SCoT\ et les \PLU\ 
peuvent être mises en évidence.

L'échelle inter-\SCoT\ est d'une autre manière intéressante
pour examiner de possibles incohérences à l'échelle supérieure,
notamment par rapport à la fixation d'objectifs de limitation
de l'étalement urbain et l'examen de l'efficacité
d'une politique normative \cite[dodane_simuler_2014].

\subsection{Terrains envisagés}

[\dots]
%\input terrains