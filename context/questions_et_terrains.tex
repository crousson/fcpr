\section[sujet]{Sujet de thèse}

\subsection{Objets de recherche}

Dans le contexte d'une volonté affichée de mieux intégrer les fonctions agricoles
dans le projet urbain,
et de limiter la consommation d'espaces agricoles et naturels,
nous proposons d'étudier les objets suivants et leurs relations :

\startitemize[packed]

\item l'étalement urbain et son pendant, la consommation d'espaces agricoles et naturels ;
\item l'agriculture urbaine et les fonctions qu'elle remplit ;
\item la localisation des usages agricoles
  et les interactions spatiales entre ville et agriculture ;
\item les modèles d'action, c'est-à-dire l'ensemble
  des projets et des actions que souhaitent (peuvent ?) mettre en œuvre
  les acteurs pour modifier les dynamiques territoriales.

\stopitemize

\subsection{Questions}

Les changements d'occupation et d'usage du sol apparaissent comme des signaux faibles ;
il sont difficiles à observer et à objectiver sur le temps court.
Quel peut être dès lors un dispositif d'observation et de suivi de ces changements
qui permette de nourrir le débat public, de sensibiliser aux enjeux de la protection
des espaces agricoles et naturels et d'aider à construire des scénarios
plus économes et globalement plus cohérents ?

Nous envisageons cette question de portée assez générale
sous deux aspects complémentaires. D'abord, sous l'angle de 
l'évolution des relations entre la ville et son territoire et l'adaptation
des usages agricoles, participant à l'écologisation de la ville et
au renforcement de la place occupée par les espaces naturels et agricoles,
contrecarrant ainsi le processus d'artificialisation.
Ensuite, sous l'angle de possibles couplages
qui interviennent dans la dynamique périurbaine.

Premièrement, si la proximité urbaine
et l'étalement urbain ont un effet sur les usages agricoles
périurbains, on peut supposer que la localisation et l'adaptation de ces usages
aux nouvelles demandes de proximité
ont également un effet sur l'étalement urbain et les formes urbaines.
Peut-on identifier des relations spatiales et fonctionnelles entre les usages agricoles
et la ville qui concourent à des formes urbaines
plus favorables que d'autres à la maîtrise de l'artificialisation
et de l'étalement urbain ?

Deuxièmement, nous pouvons faire l'hypothèse de couplages spatiaux et temporels
non maîtrisés dans le processus de consommation d'espaces agricoles et naturels
qui résultent de décisions insuffisamment coordonnées des acteurs territoriaux,
aboutissant à des contradictions aux échelles de niveau supérieur.
On peut envisager également la possibilité que les acteurs de la planification
introduisent volontairement, sous forme de compensation par exemple, un couplage
entre la consommation d'espaces au profit de l'artificialisation
et la protection d'espaces naturels et agricoles.
Quelle forme peut prendre ce couplage, et quel est alors
l'effet sur la dynamique territoriale ? Ce couplage
à travers les règles permet-il de mieux atteindre les objectifs, et à quelles conditions ?

\subsection{Objectifs}

Le premier objectif de la thèse sera de développer une méthode pour caractériser
l'évolution des espaces agricoles et naturels périurbains et l'adaptation
de ces espaces aux nouvelles demandes de proximité.
Cette méthode doit permettre de faciliter les comparaisons entre cas d'études.
Un modèle conceptuel des processus territoriaux
qui interviennent dans la consommation d'espaces agricoles périurbains
sera établi à partir d'une revue de la littérature
et d'un diagnostic territorial des relations ville-agriculture
sur les cas d'études sélectionnés.

Le deuxième objectif consistera à proposer un modèle d'observation des changements 
d'utilisation du sol qui permette de les relier aux interactions multiples, proches et distantes,
qui en sont la cause et de rendre compte des couplages
dont nous faisons l'hypothèse.
La modélisation des changements d'utilisation du sol
à l'échelle de décision des acteurs individuels
et à l'échelle du territoire servira à mettre en relation les deux échelles,
et à établir des comparaisons entre les cas d'études.

Enfin, le troisième objectif sera de représenter ces interactions
dans des modèles dynamiques explicitement spatialisés
susceptibles de servir de support à la médiation territoriale 
et d'aider à construire des scénarios d'urbanisation
plus économes en espaces agricoles et naturels, et globalement plus cohérents.


\subsection{Choix de l'échelle de travail}

Le choix de l'échelle d'analyse et de modélisation est guidé
par les considérations suivantes :

\startitemize

\item cette échelle doit être proche de l'échelle de décision
  pour permettre de représenter les processus sur lesquels il est possible
  d'agir ;

\item elle doit permettre une mise en évidence des incohérences
  qui résultent de décisions prises
  à des échelles inférieures et supérieures.

\stopitemize

Nous proposons de travailler à l'échelle du territoire de projet,
qui peut être le \SCoT, l'inter-\SCoT\ ou le \PLUi,
pris en considération dans sa relation avec l'aire urbaine
qui l'influence et dans son contexte régional.

À l'échelle du \SCoT, dans le cadre d'un territoire sous influence d'un pôle
urbain important, les interactions peuvent être modélisées
à l'intérieur du système plus large de l'agglomération
et d'éventuelles contradictions entre le \SCoT\ et les \PLU\ 
peuvent être mises en évidence.

L'échelle inter-\SCoT\ est d'une autre manière intéressante
pour examiner de possibles incohérences à l'échelle supérieure,
notamment par rapport à la fixation d'objectifs de limitation
de l'étalement urbain et l'examen de l'efficacité
d'une politique normative \cite[dodane_simuler_2014].

\subsection{Terrains envisagés}

[\dots]
%\input terrains