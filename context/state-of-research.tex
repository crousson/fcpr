\section{
	État de l'art
	Dynamiques spatiales, interactions ville-agriculture
}

Ce paragraphe ne prétend pas faire une revue bibliographique complète
des travaux sur les dynamiques spatiales et
les changements d'utilisation du sol,
mais présente les principaux courants épistémiques
qui s'intéressent à cette problématique et
leurs contributions sur le plan théorique.

L'étude des changements d'utilisation du sol et de leurs dynamiques
constitue un domaine de connaissances à l'intersection entre
sciences de la Terre, sciences sociales et sciences de l'ingénieur.

Les théories les plus anciennes sont géographiques et économiques

von Thünen (1842) influence de la ville sur l'organisation spatiale des productions agricoles
rente de localisation > fertilité Ricardo

études des formes urbaines
La théorie des lieux centraux de Cristaller remonte à 1933
et rend compte de la taille et de l'espacement des villes dans
une organisation hiérarchisée 


