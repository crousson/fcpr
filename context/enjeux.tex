% \section{Enjeux}

\section
{La consommation d'espaces naturels, agricoles et forestiers
  en lien avec l'étalement urbain :
  une problématique de développement durable sujette à controverses}

La consommation d'espaces agricoles et naturels a principalement lieu en zone
périurbaine sous l'effet de l'étalement urbain.

En 2010, 46~\% des exploitations agricoles étaient situées
dans les couronnes périurbaines (selon la définition de l'INSEE).
C'est dans ces zones, qui concentrent de fait de nombreux enjeux,
que se produit l'essentiel de la consommation d'espaces naturels,
agricoles et forestiers.

La problématisation de l'étalement urbain - ou de son corollaire, la consommation
d'espaces agricoles et naturels en zone périurbaine -
montre qu'il ne se réduit pas à un problème quantitatif
\cite[charmes_artificialisation_2013, martin-scholz_quand_2013]
et que cette problématisation fait bien moins consensus
que certains discours globalisants ne le laissent croire.

L'univers controversé \cite[godard_strategies_1993] dans lequel les acteurs locaux doivent
prendre des décisions présente des caractéristiques typiques
des problématiques environnementales :

\startitemize

\item une prédominance de la construction scientifique et sociale du problème
  sur la perception directe des acteurs ;

\item la représentation déséquilibrée ou absente des intérêts de certaines parties
  prenantes (habitants, agriculteurs, consommateurs)
  par rapport à d'autres (urbanistes, propriétaires),
  avec l'intervention de porte-parole contradictoires ;

\item une connaissance scientifique encore controversée par manque
  de données précises utiles à la construction d'une vision partagée des enjeux ;

% \item l'insuffisante compréhension des processus
%   sur lesquels il est possible d'agir ;

\item la difficulté à anticiper les processus territoriaux,
  par manque de connaissance de ces processus, afin d'agir en temps utile
  aux endroits où il est encore possible d'agir compte-tenu des phénomènes
  d'irréversibilité ;

\item l'existence d'incohérence à des échelles différentes,
  par exemple entre la stratégie d'offre foncière des communes
  et l'objectif affiché au niveau SCoT ou PLUi de limitation
  de l'étalement urbain ;

\item la captation du débat public pour défendre des intérêts particuliers
  et « ventriloquer » des préoccupations variées :
  sociales, économiques, écologiques, foncières, urbanistiques ;

% \item l'accumulation des textes qui établissent de nouvelles normes environnementales
%   et traduisent la difficulté à répondre aux enjeux.

\stopitemize

Du point de vue de la consommation d'espaces naturels, agricoles et forestiers,
l'étalement urbain interroge ce que doit être la {\em bonne} gouvernance du foncier,
en prenant en compte les préoccupations du développement durable
qui vont de la demande d'aménités environnementales au « besoin de construire des stratégies alimentaires
où les formes d'agriculture urbaine sont parties prenantes » \cite[mansfield_municipal_2013].
À travers ces préoccupations, c'est la multifonctionnalité
des espaces agricoles et forestiers qui est interrogée.


\section{L'information et la connaissance comme conditions de la {\em bonne} gouvernance}

L'adaptation au changement climatique et la nécessité
de mieux valoriser les ressources territoriales sur le plan économique
et social dans un contexte de crises récurrentes
(économique, sanitaire, foncière, etc.) remettent en perspective
les relations entre ville et agriculture
\cite[scalenghe_anthropogenic_2009, toth_impact_2012, salvati_spatial_2013].

La valeur attribuée à la préservation du foncier et des usages agricoles
n'est pas la même à l'échelle de décision individuelle et à l'échelle
du collectif (institutions, société civile) ;
ces contradictions créent des tensions entre les différents niveaux de décision
et reflètent la compétition qui existe
entre les différents usages du sol
\cite[duvernoy_agriculture_2005, bertrand_quelle_2006, charmes_artificialisation_2013].

L'efficacité du débat sur la consommation d'espaces agricoles, naturels et forestiers
suppose un réinvestissement dans l'évaluation
et nécessite un effort méthodologique pour comprendre les dynamiques de changement
et pour mieux appréhender le rôle central de la temporalité dans les processus de décision,
et ainsi éclairer les exercices de prospective.

L'information et la connaissance occupent par conséquent une place importante
dans la mise en place et le fonctionnement d'une gouvernance innovante,
non seulement comme condition de mise en œuvre de politiques normatives,
mais aussi comme alternatives possibles à ces politiques
\cite[theys_gouvernance_2002].
