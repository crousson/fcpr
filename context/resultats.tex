\page[yes]
\section[resultats]{Résultats attendus}

En sus de la valorisation des résultats de la thèse
par des articles dans des revues scientifiques reconnues,
nous espérons produire et valider au terme du projet de recherche
une méthode et un modèle d'observation pour caractériser l'évolution des espaces agricoles et
naturels périurbains ainsi que l'adaptation de ces espaces aux nouvelles demandes de proximité.
Ce modèle d'observation doit aboutir à la proposition d'un système d'indicateurs
plus pertinent que ceux qui existent déjà
et à une méthode pour coder les informations sur la dynamique périurbaine
et les règles de gouvernance du foncier,
qui permettra d'établir des comparaisons entre différents systèmes territoriaux.

Ce projet doit également contribuer aux méthodes opérationnelles
pour mobiliser plus efficacement les données disponibles
(bases de données topographiques et administratives, données socio-économiques, imagerie
satellitaire) afin de produire des indicateurs utiles, utilisables et utilisés
pour accompagner le débat sur l'étalement urbain.

Nous espérons également mettre au point un dispositif collaboratif
de collecte d'information sur l'occupation et l'usage des sols
qui sera utile pour compléter les informations accessibles par les moyens
d'observation et par les enquêtes de terrain.

Nous incluons enfin dans nos objectifs la production d'éléments méthodologiques
sur la géoprospective transférables aux services déconcentrés, sous forme de fiches {\sc valor},
de ressources de formation ({\FOAD}, ...) et de missions d’expertise et d’appui.
La {\DRAAF} Midi-Pyrénées-Languedoc-Roussillon souhaite par exemple
continuer le travail sur l'artificialisation des sols à potentiel agronomique
initialisé par \citet{balestrat_reconnaissance_2011} et
le reproduire à l'échelle de la nouvelle région.
Nous pourrons dans le cadre de la thèse participer à son accompagnement,
mais aussi à l'accompagnement au niveau national de la mise en place de l'{\OENAF}.