\section
{Inscription du projet dans le parcours professionnel}

\subsection
{Présentation du parcours professionnel}

De formation Agro-GREF avec une spécialisation en sciences forestières,
mon parcours à l'Office national des forêts (ONF)
puis à la Sous-direction des systèmes d'information (SDSI)
m'a amené à me spécialiser
dans les systèmes d'information environnementaux.

\startitemize[n]

\item j'ai réalisé mon stage de fin d'études à Epping Forest
      où j'ai travaillé à la mise en place d'inventaires patrimoniaux
      et à leur organisation dans un système d'information
      ouvert sur Internet ; première expérience de la forêt
      périurbaine dans un contexte favorable à l'expérimentation ;

\item cette expérience périurbaine s'est ensuite prolongée à l'ONF
      à l'agence de Créteil puis à l'agence de Fontainebleau
      où j'ai exercé plusieurs fonctions, dont celle de responsable
      des projets d'aménagements forestiers. J'ai mis en place
      des contrats innovants avec les ETF pour l'exploitation
      en forêt périurbaine et développé une méthode d'estimation
      des coupes vendues sur pied qui s'appuyait sur la cartographie
      des données historiques de commercialisation des bois ;

\item après un passage d'un an au Centre de recherche en géomatique
      de l'université Laval au Québec, j'ai rejoint en 2008 le ministère
      en charge de l'Agriculture à la Sous-direction des systèmes 
      d'information pour prendre la responsabilité
      d'un département d'ingénierie spécialisé dans
      l'information géographique. Mon arrivée à la SDSI
      coïncidait avec le début de la réorganisation territoriale
      des services de l'État et les travaux de convergence
      interministérielle. Deux objectifs principaux ont été
      poursuivis depuis 2008 :
      i) la mise en place d'une infrastructure
      de données géographiques nationales pour répondre aux besoins
      des services déconcentrés et aux exigences
      de la directive INSPIRE,
      ii) le développement d'un cadre technique
      qui favorise la prise en compte de la dimensions spatiale
      dans les projets informatiques nationaux,
      qui a essentiellement bénéficié aux projets du MEDDE.

\stopitemize

La spécificité qui ressort de ce parcours est d'avoir investi autant dans
des compétences de chef de projet que dans une très bonne maîtrise
des outils géomatiques et des méthodes d'ingénierie informatique,
avec une double compétence fonctionnelle
et technique encore perçue comme peu lisible dans une dichotomie
culturelle qui oppose maîtrise d'ouvrage et maîtrise d’œuvre.

Ce parcours m'a permis de bien connaître les enjeux et les attentes
des services déconcentrés, de nouer des contacts avec les principaux acteurs nationaux de l'information géographique.


\subsection
{Motivations du projet de thèse}

Ce projet de formation complémentaire par la recherche 
ce projet participe de la logique d'un parcours d'expert
des systèmes d'information environnementaux
et constitue un moyen de concrétiser et de faire reconnaître
l'expertise acquise.

Il s'inscrit dans le
prolongement de mon expérience professionnelle à l'ONF d'abord,
à la SDSI ensuite.

Cette expérience m'incite à aller chercher des outils intellectuels
complémentaires pour enrichir la contribution que je peux apporter
dans le domaine de l'information spatialisée et des systèmes
d'information environnementaux,
dans le cadre de l'appui aux politiques publiques et dans la perspective
d'un renouvellement institutionnel qui refonde la gouvernance
des problématiques environnementales et accorde une importance
grandissante au partage de l'information entre de nombreux acteurs.

Il est aussi motivé par un contexte (Équipex GéoSUD, pôle Théia, programme
européen Copernicus, programme Sentinel de l’ESA, etc.) qui est très favorable
au développement des usages des données d’observation de la Terre en appui aux
politiques publiques agricoles et environnementales
\cite[centre_detudes_et_de_prospective_recours_2014].

L'objectif de ce projet est donc aussi de me positionner par rapport à ce
contexte qui justifie d’accompagner davantage utilisateurs et décideurs
pour les aider à développer à bon escient leurs usages
de l'information géographique et à tirer parti des données disponibles
et des nouvelles modalités d'organisation et d'action
permises par les technologies numériques.


\subsection
{Le paradoxe de la sous-utilisation des données}

Le paradoxe est que nous disposons aujourd'hui de beaucoup de données dont
nous ne savons souvent pas quoi faire, alors qu'en même temps, nous ne
réussissons pas à obtenir l'information utile qui nous permettrait d'agir
efficacement dans un monde complexe. L'efficacité doit être
ici comprise comme la capacité à adapter l'action à ses objectifs, à apprendre
des erreurs et trouver des modes d'action organisée plus résilients.

Des investissements importants ont été réalisés pour acquérir
et produire des données d'observation de la Terre.
Cela se traduit par l'accroissement régulier du
nombre de capteurs en opération et une augmentation quasi exponentielle
de la quantité de données disponibles.

Néanmoins, cet accroissement de l'offre n'induit proportionnellement qu'un
faible accroissement des usages, ce qui traduit les limites d'un progrès
technologique insuffisamment adossé à une réflexion sur les usages et les
conditions du transfert opérationnel.

L'intérêt pour les données, en particulier dans les applications
environnementales et de connaissance du territoire, existe cependant chez les
acteurs publics sans qu'ils soient pour autant prêts à payer pour utiliser des
données qu'ils ont en pratique beaucoup de mal à exploiter. D'où un décalage
important entre le potentiel de ces données et leur valeur réelle d’usage.

Les plateformes, comme c'est le cas dans la démarche GéoSUD/Théia, poursuivent
l'objectif de :

\startitemize

\item lever le verrou de l'accès aux données ;

\item créer des « écosystèmes » données-utilisateurs-infrastructures à l'intérieur
desquels l'expérimentation et l'innovation sont facilitées et permettent de
développer les usages.

\stopitemize

Cette attention apportée aux usages est devenue nécessaire pour identifier de
nouveaux gisements de valeur et améliorer la production de connaissance
dans la chaîne aval du traitement de l'information, qui reste
insuffisamment développée et ne prend pas assez en compte les utilisateurs
finaux.


\subsection
{Des progrès dans l'accès et le partage de l'information géographique pour les
services déconcentrés}


En 2006, les principaux freins au développement
des usages de l'information géographique à l'ONF étaient :

\startitemize

\item la difficulté à acquérir les données nécessaires,
      et en particulier le coût de ces acquisitions

\item un retard important dans l'informatisation et l'outillage
      des processus métiers qui empêchait de faire
      des données spatialisées le support transversal
      du système d'information métier ;

\item un cloisonnement organisationnel défavorable
      à la circulation de l'information en rendant difficiles
      les échanges internes et interinstitutionnels de
      données, et plus généralement défavorable à l'innovation, à la transformation des processus.

\stopitemize

En 2015, presque dix ans plus tard, ce constat ne serait plus exactement le
même ; la problématique d'accès aux données a favorablement évolué :
i) les données de référence sont beaucoup plus facilement disponibles et
accessibles, et ii) l'idée que les données publiques sont un bien commun a
progressé avec la culture de l'open data, ce qui bénéficie d'abord à
l'administration elle-même. Mais les aspects méthodologiques et
organisationnels restent sinon à peu près les mêmes.

Entre 2007 et 2015, on peut néanmoins noter que des progrès significatifs ont
été enregistrés :

\startitemize

\item les infrastructures de données géographiques ont acquis en maturité et
réussissent à entretenir des partenariats importants entre acteurs publics, en
particulier au niveau régional, bien qu'elles restent principalement portées
par les services de l'État ;

\item d'autres plateformes se mettent en place et permettent d'accéder dans de
meilleures conditions à un bouquet de plus en plus large de données
spatialisées et d'observation de la Terre.

\stopitemize

À l'échelle ministérielle et interministérielle, ces progrès se sont traduits
par des avancées principalement techniques, mais aussi par des évolutions
institutionnelles, notamment en ce qui concerne l'accès et le partage des
données :

\startitemize

\item les deux ministères techniques en charge du territoire ont mis en place,
malgré les difficultés propres à l’interministérialité, une infrastructure de
données géographiques nationale commune (Géo-IDE) pour répondre, quoique
encore imparfaitement, aux exigences de la directive INSPIRE et de l'open data
;

\item à la SDSI, nous avons aussi beaucoup investi dans la maîtrise des risques et
des coûts liés aux projets informatiques, pour prendre en compte l'information
spatiale dès la conception des projets, avec un travail important sur les cas
d'usage, le recensement des patrons applicatifs, l'ergonomie applicative et
l'urbanisation du système d'information ;

\item la gratuité des données du référentiel géographique à grande échelle (RGE)
pour un usage dans le cadre d'une mission de service public sécurise l'accès à
des données de base et garantit l'utilisation de référentiels communs entre
acteurs du service public ;

\item le GéoPortail s'est transformé pour faciliter l'accès aux données de l'IGN et
leur utilisation dans les systèmes d'information métier, en même temps que le
ministère réussit mieux à collaborer avec l'IGN, ce qui lui permet de
bénéficier de son savoir-faire.

\stopitemize

Malgré ces progrès, et dans un contexte où la problématique d'acquisition et
d'accès aux données a significativement évolué, il reste des difficultés
d'ordre organisationnel autant que technique.


% \subsection
% {Une stratégie d'information géographique difficile à définir}

% La faiblesse des ambitions du SDNSIv4 en matière d'information géographique
% entérine l'idée que l'information géographique est une problématique
% métier, donc de la responsabilité des MOA, avec pour corollaire
% un désengagement de la SDSI qui se concentre sur les moyens informatiques,
% la gestion de projet et les applications de gestion.

% Le rapport à l'information géographique des services du ministère
% pourrait se caractériser par :

% \startitemize

% \item des usages de l'information géographique
%       qui reste principalement orienté vers la cartographie thématique ;

% \item une ambition tranversale des SIG qui questionne
%       l'existence d'une fonction SI au niveau local,
%       dans les services déconcentrés, alors que celle-ci est découragée
%       au profit d'une rencentralisation des systèmes d'information ;

% \item une mutualisation au niveau central et interministériel
%       qui se justifie par la recherche d'économie d'échelle,
%       mais qui donne des résultats décevants
%       et apparaît en pratique difficile à gouverner ;

% \item la difficile émergence du collaboratif,
%       qui apparaît pourtant de plus en plus déterminant
%       pour collecter et enrichir l'information spatiale
%       existante, mais aussi pour exploiter les données ;

% \item l'appropriation très difficile de la dimension spatiale
%       par les maitrises d'ouvrages des SI nationaux ;

% \item un niveau élevé d'attentisme.

% \stopitemize


\subsection
{Des difficultés persistantes qui demandent à être levés par des approches
innovantes}

On peut analyser les difficultés qui subsistent
autour de trois enjeux principaux :

\startitemize

\item l'enjeu de la transversalité, qui est de mieux faire circuler l'information en
se libérant des contraintes des « silos » existants ;

\item l'enjeu de l'accompagnement méthodologique, qui est d'homogénéiser le niveau
des pratiques et d'assurer un transfert suffisant vers l'opérationnel ;

\item l'enjeu de valorisation des données, qui est de créer davantage de valeur en
répondant à des usages pertinents adaptés aux besoins des utilisateurs.

\stopitemize

Les constats que je peux tirer de mon expérience à l'ONF et au ministère
chargé de l'agriculture par rapport à ces trois enjeux sont détaillés en
annexe.

En ce qui concerne les deux premiers enjeux, les plateformes qui se
développent à différents niveaux ont considérablement amélioré les conditions
d'accès aux données, tout en créant des écosystèmes de réutilisation
favorables à l'accompagnement et aux développements méthodologiques. Même s'il
reste des difficultés de mise en œuvre qui sont principalement
organisationnelles, la voie est tracée.

L'enjeu de valorisation des données pose d'autres difficultés parce que les
utilisateurs ont en réalité beaucoup de mal à dépasser leurs pratiques
existantes et à revisiter leurs processus métiers pour imaginer comment ils
pourraient davantage tirer parti des données
et exprimer des besoins concrets et réalistes,
c'est-à-dire auxquels on puisse répondre.

J'ai voulu construire ce projet de thèse autour de cet enjeu en choisissant de
travailler sur le suivi de la consommation d'espaces agricoles,
naturels et forestiers pour les problèmes qu'il soulève en terme
de sélection d'indicateurs utiles et d'accompagnement méthodologique des services de l'État
et des collectivités territoriales.

En s'intéressant à la modélisation des dynamiques spatiales comme mode de
mobilisation des données géographiques, ce projet est un prétexte pour
travailler sur les usages de l'information spatialisée et développer un peu
plus la chaîne de valeur aval, en faisant appel à des méthodes en
mathématiques appliquées et en sciences sociales complémentaires des
compétences dont je dispose déjà.

C'est en même temps le choix d'une thématique importante pour le ministère et
ses services déconcentrés, inscrite dans les dernières lois de programmation
agricole, alors que l'observatoire national de la consommation d'espaces
naturels, agricoles et forestiers (OENAF) peine encore à définir ses
objectifs et à mettre en place les outils nécessaires.