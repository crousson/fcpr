\section
{Inscription du projet dans le parcours professionnel}

\subsection
{Présentation du parcours professionnel}

Mon parcours à l'Office national des forêts (ONF)
puis à la Sous-direction des systèmes d'information (SDSI)
m'a amené à me spécialiser dans les systèmes d'information
et plus spécifiquement d'information géographique.

\startitemize[n]

\item j'ai effectué mon stage de fin d'études à Epping Forest
      où j'ai travaillé à la mise en place d'inventaires patrimoniaux
      et à leur organisation dans un système d'information
      ouvert sur Internet ; première expérience de la forêt
      périurbaine dans un contexte favorable à l'expérimentation ;

\item je rejoins ensuite fin 2002 l'ONF
      à l'agence de Créteil puis à l'agence de Fontainebleau
      où j'exerce les fonctions de chef de bureau d'études, de responsable
      des projets d'aménagements forestiers et de responsable
      de la commercialisation du bois. J'ai mis en place
      des contrats innovants avec les ETF pour l'exploitation
      en forêt périurbaine et développé une méthode d'estimation
      des coupes vendues sur pied qui s'appuyait sur la cartographie
      des données historiques de commercialisation des bois ;

\item après un passage d'un an au Centre de recherche en géomatique
      de l'université Laval au Québec, j'ai rejoint en 2008 le ministère
      chargé de l'Agriculture à la Sous-direction des systèmes 
      d'information pour prendre la responsabilité
      d'un département d'ingénierie spécialisé dans
      l'information géographique. Mon arrivée à la SDSI
      coïncidait avec le début de la réorganisation territoriale
      des services de l'État et les travaux de convergence
      interministérielle.

\stopitemize

Deux objectifs principaux ont été poursuivis depuis 2008 :

\startitemize

\item la mise en place d'une infrastructure
      de données géographiques nationale pour répondre aux besoins
      des services déconcentrés et aux exigences
      de la directive INSPIRE ;
      
\item le développement d'un cadre technique
      qui réduise les coûts et les risques liés
      à l'intégration de la composante spatiale
      dans les projets informatiques nationaux,
      projet mené avec succès mais
      qui a essentiellement bénéficié aux projets du MEDDE
      sous maîtrise d’œuvre MAAF.

\stopitemize

La spécificité de ce parcours est d'avoir investi autant dans
des compétences de chef de projet que dans une très bonne maîtrise
des outils géomatiques et des méthodes d'ingénierie informatique,
avec une double compétence fonctionnelle
et technique qui n'est pas toujours bien valorisée dans une dichotomie
culturelle qui oppose maîtrise d'ouvrage et maîtrise d’œuvre.

Ce parcours m'a permis de bien connaître les enjeux et les attentes
des services déconcentrés, de nouer des contacts avec les principaux
acteurs nationaux de l'information géographique.

% développer l'intérêt pour l'innovation et les méthodes agiles

\subsection
{Motivations du projet de thèse}

Ce projet de formation complémentaire par la recherche 
participe de la logique d'un parcours de spécialiste
des systèmes d'information environnementaux
et constitue un moyen de concrétiser et de faire reconnaître
l'expertise acquise.

Il s'inscrit dans le
prolongement de mon expérience professionnelle à l'ONF d'abord,
à la SDSI ensuite.

Cette expérience m'incite à aller chercher des outils intellectuels
complémentaires pour enrichir la contribution que je peux apporter
dans le domaine de l'information spatialisée et des systèmes
d'information environnementaux,
dans le cadre de l'appui aux politiques publiques et dans la perspective
d'un renouvellement institutionnel qui refonde la gouvernance
des problématiques environnementales et accorde une importance
grandissante au partage de l'information entre de nombreux acteurs.

Il est aussi motivé par un contexte (Équipex GéoSUD, pôle Théia, programme
européen Copernicus, programme Sentinel de l’ESA, etc.) qui est très favorable
au développement des usages des données d’observation de la Terre en appui aux
politiques publiques agricoles et environnementales
\cite[centre_detudes_et_de_prospective_recours_2014].

L'objectif de ce projet est donc aussi de me positionner par rapport à ce
contexte qui justifie d’accompagner davantage utilisateurs et décideurs
pour les aider à développer à bon escient leurs usages
de l'information géographique et à tirer parti des données disponibles
et des nouvelles modalités d'organisation et d'action
permises par les technologies numériques.

\subsection
{Constats et questionnements tirés de l'expérience}

Cette analyse est développée plus complètement en \in{annexe}[constats].
Pour des raisons de brièveté, seulement les points principaux
en sont exposés ici.

La stratégie d’information géographique nationale reste difficile à définir :
c'est un sujet que l'administration centrale a beaucoup des difficultés à prendre en compte
alors qu'il interpelle principalement les services déconcentrés.
Les objectifs divergents, mais rarement analysés comme tels, des deux niveaux
font bien sûr penser à l'analyse de Pornon (1998).    %% TODO Référence manquante

Les projets nationaux et interministériels comme Géo-IDE
(infrastructure de données géographiques nationale)
restent principalement pilotés par des considérations techniques
et méconnaissent les besoins des utilisateurs,
certes difficiles à capturer.

La diminution importante des moyens des services déconcentrés,
le fonctionnement en « silos » verticaux,
la délégation du SI Économie agricole à l'ASP
mais aussi la gouvernance difficile de la mutualisation interministérielle
autour des infrastructures de données géographiques
concourent à une situation où l'information géographique,
au delà des besoins de la PAC,
n'est pas une priorité et disparaît des stratégies nationales
du ministère.

Pourtant, dans le même temps que l'État se veut plateforme,
les collectivités territoriales réclament davantage de moyens
d'observation du territoire \cite[rousset_roles_2015].

Le MEDDE à travers le SINP a davantage anticipé ce virage
et continue à mobiliser des moyens importants dévolus à la connaissance
territoriale au niveau des DREAL. Il se heurte cependant lui aussi
au paradoxe de la sous-utilisation des données.

Le paradoxe est que les acteurs publics disposent aujourd'hui de beaucoup de données dont
ils ne savent souvent pas quoi faire, alors qu'ils ne
réussissent pas à obtenir l'information utile qui leur permettrait d'agir
efficacement dans un monde complexe. L'efficacité doit être
ici comprise comme la capacité à adapter l'action à ses objectifs, à apprendre
des erreurs et trouver des modes d'action organisée plus résilients.

Pourtant, de nouveaux indicateurs sont proposés incessamment.
Lesquels sont vraiment utilisés ?

Que peut-on conclure des remarques précédentes ?

\startitemize

\item d'abord, on peut remarquer les limites d'un progrès
      technologique insuffisamment adossé à une réflexion sur les usages.

\item ensuite, du point de vue de l'usage des données,
      on peut s'interroger sur l'insuffisance fondamentale
      de toute démarche de définition et de production
      d'indicateurs lorsqu'elle ne s'appuie pas sur une démarche
      de modélisation.

\stopitemize


\subsection
{Justification de la conception générale du projet}

J'ai voulu construire ce projet de thèse autour de
l'enjeu des usages de l'information en choisissant de
travailler sur le suivi de la consommation d'espaces agricoles,
naturels et forestiers pour les problèmes qu'il soulève en terme
de sélection d'indicateurs utiles et d'accompagnement méthodologique des utilisateurs
(services de l'État et des collectivités territoriales).

En s'intéressant à la modélisation des dynamiques spatiales comme mode de
mobilisation des données géographiques, ce projet est un prétexte pour
travailler sur les usages de l'information spatialisée et développer un peu
plus la chaîne de valeur aval, en faisant appel à des méthodes en
mathématiques appliquées et en sciences sociales complémentaires des
compétences dont je dispose déjà.

C'est en même temps le choix d'une thématique importante pour le ministère et
ses services déconcentrés, inscrite dans les dernières lois de programmation
agricole, alors que l'observatoire national de la consommation d'espaces
naturels, agricoles et forestiers (OENAF) peine encore à définir ses
objectifs et à mettre en place les outils nécessaires.