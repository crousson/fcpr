\section
{Inscription du projet dans le parcours professionnel}

\subsection
{Présentation du parcours professionnel}

Mon parcours à l'Office national des forêts (ONF)
puis à la Sous-direction des systèmes d'information (SDSI)
m'a amené à me spécialiser dans la conception et la mise en oeuvre
de systèmes d'information et plus spécifiquement d'information géographique.

Ce parcours m'a permis de bien connaître les enjeux et les attentes
des services déconcentrés et de nouer des contacts avec les principaux
acteurs nationaux de l'information géographique.
Sa spécificité est de m'avoir permis d'investir autant dans
des compétences de chef de projet que dans une excellente maîtrise
des outils géomatiques et des méthodes d'ingénierie informatique,
avec une double compétence fonctionnelle
et technique.
Cette double compétence reste difficile à valoriser du fait d'une dichotomie
culturelle qui oppose maîtrise d'ouvrage et maîtrise d’œuvre,
administration centrale et services de terrain.

\startitemize[n]

\item mon stage de fin d'études à Epping Forest
      a été une première expérience de la forêt
      périurbaine dans un contexte favorable à l'expérimentation ;
      mon projet de stage consistait à mettre en place des inventaires patrimoniaux
      et à organiser les informations collectées dans un système d'information
      ouvert sur Internet ;

\item j'ai rejoint ensuite l'ONF
      où j'ai exercé, à l'agence de Créteil puis à l'agence de Fontainebleau,
      les fonctions de chef de bureau d'études, de responsable
      des projets d'aménagements forestiers et de responsable
      de la commercialisation du bois. J'ai mis en place
      des contrats innovants avec les ETF pour l'exploitation
      en forêt périurbaine et développé une méthode d'estimation
      des coupes qui s'appuyait sur la cartographie
      des données de martelage et de commercialisation
      récupérées sur de vieilles bandes de sauvegarde ;

\item les difficultés que rencontrait alors l'ONF
      à développer la composante spatiale de son
      système d'information métier, et les oppositions
      difficiles à résoudre entre les besoins de la direction générale
      et les attentes du terrain, m'ont motivé à rejoindre
      le Centre de recherche en géomatique de l'université Laval au Québec
      pour un projet de recherche qui devait être au départ
      centré sur les problématiques de l'ONF ;

\item ce projet n'a pas pu aboutir,
      et après un passage d'un an au CRG qui m'a permis
      d'avoir une première expérience de la recherche et de
      compléter ma formation sur les SIG, j'ai rejoint en 2008 le ministère
      chargé de l'Agriculture à la Sous-direction des systèmes 
      d'information pour prendre la responsabilité
      d'un département d'ingénierie spécialisé dans
      l'information géographique (DIG).

\stopitemize

Mon arrivée à la SDSI
coïncidait avec le début de la réorganisation territoriale
des services de l'État et les travaux de convergence
interministérielle.
Le DIG était une structure atypique qui répondait à mes attentes
dans la mesure où il apportait une réponse au problème que j'avais identifié à l'ONF :
petite unité agile au contact des utilisateurs en services déconcentrés,
le DIG était l'endroit idéal pour réaliser des projets pragmatiques à la frontière
entre maîtrise d'oeuvre et maîtrise d'ouvrage,
en mêlant expertise des usages et expertise technique.

Trois objectifs principaux ont été poursuivis depuis 2008 :

\startitemize

\item la mise en place d'une infrastructure
      de données géographiques nationale pour répondre aux besoins
      des services déconcentrés et aux exigences
      de la directive INSPIRE ;

\item le développement des usages de l'information géographique,
      aussi bien dans les systèmes d'information métiers
      qu'au niveau des services déconcentrés ;

\item la maîtrise des coûts et des risques
      liés à l'intégration de la composante spatiale
      dans les projets informatiques métiers.

\stopitemize

Le bilan du DIG sur la période 2008-2014 est incontestablement positif,
bien qu'il ait essentiellement contribué à l'interministériel
et aux projets du SI Eau commandités par le MEDDE
mais sous maîtrise d’œuvre MAAF et que les objectifs partagés à l'interministériel
ne'aient pas été atteints de manière satisfaisante.
La décision de sa suppression
en 2014 résulte d'un désintérêt de l'administration centrale
pour les systèmes d'information territoriaux et environnementaux, des difficultés
de la gouvernance interministérielle, de la forte diminuation
des moyens d'accompagnement consacrés aux services déconcentrés,
et d'un déficit important de culture de l'innovation et de la gestion
de projet agile au sein du ministère. La disparition du DIG
laisse aujourd'hui la SDSI sans structure opérationnelle pour
répondre efficacement aux besoins des directions métiers
et des services déconcentrés, alors que ni les programmes interministériels
ni les opérateurs comme l'IGN ne sont en mesure de le faire.

% développer l'intérêt pour l'innovation et les méthodes agiles

\subsection
{Motivations du projet de thèse}

Ce projet de formation complémentaire par la recherche 
participe de la logique d'un parcours de spécialiste
des systèmes d'information environnementaux
et constitue un moyen de concrétiser et de faire reconnaître
l'expertise acquise.

Il s'inscrit dans le
prolongement de mon expérience professionnelle à l'ONF d'abord,
à la SDSI ensuite.

Cette expérience m'invite à aller chercher des outils intellectuels
complémentaires pour enrichir la contribution que je peux apporter
dans le domaine de l'information spatialisée et des systèmes
d'information environnementaux.

% dans le cadre de l'appui aux politiques publiques et dans la perspective
% d'un renouvellement institutionnel qui refonde la gouvernance
% des problématiques environnementales et accorde une importance
% grandissante au partage de l'information entre de nombreux acteurs.

Il est aussi motivé par un contexte (Équipex GéoSUD, pôle Théia, programme
européen Copernicus, programme Sentinel de l’ESA, etc.) qui est favorable
au développement des usages des données d’observation de la Terre en appui aux
politiques publiques agricoles et environnementales
\cite[centre_detudes_et_de_prospective_recours_2014],
dans le contexte
d'un renouvellement institutionnel qui refonde la gouvernance
des problématiques environnementales et accorde une importance
grandissante au partage de l'information et aux systèmes d'information
environnementaux.

L'objectif de ce projet est donc aussi de me positionner par rapport à ce
contexte qui justifie d’accompagner davantage utilisateurs et décideurs
pour les aider à développer à bon escient leurs usages
de l'information géographique et à tirer parti des données disponibles
et des nouvelles modalités d'organisation et d'action
permises par le numérique et les systèmes d'information.

\subsection
{Constats et questionnements tirés de l'expérience}

Cette analyse est développée plus complètement en \in{annexe}[constats].
Pour des raisons de brièveté, seulement les points principaux
en sont exposés ici.

La stratégie d’information géographique du ministère reste difficile à définir au niveau national :
l'administration centrale éprouve beaucoup des difficultés à prendre en compte ce sujet
alors qu'il interpelle principalement les services déconcentrés.
Les objectifs divergents, mais rarement analysés comme tels, des deux niveaux
rappellent l'analyse qu'en a faite Pornon (1998).    %% TODO Référence manquante

Les projets nationaux et interministériels comme Géo-IDE
(infrastructure de données géographiques nationale)
restent principalement pilotés par des considérations techniques
et méconnaissent largement les besoins des utilisateurs,
difficiles à capturer, tandis qu'une gestion de projet inadaptée,
avec des cycles très longs,
désespère les attentes des utilisateurs.

La diminution importante des moyens des services déconcentrés,
le fonctionnement en « silos » verticaux,
la délégation du SI Économie agricole à l'ASP
mais aussi la gouvernance difficile de la mutualisation interministérielle
autour des infrastructures de données géographiques
concourent à une situation où l'information géographique
n'est pas une priorité et disparaît des stratégies nationales
du ministère.

Pourtant, dans le même temps que l'État se veut plateforme,
les collectivités territoriales réclament davantage de moyens
d'observation du territoire \cite[rousset_roles_2015].

Le MEDDE à travers le \SINP\ a davantage anticipé ce virage
et continue à mobiliser des moyens importants dévolus à la connaissance
territoriale au niveau des DREAL. Il se heurte cependant lui aussi
au paradoxe de la sous-utilisation des données.

Le paradoxe est que les acteurs publics ont aujourd'hui facilement accès à
beaucoup de données dont ils ne savent souvent pas quoi faire, alors qu'ils ne
réussissent pas à obtenir l'information utile qui leur permettrait d'agir
efficacement, à la bonne échelle. Des indicateurs toujours plus nombreux
sont proposés, mais finalement très peu sont utilisables ou utilisés.

% L'efficacité doit être
% ici comprise comme la capacité à adapter l'action à ses objectifs, à apprendre
% des erreurs et trouver des modes d'action organisée plus résilients.

D'abord, il faut remarquer les limites d'un progrès
technologique insuffisamment adossé à une réflexion sur les usages.

Ensuite, du point de vue de l'usage des données,
il est nécessaire s'interroger sur l'insuffisance fondamentale
de toute démarche de définition et de production
d'indicateurs lorsqu'elle ne s'appuie pas sur une démarche
de modélisation.

Cette attention apportée aux usages est devenue nécessaire pour identifier de
nouveaux gisements de valeur et améliorer la production de connaissance
dans la chaîne aval du traitement de l'information, qui reste
insuffisamment développée et ne prend pas assez en compte les utilisateurs
finaux.

\stopitemize


\subsection
{Justification de la conception générale du projet}

J'ai voulu construire ce projet de thèse autour de
l'enjeu des usages de l'information en choisissant de
travailler sur le suivi de la consommation d'espaces agricoles,
naturels et forestiers pour les problèmes qu'il soulève en terme
de sélection d'indicateurs utiles et d'accompagnement méthodologique des utilisateurs
(services de l'État et collectivités territoriales).

En s'intéressant à la modélisation des dynamiques spatiales comme mode de
mobilisation des données géographiques, ce projet est un prétexte pour
travailler sur les usages de l'information spatialisée et développer un peu
plus la chaîne de valeur aval, en faisant appel à des méthodes en
mathématiques appliquées et en sciences sociales complémentaires des
compétences dont je dispose déjà.

C'est en même temps le choix d'une thématique importante pour le ministère et
ses services déconcentrés, inscrite dans les dernières lois de programmation
agricole, alors que l'observatoire national de la consommation d'espaces
naturels, agricoles et forestiers (OENAF) peine encore à définir ses
objectifs et à mettre en place les outils nécessaires.