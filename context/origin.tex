\section{Pourquoi un projet FCPR ?}


\subsection
{Une évolution à la fois dans la continuité et en rupture avec mes acquis
professionnels}

Ce projet de formation complémentaire par la recherche s'inscrit dans le
prolongement d'une première expérience professionnelle et est le fruit d'une
prise de recul par rapport aux obstacles auxquelles je me suis confronté
pendant cette expérience et qui m'incitent à aller chercher des outils
intellectuels complémentaires pour enrichir la contribution que je peux
apporter dans le domaine de l'information spatialisée.

Il est aussi motivé par un contexte (Équipex GéoSUD, pôle Théia, programme
européen Copernicus, programme Sentinel de l’ESA, etc.) qui est très favorable
au développement des usages des données d’observation de la Terre en appui aux
politiques publiques agricoles, d’aménagement du territoire et de
développement durable \cite[centre_detudes_et_de_prospective_recours_2014].

L'objectif de ce projet est donc aussi de me positionner par rapport à ce
contexte qui justifie d’accompagner davantage utilisateurs et décideurs
publics pour les aider à tirer parti du progrès technique en innovant sur les
usages de l'information géographique.

Enfin, ce projet participe de la logique d'un parcours d'expert et constitue
un moyen de concrétiser et de faire reconnaître l'expertise acquise.


\subsection
{Un projet dans la continuité de l'expertise développée à la SDSI}

Ce projet s'inscrit dans la continuité d'un parcours qui m'a amené à me
spécialiser dans les systèmes d'information environnementaux autour d'une
motivation principale : mieux mobiliser l'information et ses technologies au
service de la gestion durable des ressources naturelles et du territoire.

Après une expérience de terrain en contexte périurbain à l’Office national des
forêts, mon parcours professionnel m'a permis de développer des compétences
dans trois domaines, à l'intersection l'un de l'autre et fortement
complémentaires :

\startitemize

\item la géomatique, c'est-à-dire les technologies du traitement de l'information
spatiale, et plus largement les technologies du numérique ;

\item la gestion de projet complexe à travers la gestion de projet informatique ;

\item un troisième domaine qu'on peut rattacher à la gestion de l’innovation, et qui
a à voir à la fois avec la maîtrise des risques dans les projets de systèmes
d'information et la maîtrise des usages.

\stopitemize

La spécificité de ce parcours est d'avoir investi autant dans des compétences
managériales, la conduite de projet, que dans des compétences résolument
techniques, une combinaison qui est encore perçu comme peu lisible dans une
dichotomie culturelle qui oppose inutilement maîtrise d'ouvrage et maîtrise
d’œuvre, décideurs et exécutants, solutions et problèmes.

Il m’a permis de bien connaître les enjeux et les attentes des services
déconcentrés, de nouer des contacts avec les principaux acteurs nationaux de
l’information géographique au niveau de la sphère État, et aussi de prendre du
recul pour tirer quelques constats de l’expérience que je détaille brièvement
ci-après.


\subsection
{Le paradoxe de la sous-utilisation des données}

Le paradoxe est que nous disposons aujourd'hui de beaucoup de données dont
nous ne savons souvent pas quoi faire, alors qu'en même temps, nous ne
réussissons pas à obtenir l'information utile qui nous permettrait d'agir
efficacement dans un monde de plus en plus complexe. L'efficacité doit être
ici comprise comme la capacité à adapter l'action à ses objectifs, à apprendre
des erreurs et trouver des modes d'action organisée plus résilients.

Il y a eu de nombreux investissements, pour acquérir et produire des données
d'observation de la Terre. Cela se traduit par l'accroissement régulier du
nombre de capteurs et une augmentation quasi exponentielle de la quantité de
données disponibles.

Néanmoins, cette accroissement de l'offre n'induit proportionnellement qu'un
faible accroissement des usages, ce qui traduit les limites d'un progrès
technologique insuffisamment adossé à une réflexion sur les usages et les
conditions du transfert opérationnel.

L'intérêt pour les données, en particulier dans les applications
environnementales et de connaissance du territoire, existe cependant chez les
acteurs publics sans qu'ils soient pour autant prêts à payer pour utiliser des
données qu'ils ont en pratique beaucoup de mal à exploiter. D'où un décalage
important entre le potentiel de ces données et leur valeur réelle d’usage.

Les plateformes, comme c'est le cas dans la démarche GéoSUD/Théia, poursuivent
l'objectif de :

\startitemize

\item lever le verrou de l'accès aux données ;

\item créer des « écosystèmes » données-utilisateurs-infrastructures à l'intérieur
desquels l'expérimentation et l'innovation sont facilitées et permettent de
développer les usages.

\stopitemize

Cette attention apportée aux usages est devenue nécessaire pour identifier de
nouveaux gisements de valeur et améliorer la capacité de traitement de
l'information et de production de connaissance dans la chaîne aval, qui reste
insuffisamment développée et ne prend pas assez en compte les utilisateurs
finaux.

Des progrès dans l'accès et le partage de l'information géographique pour les
services déconcentrés

Fin 2006, quand je préparais mon séjour au Centre de recherche en géomatique à
l'Université Laval à Québec, je me suis intéressé aux systèmes d'information
géographique du point de vue du rapport entre le progrès technique et les
problématiques organisationnelles ; je notais à partir de mon expérience à
l'Office  national des forêts que les freins à un élargissement de
l'utilisation de l'information géographique dans les outils de gestion de
l'établissement trouvaient leurs causes dans :

\startitemize

\item la difficulté à acquérir les données nécessaires, et le coût de ces
acquisitions

\item des méthodes et des processus mal adaptés aux spécificités de l'information
géographique

\item un cloisonnement organisationnel défavorable à la circulation de l'information
en rendant difficiles les échanges internes et interinstitutionnels de
données, et plus généralement défavorable à l'innovation, à la transformation
des processus.

\stopitemize

En 2015, presque dix ans plus tard, ce constat ne serait plus exactement le
même parce que la problématique des données a favorablement évolué : les
données de base disponibles sont beaucoup plus facilement disponibles et
accessibles, et l'idée que les données publiques sont un bien commun a
progressé avec la culture de l'open data, ce qui bénéficie d'abord à
l'administration elle-même. Mais les aspects méthodologiques et
organisationnels restent sinon à peu près les mêmes.

Entre 2007 et 2015, on peut néanmoins noter que des progrès significatifs ont
été enregistrés :

\startitemize

\item les infrastructures de données géographiques ont acquis en maturité et
réussissent à entretenir des partenariats importants entre acteurs publics, en
particulier au niveau régional, bien qu'elles restent principalement portées
par les services de l'État ;

\item d'autres plateformes se mettent en place et permettent d'accéder dans de
meilleures conditions à un bouquet de plus en plus large de données
spatialisées et d'observation de la Terre.

\stopitemize

À l'échelle ministérielle et interministérielle, ces progrès se sont traduits
par des avancées principalement techniques, mais aussi par des évolutions
institutionnelles, notamment en ce qui concerne l'accès et le partage des
données :

\startitemize

\item les deux ministères techniques en charge du territoire ont mis en place,
malgré les difficultés propres à l’interministérialité, une infrastructure de
données géographiques nationale commune (Géo-IDE) pour répondre, quoique
encore imparfaitement, aux exigences de la directive INSPIRE et de l'open data
;

\item à la SDSI, nous avons aussi beaucoup investi dans la maîtrise des risques et
des coûts liés aux projets informatiques, pour prendre en compte l'information
spatiale dès la conception des projets, avec un travail important sur les cas
d'usage, le recensement des patrons applicatifs, l'ergonomie applicative et
l'urbanisation du système d'information ;

\item la gratuité des données du référentiel géographique à grande échelle (RGE)
pour un usage dans le cadre d'une mission de service public sécurise l'accès à
des données de base et garantit l'utilisation de référentiels communs entre
acteurs du service public ;

\item le GéoPortail s'est transformé pour faciliter l'accès aux données de l'IGN et
leur utilisation dans les systèmes d'information métier, en même temps que le
ministère réussit mieux à collaborer avec l'IGN, ce qui lui permet de
bénéficier de son savoir-faire.

\stopitemize

Malgré ces progrès, et dans un contexte où la problématique d'acquisition et
d'accès aux données a significativement évolué, des verrous persistent.

Les services déconcentrés régionaux et départementaux se retrouvent de plus en
plus souvent en situation d’arbitres, en particulier lors de l’élaboration des
documents de planification. Ils sont pour cette raison les plus sensibilisés à
l’utilisation des outils de l’information spatiale et à une approche
géographique du territoire. Ils ont adhéré en très grand nombre à la démarche
de l’Equipex GéoSUD. Individuellement, leurs moyens d’action sont de plus en
plus limités. Ils sont donc en attente d’une plus grande harmonisation des
outils et souhaitent que les savoir-faire développés localement soient
davantage partagés.

\subsection
{Des verrous persistants qui demandent à être levés par des approches
innovantes}

On peut analyser les verrous qui subsistent autour de trois enjeux principaux
:

\startitemize

\item l'enjeu de la transversalité, qui est de mieux faire circuler l'information en
se libérant des contraintes des « silos » existants ;

\item l'enjeu de l'accompagnement méthodologique, qui est d'homogénéiser le niveau
des pratiques et d'assurer un transfert suffisant vers l'opérationnel ;

\item l'enjeu de valorisation des données, qui est de créer davantage de valeur en
répondant à des usages pertinents adaptés aux besoins des utilisateurs et de
la société.

\stopitemize

Les constats que je peux tirer de mon expérience à l'ONF et au ministère
chargé de l'agriculture par rapport à ces trois enjeux sont détaillés en
annexe.

En ce qui concerne les deux premiers enjeux, les plateformes qui se
développent à différents niveaux ont considérablement amélioré les conditions
d'accès aux données, tout en créant des écosystèmes de réutilisation
favorables à l'accompagnement et aux développements méthodologiques. Même s'il
reste des difficultés de mise en œuvre qui sont principalement
organisationnelles, la voie est tracée.

L'enjeu de valorisation des données pose d'autres difficultés parce que les
utilisateurs ont en réalité beaucoup de mal à dépasser leurs pratiques
existantes et à revisiter leurs processus métiers pour imaginer comment ils
pourraient mieux tirer parti des données et exprimer des besoins concrets et
réalistes, c'est-à-dire auxquels on puisse répondre.

J'ai voulu construire ce projet de thèse autour de cet enjeu en choisissant de
travailler sur un besoin encore mal défini, celui du suivi de la consommation
d'espaces agricoles, naturels et forestiers.

En s'intéressant à la modélisation des dynamiques spatiales comme mode de
mobilisation des données géographiques, ce projet est un prétexte pour
travailler sur les usages de l'information spatialisée et développer un peu
plus la chaîne de valeur aval, en faisant appel à des méthodes en
mathématiques appliquées et en sciences sociales complémentaires des
compétences dont je dispose déjà.

C'est en même temps le choix d'une thématique importante pour le ministère et
ses services déconcentrés, inscrite dans les dernières lois de programmation
agricole, alors que l'observatoire national de la consommation d'espaces
naturels, agricoles et forestiers (ONCENAF) peine encore à définir ses
objectifs et à mettre en place les outils nécessaires.