\subsection{Hypothèses théoriques :
   importance de l'espace, phénomènes de polarisation
   et interactions globales-locales}

\startitemize[n]

\item	L'utopie urbaine de \citet{howard_tomorrow_2003} fait de la cité-jardin ou de la ville-campagne
	l'organisation urbaine idéale.
	Cette utopie, qui peut être considérée comme le modèle limite de l'organisation urbaine polycentrique,
	correspond à une vision aspatiale (ou post-spatiale)
	de la société, où les contraintes liées à l'espace sont abolies ;
	son caractère utopique met en perspective
	l'importance de l'espace dans l'organisation urbaine
	et la structuration du territoire.

\item 	Les forces économiques, à travers les mécanismes
	du marché foncier et de la propriété privée,
	sont importantes et déterminent en grande partie
	les configurations spatiales de l'utilisation du sol
	dans la frange périurbaine ;
	mais elles ne suffisent pas à rendre compte
	de la complexité des interactions qui sont en jeu.

\item	L'organisation urbaine comme l'organisation
	des usages agricoles résultent
	de forces centrifuges (par ex. externalités négatives
	liées à la concentration urbaine) et centripètes
	(modèle gravitaire, modèle centre-périphérie)

	On peut rendre compte, au moins en partie, de ces forces
	par une double polarisation :

	\startitemize[a,packed]

    \item 	polarisation centre-périphérie
     	dans les préférences des habitants de la ville,
		qui préfèrent pour certains la ville dense,
		pour d'autres une ville plus étalée, en interface avec des
		espaces agricoles et naturels pourvoyeurs d'aménités ;

    \item	polarisation local-global dans les usages agricoles,
		en fonction du niveau d'adaptation de ces usages
		à l'influence urbaine, par ex. des cultures maraîchères
		destinées à des circuits de proximité ou une parcelle
		de grande culture destiné au marché mondial.

	\stopitemize

	% Une manière de représenter cette polarisation comme variable
	% des dynamiques spatiales
	% est d'utiliser les champs moyens \cite[sanders_models_2010].

\item	Les interactions spatiales entre ville et agriculture
	ne résultent pas seulement de la distance géographique
	mais aussi d'interactions globales / locales \cite[seto_urban_2012].
	Cette proximité fonctionnelle s'inscrit
	dans la participation à des réseaux (filière, circuit de commercialisation, etc.)
	qui dépassent la seule échelle locale,
	et dépend aussi de facteurs institutionnels.
	De ce point de vue, la notion de circuit court peut s'entendre en terme de transport
	(kilomètres parcourus entre le lieu de production et le consommateur)
	mais aussi en nombre d'intermédiaires.

\stopitemize