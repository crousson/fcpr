% \writetolist[section]{}{Introduction}
\section
{Introduction}

Ce document présente un projet de thèse préparé avec l’{\EVS} et l’{\TETIS}
pour lequel je sollicite un poste FCPR auprès du Ministère chargé de
l’Agriculture.

J’ai choisi de construire ce projet autour de la consommation d’espaces
naturels et agricoles et des relations entre ville et agriculture
parce qu’il s’agit d’un enjeu prioritaire pour le ministère
et que cet enjeu s'inscrit également dans la réflexion non moins importante
sur la transition énergétique.
Le projet est conçu comme une contribution aux outils et aux méthodes
de l’ingénierie territoriale et s’intéresse en particulier à la modélisation
et à la simulation numérique comme supports de la construction
des politiques d'aménagement.

Tout d’abord, je présente les motivations qui me conduisent à présenter un
projet FCPR  en cours de carrière et en quoi celui-ci s’inscrit dans l'évolution logique
de mon parcours professionnel (\in[origin]). À cette
occasion, je tire quelques constats de mon expérience sur lesquels
je m'appuie pour justifier un investissement complémentaire dans la recherche
scientifique.

Ensuite, je décris le contexte du projet à parti d’une analyse de la
problématique de l’étalement urbain (\in[contexte]). À partir des contradictions
de l'étalement urbain, j'explicite le problème que j'envisage de traiter (\in[probleme]),
les verrous scientifiques et méthodologiques à lever (\in[verrous]) et je justifie
le recours à la modélisation comme méthode de recherche (\in[methodo]).

À partir de ces éléments, j'introduis le sujet de thèse choisi
en précisant les questions de recherche et les objectifs,
le cadre conceptuel et les choix méthodologiques (\in[sujet]),
ainsi que les activités et les jalons du projet (\in[planning]).
L'encadrement de la thèse et l'environnement dans lequel ce travail
sera réalisé sont ensuite présentés (\in[environnement]).

Pour finir, j’indique quels sont les résultats attendus (\in[resultats]) et j’envisage
pour conclure les valorisations possibles de ce projet dans la suite de mon
parcours professionnel (\in[suites]).