% \writetolist[section]{}{Introduction}
\section
{Introduction}

Ce document présente un projet de thèse préparé avec l’{\UMR\ \EVS} et l’{\UMR\ \TETIS}
pour lequel je sollicite un poste {\FCPR} auprès du Ministère chargé de
l’Agriculture.
Le projet est conçu comme une contribution aux outils et aux méthodes
de l’ingénierie territoriale et s’intéresse en particulier à la modélisation
et à la simulation numérique comme supports de la construction
des politiques d'aménagement, en prenant pour point de départ
la problématique de la consommation d’espaces
naturels et agricoles et les relations entre ville et agriculture.

Tout d’abord, je présente les motivations qui me conduisent à présenter un
projet {\FCPR}  en cours de carrière et en quoi celui-ci s’inscrit dans l'évolution logique
de mon parcours professionnel (\in[origin]). À cette
occasion, je tire quelques constats de mon expérience sur lesquels
je m'appuie pour justifier un investissement complémentaire dans la recherchet
et ses méthodes.

Ensuite, je décris le contexte du projet à partir d’une analyse de la
problématique de l’étalement urbain (\in[contexte]). Constatant les contradictions
de l'étalement urbain, j'explicite le problème que j'envisage de traiter (\in[probleme]),
les verrous scientifiques et méthodologiques à lever (\in[verrous]) et je justifie
le recours à la modélisation comme méthode de recherche (\in[methodo]).

À partir de ces éléments, je présente le sujet de thèse (\in[sujet]),
ainsi que les activités et les jalons du projet (\in[planning])
et les résultats attendus (\in[resultats]).
L'encadrement de la thèse et l'environnement dans lequel ce travail
sera réalisé sont ensuite détaillés (\in[environnement]).

Pour finir, j’envisage les valorisations possibles de ce projet dans la suite de mon
parcours professionnel (\in[suites]).