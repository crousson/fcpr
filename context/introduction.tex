\section
{Introduction}

Ce document présente un projet de thèse préparé avec l’UMR EVS et l’UMR TETIS
pour lequel je sollicite un poste FCPR auprès du Ministère chargé de
l’Agriculture.

J’ai choisi de construire ce projet autour de la consommation d’espaces
agricoles, naturels et forestiers parce qu’il s’agit d’une préoccupation
importante pour le ministère et ses services déconcentrés, inscrite dans les
dernières lois de programmation agricole. Le projet est conçu comme une
contribution aux outils et aux méthodes de l’ingénierie territoriale et
s’intéresse en particulier à la modélisation et à la simulation numérique
comme supports de la géoprospective.

Tout d’abord, je présente les motivations qui me conduisent à présenter un
projet FCPR et en quoi celui-ci s’inscrit dans une évolution à la fois dans la
continuité et en rupture avec mon expérience professionnelle (1). À cette
occasion, je propose quelques constats tirés de cette expérience et qui me
paraissent justifier un investissement complémentaire dans la recherche
scientifique.

Ensuite, je décris le contexte du projet à parti d’une analyse de la
problématique de l’étalement urbain (2) puis je développe une présentation
succincte des travaux existants sur la mesure de la consommation d’espaces
agricoles et naturels (3) et j’explique le rapport entre la géoprospective, la
modélisation en géographie et les méthodes de l’analyse spatiale (4).

À partir de ces éléments, je détaille la question de recherche de ce projet de
thèse et les choix disciplinaires et méthodologiques que sont faits (5) avant
de préciser le cadre conceptuel utilisé (6), les étapes du projet de recherche
(7) et les terrains envisagés pour appuyer ce travail (8). L’environnement
dans lequel ce travail sera réalisé est ensuite présenté, ainsi que
l’encadrement de la thèse (9).

Pour finir, j’indique quels sont les résultats attendus (10) et j’envisage
pour conclure les valorisations possibles de ce projet dans la suite de mon
parcours professionnel (11).