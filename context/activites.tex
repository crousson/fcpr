\section{Activités de la thèse}

\subject{1ère année}

\startitemize[n]

\item	Revue bibliographique des travaux
	portant sur les interactions ville-agriculture
	afin de :

	\startitemize[a,packed]

	\item dégager une typologie des fonctions remplies par l'agriculture urbaine
	\item identifier les interactions entre ville et agriculture
	  et les modèles d'action possibles
	\item préciser des scénarios possibles d'évolution
	  des rapports entre ville et agriculture.

	  \stopitemize

\item	Méthode de description des zones {\em grises} périurbaines
	(zones qui ne sont recensées comme ni agricoles ni urbanisées
	dans les bases de données administratives)
	Cartographie des fonctions de l'agriculture urbaine
	et des variables identifiées comme forces motrices
	dans le processus d'étalement urbain.
	Expérimentation d'un dispositif collaboratif
	pour remplir les {\em trous}.

\item Précision des choix de modélisation
   pour la suite du travail de recherche

\item Rédaction d'un article pour une revue internationale
   qui présentera le projet de thèse,
   les choix épistémologiques et les premiers résultats.

% \stopitemize

\subject{2ème année}

% \startitemize[n][start=5]

\item	Modélisation individu-centrée (SMA) :
	mettre en évidence, avec des hypothèses simplificatrices,
	les effets d'auto-organisation et de couplage
	qui peuvent résulter de la compétition pour l'occupation du sol
	entre deux populations d'agents (citadins / agriculteurs)
	en fonction des tropismes (polarisation) de ces agents.

\item	Analyse spatiale des relations entre étalement urbain
	et usages agricoles
	Modélisation sous forme de graphe des relations fonctionnelles
	et des interactions inter-échelles

\item	Rédaction d'un article pour chaque type de modélisation

% \stopitemize

\subject{3ème année}

% \startitemize[n][start=8]

\item	Couplage entre modélisation individu centrée
	et modélisation en réseau à l'aide de
	la plateforme de modélisation Ocelet

\item	Comparaison des deux cas d'étude
	à travers les résultats de la modélisation

\item	Article de synthèse

\item	Rédaction finale de la thèse

\stopitemize