\section{Activités de la thèse}

\subject{1ère année}

\startitemize[n]

\item	La première étape d'analyse bibliographique de la thèse
	consistera à recenser les cas d'études existants en France et à l'international
	s'intéressant aux contradictions spatiales entre la limitation de l'étalement urbain
	et la préservation d'espaces naturels et agricoles ;

	cette méta-analyse permettra de :

	\startitemize[a,packed]

	\item dégager une typologie des fonctions remplies par l'agriculture urbaine
	\item analyser les systèmes de règles (règles d'urbanisme,
		normes environnementales) mises en place par les acteurs locaux ;
		\item définir des scénarios possibles d'évolution
	  des rapports entre ville et agriculture.

	\stopitemize

\item	En préparation du travail de modélisation,
	un diagnostic des relations entre ville et agriculture
	sera établi sur le premier cas d'études.

	Ce diagnostic doit préciser les enjeux locaux
	et les vulnérabilités des différents espaces
	à l'artificialisation à court et moyen terme.

	Cette étape de diagnostic nécessitera de caractériser les zones
	qui ne sont recensées comme ni agricoles ni urbanisées
	dans les bases de données administratives,
	mais aussi de préciser les fonctions remplies par les différents espaces
	et la nature des relations entre espaces.

	Les informations manquantes perçues comme utiles
	seront identifiées à cette étape et des pistes seront recherchées
	pour compléter les données disponibles.

\item Rapidement, dès le début de la thèse,
	les choix de modélisation pour la suite du travail de recherche
	seront précisés.

\item Un article sera rédigé pour une revue internationale
   et présentera le projet de thèse,
   les choix épistémologiques et de modélisation envisagée,
   ainsi que les résultats de la phase de diagnostic.

% \stopitemize

\subject{2ème année}

% \startitemize[n][start=5]

\item   Les données rassemblées pendant le diagnostic
	seront complétées par un travail de terrain
	autant que nécessaire.

	Cette étape pourrait être l'occasion d'expérimenter un dispositif collaboratif
	pour compléter les informations sur l'occupation du sol
	et les fonctions remplies par les différents espaces.

\item	L'analyse spatiale des données recueillies
	permettra de préciser qualitativement et quantitativement
	les relations entre les différentes variables et les règles appliquées
	par les acteurs locaux.
	
	Les relations fonctionnelles
	et des interactions inter-échelles seront modélisées
	sous la forme d'un graphe.

\item La modélisation multi-agents ou avec la plateforme Ocelet
	permettra de tester l'existence des couplages et des effets de polarisation
	dont nous avons fait l'hypothèse
	mais aussi l'influence des règles d'urbanisme et des normes environnementales.

	Les modèles seront exploités pour explorer
	plusieurs scénarios d'évolution des relations entre ville et
	agriculture pertinents dans le contexte du cas d'études.

\item Un article sera rédigé pour présenter
	les résultats de modélisation obtenus sur chacun des cas d'études sélectionnés.

% \stopitemize

\subject{3ème année}

% \startitemize[n][start=8]

\item La démarche sera reproduite à titre de validation
	sur le deuxième cas d'étude,
	et les résultats feront également l'objet d'une publication

\item La fin de la thèse sera consacrée à établir des comparaisons
	entre les deux cas d'études à travers la modélisation,
	à la rédaction d'un article de synthèse
	et à la rédaction finale de la thèse.

\stopitemize