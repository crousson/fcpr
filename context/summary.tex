{\ss\bf Résumé}

La préservation des espaces naturels, agricoles et forestiers
contre les effets de l'étalement urbain et de l'artificialisation
a été inscrite dans la Loi de modernisation agricole en 2010
et dans la Loi d'avenir pour l'agriculture en 2014.

La problématisation de l'étalement urbain ne fait cependant pas consensus.
Alors que des observatoires territoriaux
se mettent en place à l'échelle nationale, régionale et locale,
les indicateurs qui mesurent la consommation d'espaces sont contestés ou peu utiles.
La construction d'une vision partagée des enjeux restent difficiles
et les acteurs doivent prendre des décisions dans un univers controversé,
caractéristique de nombreuses problématiques environnementales.

L'information et la connaissance occupent une place importante
dans la mise en œuvre d'une gouvernance innovante, non seulement comme condition
de mise en œuvre de politiques normatives, mais aussi comme
alternatives possibles à ces politiques.

L'adaptation au changement climatique et la nécessité
de mieux valoriser les ressources territoriales sur le plan économique
et social dans un contexte de crises récurrentes
remettent en perspective les relations entre ville et agriculture.

Notre objectif est
i) d'identifier les processus territoriaux par lesquels
la proximité urbaine et les usages agricoles interagissent,
ii) d'en analyser les composantes afin de les expliciter
et iii) de mettre en évidence les effets de la relocalisation et de l'adaptation
des usages agricoles sur l'étalement urbain.

L'analyse spatiale et la modélisation pour représenter les processus en jeu
dans des modèles dynamiques explicitement spatialisés
susceptibles de servir de support à la médiation territoriale 
et aider à construire des scénarios d'urbanisation
plus économes en espaces agricoles et naturels.

En poursuivant cet objectif,
ce projet doit également permettre de sélectionner des systèmes d'indicateurs
spatialisés utiles et utilisables pour accompagner le débat
sur l'étalement urbain et contribuer aux méthodes opérationnelles
pour mieux mobiliser les données disponibles
(bases de données topographiques et administratives, données socioéconomiques,
imagerie satellitaire).

\blank[2*big]

{\ss\bf Mots-clés} :
dynamiques périurbaines, interactions ville-agriculture,
changements d'utilisation du sol, analyse spatiale,
modélisation des processus,
géoprospective, ingénierie territoriale, observatoires territoriaux