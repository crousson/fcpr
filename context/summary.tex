{\ss\bf Résumé}

La préservation des sols et des espaces naturels, agricoles et forestiers
contre les effets de l'étalement urbain et de l'artificialisation
a été inscrite dans la Loi de modernisation agricole en 2010
et dans la Loi d'avenir pour l'agriculture en 2014.

La problématisation de l'étalement urbain ne fait cependant pas consensus.
Au moment où des observatoires territoriaux
se mettent en place à l'échelle nationale, régionale et locale,
les indicateurs qui mesurent la consommation d'espaces sont contestés ou peu utiles.
La nécessité de mieux valoriser les ressources territoriales sur le plan économique
et social dans un contexte de crises économiques et sanitaires récurrentes,
mais aussi les nécessités de la transition énergétique et de l'adaptation au changement climatique
remettent en perspective les relations entre ville et agriculture,
alors que la construction d'une vision partagée des enjeux restent difficiles
et que les acteurs doivent prendre des décisions dans un univers controversé,
caractéristique de nombreuses problématiques environnementales.

Nous faisons l'hypothèse de couplages spatiaux et temporels non maîtrisés dans le processus
de consommation d'espaces agricoles et naturels
qui résultent de décisions insuffisamment coordonnées des acteurs territoriaux,
aboutissant à des contradictions aux échelles de niveau supérieur,
tandis que la mise en cohérence des politiques publiques
qui visent à limiter l'étalement urbain ne peut pas résulter uniquement
d'un renforcement des normes environnementales ou d'une reconcentration
du cadre de décision et de planification.

Notre objectif est i) de développer une méthode pour caractériser
l'évolution des espaces agricoles et naturels périurbains et l'adaptation
de ces espaces aux nouvelles demandes de proximité,
ii) de proposer un modèle d'observation des changements d'utilisation du sol
qui permette de les relier aux interactions multiples, proches et distantes,
qui en sont la cause et de rendre compte des éventuels couplages,
iii) de représenter ces interactions dans des modèles dynamiques explicitement spatialisés
susceptibles de servir de support à la médiation territoriale 
et d'aider à construire des scénarios d'urbanisation
plus économes en espaces agricoles et naturels, et globalement plus cohérents.

En poursuivant ces objectifs,
ce projet doit permettre de mieux mobiliser les données disponibles
(bases de données topographiques et administratives, données socioéconomiques,
imagerie satellitaire) afin de produire des indicateurs
utiles, utilisables et utilisés pour accompagner le débat
sur l'étalement urbain.

\blank[2*big]

{\ss\bf Mots-clés} :
dynamiques périurbaines, interactions ville-agriculture,
changements d'utilisation du sol, consommation d'espaces agricoles et naturels,
analyse spatiale, modélisation des processus,
géoprospective, ingénierie territoriale, observatoires territoriaux