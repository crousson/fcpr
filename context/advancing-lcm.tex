\section
{Analyse de la bibliographie de l’ouvrage 
« Advancing Land Change Modeling :
 Opportunities and Research Requirements »}

\cite[data][geographical_sciences_committee_advancing_2014]

337 articles cités dans 171 revues, 1000+ auteurs

\subsection{Résumé}

En réponse à une commande de l’USGS et de la NASA, ce rapport présente une
revue de la littérature scientifique et un état de l’art des recherches sur la
modélisation dans changements d’occupation et d’usage du sol (land change
model : LCM). Il présente enfin les perspectives et analyse les conditions
nécessaires aux progrès de la recherche. Deux phénomènes concourent à stimuler
la recherche sur les LCM :
a) l’augmentation en qualité et en quantité des
données disponibles, grâce aux nouveaux capteurs satellitaires notamment,
b)
la demande sociale pour des outils d’évaluation et de prédiction des
changements environnementaux qui doivent permettre d’informer les décisions
d’aménagement et les politiques publiques afin de répondre aux enjeux du
développement durable et mieux anticiper l’évolution des écosystèmes dont nous
dépendons.

Après avoir rappelé les concepts principaux liés au territoire et à
la modélisation des changements d’occupation et d’usage des sols, les
différentes approches de modélisation sont passées en revue et comparées, pour
conclure que si le choix de l’approche de modélisation doit se faire en
fonction  des objectifs, la représentation des processus présentent un
potentiel important pour rendre compte de la complexité des systèmes étudiés
et des interactions et rétro-actions entre la sphère biophysique et socio-
économique.

Plusieurs directions sont proposées pour dépasser les
connaissances et méthodes actuelles :

\startitemize[n]

\item la modélisation basée sur les processus,
la prise en compte des interactions inter-échelles, le couplage entre
différents modèles, y compris avec des modèles socio-économiques, le recours
aux méthodes d’optimisation et aux scénarios sont autant de pistes pour
améliorer la pertinence des modèles et permettre de couvrir différentes
échelles.

\item ces développements requiert d’être capable de traiter de plus grands
volumes de données et de combiner des sources de données hétérogènes. Ces
progrès méthodologiques nécessitent le développement d’infrastructures de
données et de calcul et en constituant des plateformes qui facilitent le
partage des modèles et des outils et logiciels de modélisation.

\item la constitution d’observatoires doit permettre de produire davantage de
connaissance sur l’usage du sol et sur les interactions sociales à l’œuvre
dans l’évolution des territoires.

\item enfin, davantage d’efforts doivent porter
sur la validation et l’évaluation des modèles pour être capable de mieux
caractériser les incertitudes de ces modèles.

\subsection{Mots-clés}

Mots-clés des titres des article cités, sélectionnés avec la bibliothèque gramophone à partir de l’analyse des n-grams les plus fréquents :

high resolution satellite imagery, computable general equilibrium models, land
cover change, agent based models/modeling, volunteered geographic information,
(image) time series, land transformation model, resulting secondary lands,
transitions wood harvest, spatially explicit, global environmental change,
neural network, cellular automata, multi-agent, climate change, urban growth,
urban sprawl, case study, land change, open space, ecosystem services,
satellite imagery, carbon sequestration, high resolution, integrated
assessment, geographic information, remote sensing, decision making, urban
systems, night time (satellite) imagery, scenario analysis, habitat
fragmentation, chesapeake bay, change detection, multi-objective, data fusion,
land markets, multi-scale, multi-temporal, public policies, landscape,
simulation, dynamics, economics, scenarios

\subsection
{Auteurs les plus cités}

\startcolumns[n=3]
\startitemize[n,packed]

\item P. H. Verburg (21)
\item D. G. Brown (12)
\item B. C. Pijanowski (11)
\item E. G. Irwin (8)
\item A. Veldkamp (8)
\item K. C. Clarke (7)
\item B. L. Turner (7)
\item A. Plantinga (6)
\item T. P. Evans (6)
\item D. Lewis (6)
\item S. Polasky (6)
\item N. E. Bockstael (6)
\item D. C. Parker (6)
\item J. R. Eastman (5)
\item P. Waddell (5)
\item R. G. Pontius (5)
\item S. M. Manson (5)
\item J. Wu (5)
\item E. Nelson (4)

\stopitemize

(n) = occurrences
\stopcolumns

\subsection
{Revues les plus citées}

\startitemize[n,packed]

\item Remote Sensing of Environment (12)
\item International Journal of Geographical information Science (11)
% \item Washington, DC: The National Academies Press (9)
\item Proceedings of the National Academy of Sciences of the United States of America (9)
\item Journal of Environmental Economics and Management (9)
\item Landscape Ecology (8)
\item Ecological Modelling (8)
\item International Journal of Remote Sensing (8)
\item Science (7)
\item Environmental Modelling \& Software (6)
\item Annals of the Association of American Geographers (6)
\item Journal of Land Use Science (6)
\item Ecology and Society (5)
\item American Journal of Agricultural Economics (5)
\item Computers, Environment and Urban Systems (5)
\item Journal of Urban Economics (5)
\item Environmental Management (4)

\stopitemize

(n) = occurrences
