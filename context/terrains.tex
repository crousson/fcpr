\subsection{Terrains envisagés}

Deux terrains sont envisagés afin de i) vérifier la reproductibilité de la démarche,
en particulier par rapport aux données disponibles,
ii) permettre des comparaisons entre des situations différentes et identifier
d'éventuelles régularités et iii) tester l'opérationnalité des modèles proposés.

\subsubsection{Terrain initial : Bassin de Thau}

Le bassin de Thau est un territoire de 353 \unit{km2} autour de l’étang éponyme,
proche de Montpellier, bien connu de l’\umr\ \tetis,
pour lequel on dispose déjà de nombreuses données.

Il a connu une très forte urbanisation
au cours des 50 dernières années et regroupe 14 communes sur deux intercommunalités.
Le paysage agricole est composé pour partie de vignobles,
d’espaces agricoles « d’intérêt écologique » et d’espaces agricoles
productifs « à dynamiser » (céréales, monoculture de melons),
avec une forte diversification agricole déjà
engagée sur la commune de Villeveyrac.

Ayant adopté un SCoT début 2014, le syndicat mixte du bassin de Thau (SMBT) s’est donné à travers un
plan de gestion intégrée l’objectif de mieux maîtriser l’urbanisation
tout en « préservant les espaces agricoles et les paysages, menacés par une urbanisation
trop rapide pour conserver l’identité du territoire et le cadre de vie »
(SMBT, 2010 ; SMBT, 2014). La participation de la population, et la mise en place de nombreux
partenariats sont l’une des caractéristiques de fonctionnement revendiquée
tant par le syndicat que par ses partenaires \cite[maurel_apprentissage_2008].

Pour accompagner ce programme d'actions et évaluer en permanence
ses effets sur les dynamiques territoriales, le SMBT travaille avec l'appui de l'UMR TETIS
à la définition et à la mise en place d'un observatoire territorial basé
sur un ensemble d'indicateurs spatiaux.


\subsubsection{Terrain secondaire}

Un deuxième terrain sera choisi rapidement en Rhône-Alpes pour reproduire
et valider la démarche, en utilisant les critères de sélection suivants :

\startitemize[packed]

\item dynamique urbaine différente ;
\item typologie différente d’agriculture ou de paysage ;
\item enjeux particuliers identifiés dans les documents de planification ;
\item implication des acteurs du monde agricole, de la société civile
  et des collectivités territoriales.

\stopitemize

Il sera possible d'utiliser un terrain déjà exploité par l'UMR EVS (bassin versant de l'Yzeron,
St Étienne Métropole) ou de choisir un terrain dans le cadre du projet FRUGAL.