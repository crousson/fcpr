\section{Enjeux scientifiques et méthodologiques}

Développer la connaissance du phénomène d'artificialisation
et de consommations d'espaces naturels et agricoles, nécessite encore de :

\startitemize

% \item	développer des méthodes automatiques permettant
% 	d’exploiter en temps court les flux massifs d'images
% 	d'observation de la Terre et industrialiser les chaînes de production
% 	des cartographies d'occupation du sol ;

\item	proposer une classification détaillée
	des changements des surfaces artificialisées fondée sur l’utilisation
	de données multi-sources (capteurs optique/radar, données administratives
	et socio-économiques) ;

\item	développer des indicateurs pertinents pour le suivi des
	dynamiques territoriales et des changements d'occupation du sol ;

\item	qualifier les informations produites
	en évaluant les incertitudes des cartographies et des indicateurs ;

\item	développer des méthodes d’accompagnement
	pour permettre aux utilisateurs finaux (acteurs publics, société civile, citoyens) de s'approprier
	les nouveaux services informationnels.

\stopitemize

Si les données sur l'occupation du sol sont de plus en plus abondantes,
les bases de données administratives ne sont pas conçues pour
suivre les changements et sont largement incomplètes en zone périurbaine à
l'échelle à laquelle nous proposons de travailler :
d'une part, l'artificialisation diffuse et le développement pavillonnaire
avec des extensions inférieures à 25 ha en surface sont majoritairement
comptabilisées comme espaces agricoles par Corine Land Cover \cite[aubry_les_2015],
et d'autre part, les systèmes agricoles atypiques caractéristiques de ces zones
sortent souvent du cadre de la PAC et sont donc mal
observées par le Registre parcellaire graphique (RPG).

Les données décrivant le marché foncier sont difficilement accessibles
en raison de leur coût, et il faudra éventuellement trouver
des variables de substitution pour approcher les effets d'anticipation
et de spéculation sur la disponibilité du foncier.

Distinguer entre les diverses formes d'agriculture
demande de qualifier et de quantifier les fonctions
remplies par ces différentes formes et de les hiérarchiser en fonction
des parties prenantes et de leurs attentes  \cite[ba_diversite_2011].

La caractérisation de changements qui impliquent
des processus avec des chronologies
différentes donne une importance particulière
à la prise en compte de la temporalité, à la fois
dans la structuration des données utilisées pour l'analyse
et dans la modélisation des processus. C'est une des limites actuelles
des outils de modélisation utilisés dans les études
sur les changements d'utilisation du sol.