\section[contexte]{Contexte}

% Question de développement durable

La nécessité de préserver les espaces naturels, agricoles et forestiers
contre les effets de l'étalement urbain et de l'artificialisation
a été inscrite dans la Loi de modernisation agricole en 2010
et dans la Loi d'avenir pour l'agriculture en 2014.
Elle est prise en compte depuis une dizaine d'années
au niveau européen \cite[eea_urban_2006].

« Consommation d'espaces agricoles », « étalement urbain »,
« artificialisation des terres ou du sol » renvoient à une même problématique
sans être synonymes et traduisent des points de vue différents mais
complémentaires. L’étalement urbain définit une
problématique d’aménagement du territoire caractérisée par un changement
d’occupation du sol dû à l’urbanisation à la fois irréversible et plus rapide
que la croissance démographique.
De 1992 à 2004, ce sont entre 50 000 et 60 000 ha de terres
qui ont été artificialisées par an en France \cite[chakir_analyse_2006].
L’artificialisation progresse quatre fois plus rapidement que la
croissance démographique \cite[balny_proteger_2009].
La consommation d'espaces agricoles touche pour un tiers des sols de
très bonne qualité agronomique \cite[antoni_lartificialisation_2011].
Cette perte est mise en relation avec les réflexions sur une pénurie future de
ressources alimentaires et de terres cultivées, dans un contexte d’échanges
mondialisés, d’interdépendances multi-échelles et d'urgence climatique,
avec l’apparition depuis quelques années d’une plus grande volatilité sur les
marchés agricoles et de tensions sur l’accès au foncier.

L'adaptation au changement climatique et la nécessité
de mieux valoriser les ressources territoriales sur le plan économique
et social dans un contexte de crises récurrentes
(économique, sanitaire, foncière, etc.) remettent en perspective
les relations de proximité entre ville et agriculture
\cite[scalenghe_anthropogenic_2009, toth_impact_2012, mansfield_municipal_2013, salvati_spatial_2013].
Entre les deux extrêmes que sont les jardins collectifs et la grande exploitation agro-industrielle,
la proximité peut être le
lieu de formes intermédiaires et innovantes d’agriculture qui dans leur
rapport à la terre ne sont « pas seulement un mode de production économique
mais aussi un modèle social et culturel » \cite[wolton_territoire_2015].

% Plusieurs facteurs sont à prendre en compte pour analyser
% l'évolution des dynamiques périurbaines et du rapport
% entre villes et campagnes \cite[westhoek_scenario_2006, gauvrit_les_2009] :

% \startitemize[packed]

% \item	le renchérissement du coût des transports,
% 	comme impact de l'évolution du coût des énergies carbonées
% 	ou conséquences des politiques d'atténuation du changement climatique ;

% \item 	les conséquences d'une moindre croissance économique ;

% \item 	la transition vers une économie numérique
% 	qui diminue la dépendance (économique, culturelle, etc.)
% 	au centre (au grand pôle urbain) ;

% \item	l'évolution de la mobilité et des modes de consommation
% 	à travers l'évolution des modes de vie ;

% \item	l'évolution des politiques publiques agricoles
% 	et des échanges économiques globalisés.

% \stopitemize

% Besoin d'une gouvernance innovante

La planification reste la principale réponse des pouvoirs publics
au problème de l'étalement urbain,
tandis que l’accumulation des textes traduit paradoxalement
la difficulté à répondre aux enjeux \cite[souchard_sortie_2013].
Cependant, alors que la foi dans les vertus de la planification reste
culturellement bien ancrée en France, les intérêts contradictoires des acteurs
locaux compromettent souvent la cohérence voulue dans les \SCoT.
Le foncier agricole est davantage protégé par le code de
l’urbanisme, à travers une prise en compte des enjeux dans les plans
d’urbanisme, que par le code rural dont certains dispositifs, comme les zones
agricoles protégées (\ZAP), ne rencontrent pas l’adhésion des acteurs qui
pourraient les mettre en œuvre.
L’importance donnée à la propriété privée dans le droit français complique
singulièrement la gouvernance foncière et les possibilités d’intervention de
la puissance publique \cite[bertrand_terres_2013].
La fiscalité en vigueur encourage quant à elle
les mauvaises pratiques \cite[sainteny_letalement_2008].

Le contexte du débat sur la consommation d’espaces agricoles et naturels est
aussi celui de l’affaiblissement de la représentation de la profession
agricole, qui compte aujourd’hui moins de 2~\% de la population active.
C’est aussi, avec la réforme territoriale, celui de la remise en cause d’une
organisation institutionnelle favorable au monde agricole, tandis que l’État
aménageur disparaît au profit de l’État-plateforme \cite[viard_dire_2015]. Le
transfert fin 2014 de la gestion du \FEADER\ aux régions est à ce titre
significatif.

Le sol, qui se distingue tout en se confondant avec le foncier
selon le point de vue adopté, est désormais considéré par beaucoup comme une ressource territoriale
dont les choix d'utilisation et d'affectation devraient être une affaire collective
\cite[cese_bonne_2015, cgdd_propositions_2015].
Un nouveau bien commun se construit dès lors autour du sol et des usages agricoles
et appelle une gouvernance innovante,
afin de dépasser les limites de l’aménagement planifié et
les effets pervers du marché foncier difficilement encadrés par les
pouvoirs publics \cite[bertrand_quelle_2006].
L'information et la connaissance occupent une place importante
dans la mise en place et le fonctionnement de cette nouvelle gouvernance,
non seulement comme condition de mise en œuvre de politiques normatives,
mais aussi comme alternatives possibles à ces politiques
\cite[theys_gouvernance_2002]. Pour répondre à ce besoin, les observatoires territoriaux
se développent à différentes échelles (nationale, régionale, départementale, locale),
de manière parfois désordonnée \cite[feyt_les_2011],
et leur contenu reste largement à définir.

% Univers controversé

Le \citet{cgdd_urbanisation_2012} recense 82 indicateurs regroupés en 23 familles « pour explorer
les différentes facettes du sujet de l’étalement urbain ».
Cependant, les différentes sources de
données disponibles et les indicateurs qui en dérivent ne s'accordent pas sur la mesure du phénomène
\cite[maaf_panorama_2014]. À l’échelle infrarégionale, les données spatiales
disponibles dans les bases de données existantes ne donnent pas de réponse
fiable pour un pas de temps décennal \cite[bousquet_les_2013].

Au-delà des incertitudes des indicateurs,
l'étalement urbain ne se réduit pas à un problème quantitatif
\cite[charmes_artificialisation_2013, martin-scholz_quand_2013]
et sa problématisation fait bien moins consensus
que certains discours globalisants ne le laissent croire.
La construction scientifique et sociale
du problème l'emporte sur la perception directe des acteurs.
Nous retrouvons ici l'univers controversé caractéristique des problématiques
environnementales \cite[theys_gouvernance_2002].