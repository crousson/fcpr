\section{Contexte}

% \subsection{Une problématique de développement durable sujette à controverse}

La nécessité de préserver des espaces naturels, agricoles et forestiers
contre les effets de l'étalement urbain et de l'artificialisation
est prise en compte depuis une dizaine d'années au niveau européen \cite[eea_urban_2006].
Elle a été inscrite dans la Loi de modernisation agricole en 2010
et dans la Loi d'avenir pour l'agriculture en 2014.

De 1992 à 2004, ce sont entre 50 000 et 60 000 ha de terres
qui ont été artificialisées par an en France \cite[chakir_analyse_2006].
L’artificialisation progresse quatre fois plus rapidement que la
croissance démographique \cite[balny_proteger_2009].
La consommation d'espaces agricoles touche pour un tiers des sols de
très bonne qualité agronomique \cite[antoni_lartificialisation_2011].

En 2010, 46~\% des exploitations agricoles étaient situées
dans les couronnes périurbaines (selon la définition de l'INSEE).
C'est dans ces zones, qui concentrent de fait de nombreux enjeux,
que se produit l'essentiel de la consommation d'espaces naturels,
agricoles et forestiers.

Le \citet{cgdd_urbanisation_2012} recense 82 indicateurs regroupés en 23 familles « pour explorer
les différentes facettes du sujet de l’étalement urbain ».
Cependant, les différentes sources de
données disponibles et les indicateurs qui en dérivent ne s'accordent pas sur la mesure du phénomène
\cite[maaf_panorama_2014]. À l’échelle infrarégionale, les données spatiales
disponibles dans les bases de données existantes ne donnent pas de réponse
fiable pour un pas de temps décennal \cite[bousquet_les_2013].
% Enfin, les bases de données
% administratives ne sont pas conçues pour suivre le changement.

Au delà des incertitudes des indicateurs,
l'étalement urbain ne se réduit pas à un problème quantitatif
\cite[charmes_artificialisation_2013, martin-scholz_quand_2013]
et sa problématisation fait bien moins consensus
que certains discours globalisants ne le laissent croire.

La construction scientifique et sociale
du problème l'emporte sur la perception directe des acteurs.
Nous retrouvons ici l'univers controversé caractéristique des problématiques
environnementales \cite[godard_strategies_1993].

Le contexte est aussi celui de fortes évolutions insitutionnelles : réforme territoriale, nouvelle \PAC,
transfert du \FEADER\ aux régions, disparition de l'État-aménageur
au profit de « l'État-plateforme ».

La planification reste la principale réponse de l'État,
tandis que l’accumulation des textes traduit paradoxalement
la difficulté à répondre aux enjeux \cite[souchard_sortie_2013].

Le foncier agricole est davantage protégé par le code de
l’urbanisme, à travers une prise en compte des enjeux dans les plans
d’urbanisme, que par le code rural dont certains dispositifs, comme les zones
agricoles protégées (\ZAP), ne rencontrent pas l’adhésion des acteurs qui
pourraient les mettre en œuvre.

Cependant, alors que la foi dans les vertus de la planification reste
culturellement bien ancrée en France, les intérêts contradictoires des acteurs
locaux compromettent souvent la cohérence voulue dans les \SCoT.


\section{De nouvelles demandes de proximité et le besoin d'une gouvernance innovante}

Le sol, qui se distingue tout en se confondant avec le foncier
selon le point de vue adopté, est maintenant considéré par beaucoup comme une ressource territoriale
dont les choix d'utilisation et d'affectation devrait être une affaire collective
\cite[cese_bonne_2105, cgdd_propositions_2015].

L'adaptation au changement climatique et la nécessité
de mieux valoriser les ressources territoriales sur le plan économique
et social dans un contexte de crises récurrentes
(économique, sanitaire, foncière, etc.) remettent en perspective
les relations entre ville et agriculture
\cite[scalenghe_anthropogenic_2009, toth_impact_2012, mansfield_municipal_2013, salvati_spatial_2013].

Entre les jardins collectifs et la grande exploitation industrielle, la proximité peut être le
lieu de formes intermédiaires et innovantes d’agriculture qui dans leur
rapport à la terre ne sont « pas seulement un mode de production économique
mais aussi un modèle social et culturel » \cite[wolton_territoire_2015].

La gestion de ce nouveau bien commun qui se construit autour du sol et des usages agricoles
appelle une gouvernance innovante,
afin de dépasser les limites de l’aménagement planifié et
les effets pervers du marché foncier difficilement encadré par les
pouvoirs publics \cite[bertrand_quelle_2006].

L'information et la connaissance occupent par conséquent une place importante
dans la mise en place et le fonctionnement de cette nouvelle gouvernance,
non seulement comme condition de mise en œuvre de politiques normatives,
mais aussi comme alternatives possibles à ces politiques
\cite[theys_gouvernance_2002].

Pour répondre à ce besoin, les observatoires territoriaux
se développent à différentes échelles (nationale, régionale, départementale, locale),
de manière parfois désordonnée \cite[feyt_les_2011],
et leur contenu reste largement à définir.

Des exercices de prospective \cite[westhoek_scenario_2006, gauvrit_les_2009]
identifient plusieurs facteurs à prendre en compte pour analyser
les dynamiques périurbaines à venir et l'évolution du rapport
entre villes et campagnes :

\startitemize[packed]

\item	le renchérissement du coût des transports,
	comme impact de l'évolution du coût des énergies carbonées
	ou conséquences des politiques d'atténuation du changement climatique ;

\item 	les conséquences d'une moindre croissance économique ;

\item 	la transition vers une économie numérique
	qui diminue la dépendance (économique, culturelle, etc.)
	au centre (au grand pôle urbain) ;

\item	l'évolution de la mobilité et des modes de consommation
	à travers l'évolution des modes de vie ;

\item	l'évolution des politiques publiques agricoles
	et des échanges économiques globalisés.

\stopitemize