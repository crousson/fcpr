\section{Contexte}

La préservation des espaces naturels, agricoles et forestiers
contre les effets de l'étalement urbain et de l'artificialisation
a été inscrite dans la Loi de modernisation agricole en 2010
et dans la Loi d'avenir pour l'agriculture en 2014.

De 1992 à 2004, ce sont entre 50 000 et 60 000 ha de terres
qui ont été artificialisées par an en France \cite[chakir_analyse_2006].
L’artificialisation progresse quatre fois plus rapidement que la
croissance démographique \cite[balny_proteger_2009].
La consommation d'espaces agricoles touche pour un tiers des sols de
très bonne qualité agronomique \cite[antoni_lartificialisation_2011].

La planification reste la principale réponse de l'État,
tandis que l’accumulation des textes traduit paradoxalement
la difficulté à répondre aux enjeux \cite[souchard_sortie_2013].

Le contexte est aussi celui d'une forte évolution insitutionnelle : réforme territoriale, nouvelle PAC,
transfert du FEADER aux régions, disparition de l'État-aménageur
au profit de « l'État-plateforme ».
Dans ce contexte, les observatoires territoriaux
se multiplient à différentes échelles (nationale, régionale, départementale, locale),
de manière parfois désordonnée \cite[feyt_les_2011].

Le \citet{cgdd_urbanisation_2012} recense 82 indicateurs regroupés en 23 familles « pour explorer
les différentes facettes du sujet de l’étalement urbain ».
Cependant, les différentes sources de
données et indicateurs disponibles ne s'accordent pas sur la mesure du phénomène
\cite[maaf_panorama_2014]. À l’échelle infrarégionale, les données spatiales
disponibles dans les bases de données existantes ne donnent pas de réponse
fiable pour un pas de temps décennal \cite[bousquet_les_2013]. Les bases de données
administratives ne sont pas conçues pour suivre le changement.

Parallèlement, de nouvelles demandes sociales repositionnent l'agriculture
comme ressource territoriale pour répondre aux enjeux de la proximité
et du développement durable.
Entre les jardins
collectifs et la grande exploitation industrielle, la proximité peut être le
lieu de formes intermédiaires et innovantes d’agriculture qui dans leur
rapport à la terre ne sont « pas seulement un mode de production économique
mais aussi un modèle social et culturel » \cite[wolton_territoire_2015].

Un nouveau bien commun se construit autour du sol et des usages agricoles.
La gestion de ce bien commun appelle une gouvernance innovante,
afin de dépasser les limites de l’aménagement planifié et de l’intervention publique
sur les marchés fonciers \cite[bertrand_quelle_2006].