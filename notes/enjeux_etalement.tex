\subsection
{Les enjeux de l'étalement urbain}

À l'époque contemporaine, en France, d’autres dynamiques ont pour conséquence
des changements importants d'occupation du sol : a) la déprise
agricole qui restitue des terres autrefois cultivées à l'espace naturel et
forestier, b) l'évolution des pratiques agricoles sous l’influence conjuguée
du marché, des mesures agri-environnementales et des politiques de trame verte
et bleue, et enfin c) le changement climatique dont l'impact reste beaucoup
plus difficile à mesurer.

Le terme « étalement » traduit non seulement l'extension spatiale de la ville
mais sous-entend encore un envahissement non désiré du territoire, l’entropie
d’une urbanisation dysfonctionnelle et un changement qualitatif du tissu
urbain, comme si celui-ci se diluait dans l'espace et faisait tâche d’huile.

Le dynamisme nouveau des territoires ruraux renouvelle une population rurale
de moins en moins agricole, demandeuse d’aménités parfois contradictoires avec
l’usage agricole et l’évolution des systèmes agronomiques (Pistre, 2012). La
périurbanisation n’est plus seulement l’extension périphérique de la ville,
c’est aussi la ville qui s’invite à la campagne.

La problématique de l’étalement urbain reflète, d’une part, les préoccupations
des collectivités pour lesquelles l’étalement urbain a un coût social
éventuellement important et, d’autre part, celles du monde agricole pour qui
l’artificialisation représente une perte irréversible de foncier productif.
Elle reflète aussi, directement ou indirectement, les préoccupations des
citadins soucieux de la qualité de leur cadre de vie et davantage sensibilisés
au développement durable.

Si l'étalement urbain est d'abord une dynamique de changement d'occupation du
sol, sa problématisation ventriloque des préoccupations variées : sociales,
économiques, écologiques, foncières, urbanistiques.

----

Dans cette proximité, ce n’est pas tant la distance qu’une relation de
confiance qui compte. Celle-ci peut très bien se construire entre des
agriculteurs et des citadins qui ne seront pas forcément des voisins
\cite[vidal_entre_2011].

----

La gouvernance polycentrique est mise en avant,
dans le sillage des travaux d'Elinor Ostrom,
comme une voie possible pour gérer plus efficacement la complexité
des problèmes du développement durable.

----

Le comportement des acteurs, à l'échelle individuelle,
peut être représenté par un modèle de choix discret
en se fondant sur la théorie des utilités ou inclure
un modèle hédonique pour prendre en compte les externalités
environnementales positives et négatives (Irwin, 2001; Irwin, 2009).

----

importance des drivers économiques
théorie de l'organisation urbaine (Bailly, Wegener)
état de l'art LUCC (NSC, Magliocca)
approche statistique / fouille de données
cinétique/temporalité/facteurs économiques/facteurs institutionnels/facteurs sociologiques/facteurs culturels
-> inclut dans l'interaction spatiale
(bof) endogénéité des formes urbaines et des trajectoires de changement d'occupation du sol.
      (déjà contenu dans l'idée d'une interaction réciproque)

----

Quels sont les modèles d'action possibles
	pour répondre aux demandes sociétales d'une agriculture
	de proximité et de maîtrise de l'étalement urbain ?

----

Le polycentrisme institutionnel comme la théorie systémique
invitent à examiner le comportement individuel des acteurs
et les interactions multi-scalaires pour mieux comprendre
les processus territoriaux de l'étalement urbain.

----

Compris comme un modèle conceptuel et non pas comme un modèle opérationnel,
	il permet d'identifier les concepts opérants et les relations sémantiques entre eux
	en amont d'une démarche de modélisation.

----

la résurgence de l'intérêt pour ce modèle au XXIème siècle résulte d'un décalage entre
les aspirations d'une partie de la société
et la réalité urbaine contemporaine ;

----

Les chercheurs, indépendamment de leur inscription disciplinaire,
reconnaissent l'importance de la modélisation des processus
(Irwin & Geoghegan, 2001;
 Irwin, Jayaprakash, et Munroe, 2009;
 National Research Council, 2014;
 Magliocca, 2014;
 Chakir, 2015).

----

Le développement de modèles spatialement explicites
a été stimulé par l'abondance de données disponibles
à des échelles de plus en plus précises.

----

3ème objectif : modélisation interscalaire : mettre en relation
les différentes échelles de décision et modélisation les relations fonctionnelles
à distance