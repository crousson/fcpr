\subsection
{Les enjeux de l'étalement urbain}

À l'époque contemporaine, en France, d’autres dynamiques ont pour conséquence
des changements importants d'occupation du sol : a) la déprise
agricole qui restitue des terres autrefois cultivées à l'espace naturel et
forestier, b) l'évolution des pratiques agricoles sous l’influence conjuguée
du marché, des mesures agri-environnementales et des politiques de trame verte
et bleue, et enfin c) le changement climatique dont l'impact reste beaucoup
plus difficile à mesurer.

Le terme « étalement » traduit non seulement l'extension spatiale de la ville
mais sous-entend encore un envahissement non désiré du territoire, l’entropie
d’une urbanisation dysfonctionnelle et un changement qualitatif du tissu
urbain, comme si celui-ci se diluait dans l'espace et faisait tâche d’huile.

Le dynamisme nouveau des territoires ruraux renouvelle une population rurale
de moins en moins agricole, demandeuse d’aménités parfois contradictoires avec
l’usage agricole et l’évolution des systèmes agronomiques (Pistre, 2012). La
périurbanisation n’est plus seulement l’extension périphérique de la ville,
c’est aussi la ville qui s’invite à la campagne.

La problématique de l’étalement urbain reflète, d’une part, les préoccupations
des collectivités pour lesquelles l’étalement urbain a un coût social
éventuellement important et, d’autre part, celles du monde agricole pour qui
l’artificialisation représente une perte irréversible de foncier productif.
Elle reflète aussi, directement ou indirectement, les préoccupations des
citadins soucieux de la qualité de leur cadre de vie et davantage sensibilisés
au développement durable.

Si l'étalement urbain est d'abord une dynamique de changement d'occupation du
sol, sa problématisation ventriloque des préoccupations variées : sociales,
économiques, écologiques, foncières, urbanistiques.

----

Dans cette proximité, ce n’est pas tant la distance qu’une relation de
confiance qui compte. Celle-ci peut très bien se construire entre des
agriculteurs et des citadins qui ne seront pas forcément des voisins
\cite[vidal_entre_2011].